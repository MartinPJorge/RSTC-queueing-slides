\documentclass{upmassignment}
\usepackage[spanish]{babel}
\usepackage{ifthen}
\usepackage{amsmath}
\usepackage{amsfonts}



% Para mostrar/ocultar soluciones
\newboolean{show}
%\setboolean{show}{true}
\setboolean{show}{false}
\usepackage{environ}
\NewEnviron{solucion}{
  \ifshow
      \begin{answer}\BODY\end{answer}
  \fi}






\coursetitle{Creating assignments}
\courselabel{RSTC}
\exercisesheet{Suma telescópica}{Tema 5}
\student{\ }%
\semester{Segundo Semestre}
\date{\today}
\university{Universidad Politécnica de Madrid}
\school{Departamento de Ingeniería de Sistemas Telemáticos}
%\usepackage[pdftex]{graphicx}
%\usepackage{subfigure}


\setlength{\textwidth}{5.0in}
\linespread{1.3}
\renewcommand{\PB}{{\bfseries Problema}}















\begin{document}

A la suma
\begin{equation}
    \sum_{i=A}^{B}\rho^i
    = \rho^A+\rho^{A+1}+\rho^{A+2}+
    \ldots + \rho^{B-2} + \rho^{B-1}
    + \rho^B
\end{equation}
con $\rho<1$
se le llama suma telescópica.

Y se resuelve multiplicando y
dividiendo por $1-\rho$:
\begin{multline*}
    \frac{1-\rho}{1-\rho}
    \left(
    \rho^A+\rho^{A+1}+\rho^{A+2}+
    \ldots + \rho^{B-2} + \rho^{B-1}
    + \rho^B
    \right)\\
    =\frac{1}{1-\rho}
    (\rho^A-\rho^{A+1}
    +\rho^{A+1}
    -\rho^{A+2}
    +\rho^{A+2}
    -\rho^{A+3}
    +\rho^{B-2}
    -\rho^{B-1}
    +\rho^{B-1}
    -\rho^{B}
    +\rho^B
    -\rho^{B+1}
    )\\
    = \frac{\rho^A-\rho^{B+1}}{1-\rho}
\end{multline*}

Si tuviéramos $B\to\infty$
nos queda:
\begin{equation}
    \sum_{i=A}^\infty
    \rho^i=\frac{\rho^A}{1-\rho}
\end{equation}

Del mismo modo, teniendo
$A=0,B\to\infty$ queda
\begin{equation}
    \sum_{i=0}^\infty
    \rho^i=\frac{\rho^0}{1-\rho}
    =\frac{1}{1-\rho}
\end{equation}


\end{document}


