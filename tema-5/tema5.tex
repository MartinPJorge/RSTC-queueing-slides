\documentclass[xcolor={x11names}]{beamer}
\usetheme{Madrid}

\usepackage{amssymb}
\usepackage{ulem}
\usepackage[utf8]{inputenc}
\usepackage{mathtools}
\usepackage{multicol}
%\usepackage[x11names]{xcolor}
\usefonttheme{professionalfonts}


% Subfigures
\usepackage{caption}
\usepackage{subcaption}




% Change base colour beamer@blendedblue (originally RGB: 0.2,0.2,0.7)
\colorlet{beamer@blendedblue}{DarkSeaGreen4}







%% MATH commands
\DeclareMathOperator{\Var}{Var}


%% THEOREMS
%\newtheorem{theorem}{Theorem}
\newtheorem{thm}{Teorema}[section] % the main one
% Definición
%\theoremstyle{definition}
\newtheorem{definicion}{Definición}[section]
\newtheorem{lema}{Lema}[section]



%% PFGplots %%
\usepackage{pgfplots}

%% Exponential distribution
\pgfmathdeclarefunction{exponential}{1}{%
  \pgfmathparse{(#1)*exp(-#1*x)}%
}
\pgfmathdeclarefunction{exponentialcdf}{1}{%
  \pgfmathparse{1-exp(-#1*x)}%
}

%% Poisson distribution
\pgfmathdeclarefunction{poiss}{1}{%
  \pgfmathparse{(#1^x)*exp(-#1)/(x!)}%
}

%% Normal distribution (#1=mu, #2=sigma)
% John D. Cook approx. https://tex.stackexchange.com/a/124629
\pgfmathdeclarefunction{normalcdf}{2}{%
  \pgfmathparse{1/(1 + exp(-0.07056*((x-#1)/#2)^3 - 1.5976*(x-#1)/#2))}%
}




\newcommand{\red}[1]{{\color{red}#1}}
\newcommand{\blue}[1]{{\color{blue}#1}}

%%%%%%%%%%
%% TIKZ %%
%%%%%%%%%%
\usepackage{tikz}
\usepackage{animate}
\usetikzlibrary{positioning}
\usetikzlibrary{shapes,arrows, positioning, calc}
\usetikzlibrary{overlay-beamer-styles}
\usetikzlibrary{chains,shapes.multipart}
\usetikzlibrary{scopes}
\usetikzlibrary{automata}
\usetikzlibrary{positioning}  %                 ...positioning nodes
\usetikzlibrary{arrows}       %                 ...customizing arrows
\usetikzlibrary{intersections}


%%%%%%%%%
%% PGF %%
%%%%%%%%%
\usepgfplotslibrary{fillbetween}


%%% Insert section name before the section %%%
\AtBeginSection[]{
  \begin{frame}
  \vfill
  \centering
  \begin{beamercolorbox}[sep=8pt,center,shadow=true,rounded=true]{title}
    \usebeamerfont{title}\insertsectionhead\par%
  \end{beamercolorbox}
  \vfill
  \end{frame}
}



\title[Tema 5]{Tema 5: Introducción al Teletráfico\\y a la Teoría de Colas}
\subtitle{Redes y Servicios de Telecomunicaciones (RSTC)\\
Grado en Ingeniería de Tecnologías y Servicios de Telecomunicación}
%\author{M. Saiful Bari\inst{1} \and Mr X\inst{2}}

\author{Jorge Martín Pérez\inst{1}}
\institute{
    \inst{1}
    Departamento de Ingeniería Telemática, Universidad Politécnica de Madrid
}

\date{\today}







%%%%%%%%%%%%%%%%%%%%
%%% SLIDES START %%%
%%%%%%%%%%%%%%%%%%%%
\begin{document}


%%% TITLE %%%
\frame{\titlepage}


\begin{frame}{Contenido}
    \tableofcontents
\end{frame}




\section{Introducción}
\begin{frame}{\secname}
    La teoría de colas modela:
    \begin{itemize}
        \item colas de supermercado;
        \item colas en gasolineras;
        \item colas en taquillas; o
        \item \textbf{colas de routers}.
    \end{itemize}

    \vfill
    Nos interesa saber:
    \begin{itemize}
        \item ¿cuánto vamos a esperar?; o
        \item la probabilidad de que esté llena la cola.
    \end{itemize}
\end{frame}



\begin{frame}{\secname}
    \begin{figure}
        
\begin{tikzpicture}[start chain=going right,>=latex,node distance=0pt]
% the rectangular shape with vertical lines
\node[rectangle split, rectangle split parts=6,
draw, rectangle split horizontal,text height=1cm,text depth=0.5cm,on chain,inner ysep=0pt] (wa) {};
\fill[white] ([xshift=-\pgflinewidth,yshift=-\pgflinewidth]wa.north west) rectangle ([xshift=-15pt,yshift=\pgflinewidth]wa.south);

% the circle
\node[draw,circle,on chain,minimum size=1.5cm] (se) {$\mu$};

% the arrows and labels
\draw[->] (se.east) -- +(20pt,0);
\draw[<-] (wa.west) -- +(-20pt,0) node[left] {$\lambda$};
\node[align=center,below] at (wa.south) {Waiting \\ Area};
\node[align=center,below] at (se.south) {Service \\ Node};
\end{tikzpicture}

    \end{figure}

    En una cola:
    \begin{itemize}
        \item llegan $\lambda$ [usuarios/sec]
        \item hay $q=4$ usuarios encolados;
        \item hay $n=5$ usuarios en total; y
        \item se sirven $\mu$ [usuarios/sec].
    \end{itemize}
\end{frame}


\begin{frame}{\secname}
    Problema:
    \begin{itemize}
        \item las llegadas; y
        \item tiempos de servicio
    \end{itemize}
    son \textbf{aleatorios}.

    \vfill

    \textit{Ejemplo}: la persona que nos
    atiende en caja tarda más o menos
    dependiendo de como de cansada esté,
    o de cuánto tarde la pasarela de pago
    (aleatorio).
\end{frame}






\section{Distribución Exponencial}

\begin{frame}{\secname}
    \begin{figure}
        \begin{tikzpicture}
    % Time axis
    \draw[|->] (0,0) -- (5,0) node[anchor=west] {time [sec]};

    % ticks
    \draw (1,.05) -- (1,-.05) node[below] {1};
    \draw (2,.05) -- (2,-.05) node[below] {2};
    \draw (3,.05) -- (3,-.05) node[below] {3};
    \draw (4,.05) -- (4,-.05) node[below] {4};

    
    % arrivals
    \draw[->,color=red,thick] (0.2,1.4) node[anchor=west,rotate=70] {usuario 1} --(0.2,0) node[below,anchor=north] {\tiny 0.2};
    \draw[->,color=red,thick] (2.5,1.4)node[anchor=west,rotate=70] {usuario 2} --(2.5,0) node[below,anchor=north] {\tiny 2.5};

    \draw[<->,color=blue,thick] (0.2,.4) -- node[above,pos=0.5]{$t=1.3$ [sec]} (2.5,.4);
\end{tikzpicture}

    \end{figure}
    El tiempo entre los usuarios que
    llegan a la cola \blue{$t$} se puede
    modelar con la \textbf{distribución exponencial}.
\end{frame}



\begin{frame}{\secname}
    \begin{definicion}[Distribución exponencial]
        Se dice que una variable aleatoria
        continua $t\in\mathbb{N}$ sigue una
        distribución exponencial si su
        función de densidad es:
        \begin{equation}
            f_t(\tau) = \mathbb{P}(t=\tau) = \lambda e^{-\lambda \tau}
        \end{equation}
        donde $\lambda>0$ es el parámetro
        que caracteriza la distribución.
    \end{definicion}
\end{frame}




\begin{frame}{\secname: propiedades}
    \begin{figure}
        \begin{tikzpicture}

\begin{axis}[every axis plot post/.append style={
  mark=none,domain=0:3,samples=20},
  axis x line*=bottom,
  axis y line*=left,
  xlabel={$\tau$ [sec]},
  ylabel={$f_t(\tau)$},
  enlargelimits=upper,
  width=4in,
  height=2in]
  \addplot[ultra thick,smooth,color=DodgerBlue1] {exponential(1)} node [pos=0.3,anchor=south west] {$\lambda=1$};
  \addplot[ultra thick,smooth,color=DodgerBlue4] {exponential(0.5)} node [pos=0.25,anchor=north east] {$\lambda=\frac{1}{2}$};
\end{axis}


\end{tikzpicture}


    \end{figure}


    \begin{itemize}
        \item \textbf{media}: $\mathbb{E}[t]=\tfrac{1}{\lambda}$
        \item \textbf{varianza}: $\Var[t]=\tfrac{1}{\lambda^2}$
    \end{itemize}
\end{frame}




\begin{frame}{\secname: ejemplo gasolinera}
    \textit{Ejemplo}: el tiempo medio
    que pasa un coche en un surtidor es
    $\mathbb{E}[\blue{t}]=\tfrac{1}{\lambda}=2$ [min].
    Por tanto $\lambda=\tfrac{1}{2}$ [coches/min].

    \vfill


    \begin{figure}
     \centering
     \begin{subfigure}[b]{0.45\textwidth}
         \centering
         \resizebox{\textwidth}{!}{%
             \begin{tikzpicture}
    % Time axis
    \draw[|->] (0,0) -- (5,0) node[anchor=west] {time [min]};

    % ticks
    \draw (1,.05) -- (1,-.05) node[below] {1};
    \draw (2,.05) -- (2,-.05) node[below] {2};
    \draw (3,.05) -- (3,-.05) node[below] {3};
    \draw (4,.05) -- (4,-.05) node[below] {4};

    
    % arrivals
    \draw[->,color=red,thick] (0.2,1.4) node[anchor=west,rotate=70] {coche 1} --(0.2,0) node[below,anchor=north] {\tiny 0.2};
    \draw[->,color=red,thick] (2.2,1.4)node[anchor=west,rotate=70] {coche 2} --(2.2,0) node[below,anchor=north] {\tiny 2.2};

    \draw[<->,color=blue,thick] (0.2,.4) -- node[above,pos=0.5]{$t=2$ [min]} (2.2,.4);
\end{tikzpicture}
%
         }
         \caption{$\mathbb{P}(\blue{t=2})=\tfrac{1}{2}\ e^{-\frac{1}{2}\cdot \blue{2}}=0.18$}
     \end{subfigure}
     \hfill
     \begin{subfigure}[b]{0.45\textwidth}
         \centering
         \resizebox{\textwidth}{!}{%
             \begin{tikzpicture}
    % Time axis
    \draw[|->] (0,0) -- (5,0) node[anchor=west] {time [min]};

    % ticks
    \draw (1,.05) -- (1,-.05) node[below] {1};
    \draw (2,.05) -- (2,-.05) node[below] {2};
    \draw (3,.05) -- (3,-.05) node[below] {3};
    \draw (4,.05) -- (4,-.05) node[below] {4};

    
    % arrivals
    \draw[->,color=red,thick] (0.2,1.4) node[anchor=west,rotate=70] {coche 1} --(0.2,0) node[below,anchor=north] {\tiny 0.2};
    \draw[->,color=red,thick] (4.2,1.4)node[anchor=west,rotate=70] {coche 2} --(4.2,0) node[below,anchor=north] {\tiny 4.2};

    \draw[<->,color=blue,thick] (0.2,.4) -- node[above,pos=0.5,anchor=south]{$t=4$ [min]} (4.2,.4);
\end{tikzpicture}
%
         }
         \caption{$\mathbb{P}(\blue{t=4})=\tfrac{1}{2}\ e^{-\frac{1}{2}\cdot \blue{4}}=0.07$}
     \end{subfigure}
    \end{figure}

\end{frame}





\subsection{Propiedad sin memoria}


\begin{frame}{\secname: \subsecname}
Si ya han pasado {\color{DodgerBlue1}$s$ [sec]}, ¿cuál es
    la probabilidad de que tarde {\color{DodgerBlue4}$\tau$
    [sec]} más?:
    \begin{equation}
        \mathbb{P}(\red{t}>{\color{DodgerBlue1}s}+{\color{DodgerBlue4}\tau}|\ \red{t}>{\color{DodgerBlue1}s})
    \end{equation}

    \vfill

    \begin{figure}
        \begin{tikzpicture}

    % Already passed
    \draw[draw=none,fill=DodgerBlue1!20] (0.2,0) rectangle ++(1,1.4);

    % Time axis
    \draw[|->] (0,0) -- (5,0) node[anchor=west] {time [sec]};

    % ticks
    \draw (1,.05) -- (1,-.05) node[below] {1};
    \draw (2,.05) -- (2,-.05) node[below] {2};
    \draw (3,.05) -- (3,-.05) node[below] {3};
    \draw (4,.05) -- (4,-.05) node[below] {4};
    
    % arrivals
    \draw[->,color=red,thick] (0.2,2) node[anchor=west,rotate=70] {coche 1} --(0.2,0) node[below,anchor=north] {\tiny 0.2};
    \draw[-,color=DodgerBlue1,dashed] (1.2,1.4) --(1.2,0) node[below,anchor=north] {\tiny 1.2};
    \draw[-,color=DodgerBlue4,dashed] (4.2,1.4) --(4.2,0) node[below,anchor=north] {\tiny 4.2};

    \draw[->,color=red,thick,visible on=<2>] (4.5,2.2)node[anchor=west,rotate=70] {coche 2} --(4.5,0.2) ;

    \draw[<->,color=DodgerBlue1,thick] (0.2,.4) -- node[above,pos=0.48,anchor=south]{$s=1$} (1.2,.4);
    \draw[<->,color=DodgerBlue4,thick] (1.2,.4) -- node[above,pos=0.5,anchor=south]{$\tau=3$ [sec]} (4.2,.4);
    \draw[<->,color=red,thick, visible on=<2>] (0.2,1.7) -- node[above,pos=0.5,anchor=south]{$t$} (4.5,1.7);
\end{tikzpicture}

    \end{figure}
\end{frame}



\begin{frame}{\secname: \subsecname}
    Por la propiedad sin memoria de una
    exponencial tenemos que:
    \begin{equation}
        \mathbb{P}(\red{t}>{\color{DodgerBlue1}s}+{\color{DodgerBlue4}\tau}|\ \red{t}>{\color{DodgerBlue1}s}) = \mathbb{P}(\red{t}>{\color{DodgerBlue4}\tau})
    \end{equation}

    \vfill


    \begin{figure}
     \centering
     \begin{subfigure}[b]{0.45\textwidth}
         \centering
         \resizebox{\textwidth}{!}{%
             \begin{tikzpicture}

    % Already passed
    \draw[draw=none,fill=DodgerBlue1!20] (0.2,0) rectangle ++(1,1.4);

    % Time axis
    \draw[|->] (0,0) -- (5,0) node[anchor=west] {time [sec]};

    % ticks
    \draw (1,.05) -- (1,-.05) node[below] {1};
    \draw (2,.05) -- (2,-.05) node[below] {2};
    \draw (3,.05) -- (3,-.05) node[below] {3};
    \draw (4,.05) -- (4,-.05) node[below] {4};
    
    % arrivals
    \draw[->,color=red,thick] (0.2,2) node[anchor=west,rotate=70] {coche 1} --(0.2,0) node[below,anchor=north] {\tiny 0.2};
    \draw[-,color=DodgerBlue1,dashed] (1.2,1.4) --(1.2,0) node[below,anchor=north] {\tiny 1.2};
    \draw[-,color=DodgerBlue4,dashed] (4.2,1.4) --(4.2,0) node[below,anchor=north] {\tiny 4.2};

    \draw[->,color=red,thick] (4.5,2.2)node[anchor=west,rotate=70] {coche 2} --(4.5,0.2) ;

    \draw[<->,color=DodgerBlue1,thick] (0.2,.4) -- node[above,pos=0.48,anchor=south]{$s=1$} (1.2,.4);
    \draw[<->,color=DodgerBlue4,thick] (1.2,.4) -- node[above,pos=0.5,anchor=south]{$\tau=3$ [sec]} (4.2,.4);
    \draw[<->,color=red,thick] (0.2,1.7) -- node[above,pos=0.5,anchor=south]{$t$} (4.5,1.7);
\end{tikzpicture}
%
         }
         \caption{$\mathbb{P}(\red{t}>{\color{DodgerBlue1}s}+{\color{DodgerBlue4}\tau}|\ \red{t}>{\color{DodgerBlue1}s})$}
     \end{subfigure}
     \hfill
     \begin{subfigure}[b]{0.45\textwidth}
         \centering
         \resizebox{\textwidth}{!}{%
             \begin{tikzpicture}

    % Time axis
    \draw[|->] (0,0) -- (5,0) node[anchor=west] {time [sec]};

    % ticks
    \draw (1,.05) -- (1,-.05) node[below] {1};
    \draw (2,.05) -- (2,-.05) node[below] {2};
    \draw (3,.05) -- (3,-.05) node[below] {3};
    \draw (4,.05) -- (4,-.05) node[below] {4};
    
    % arrivals
    \draw[->,color=red,thick] (0.2,2) node[anchor=west,rotate=70] {coche 1} --(0.2,0) node[below,anchor=north] {\tiny 0.2};
    \draw[-,color=DodgerBlue4,dashed] (3.2,1.4) --(3.2,0) node[below,anchor=north] {\tiny 3.2};

    \draw[->,color=red,thick] (3.7,2.2)node[anchor=west,rotate=70] {coche 2} --(3.7,0.2) ;

    \draw[<->,color=DodgerBlue4,thick] (0.2,.4) -- node[above,pos=0.5,anchor=south]{$\tau=3$ [sec]} (3.2,.4);
    \draw[<->,color=red,thick] (0.2,1.7) -- node[above,pos=0.5,anchor=south]{$t$} (3.7,1.7);
\end{tikzpicture}
%
         }
         \caption{$\mathbb{P}(\red{t}>{\color{DodgerBlue4}\tau})$}
     \end{subfigure}
    \end{figure}
\end{frame}


\begin{frame}{\secname: \subsecname}
    \textit{Ejemplo}: en media el surtidor
    de una gasolinera está ocupado 5 [min].
    Si el surtidor lleva
    {\color{DodgerBlue1}$s=1$ [min]} ocupado,
    ¿cuál es la probabildad de que esté ocupado
    {\color{DodgerBlue4}$\tau=3$ [min]} más?

    \vfill

    Por la propiedad sin memoria tenemos:
    \begin{equation*}
        \mathbb{P}(\red{t}>{\color{DodgerBlue1}s}+{\color{DodgerBlue4}\tau}|\ \red{t}>{\color{DodgerBlue1}s}) = \mathbb{P}(\red{t}>{\color{DodgerBlue4}\tau}) = \mathbb{P}(\red{t}>{\color{DodgerBlue4}3}) = \frac{1}{5}e^{-\frac{1}{5}\cdot3} = 0.11
    \end{equation*}

\end{frame}




\subsection{Mínimo de variables exponenciales}

\begin{frame}{\secname: \subsecname}
    \textit{Ejemplo}: los compactos llegan
    a gasolinera con tasa
    {\color{Firebrick1}$\lambda_1=\tfrac{1}{4}$
    [coches/min]}, y los todoterreno con tasa
    {\color{Firebrick4}$\lambda_2=\tfrac{1}{8}$ [coches/min]}.

    \vfill

    ¿Con qué probabilidad llega
    un coche cualquiera en 3 [min]?
\end{frame}



\begin{frame}{\secname: \subsecname}
    \begin{lema}[Mínimo de v.a. exponenciales]
        Sean las v.a.\footnote{v.a. significa variable aleatoria} exponenciales
        {\color{Firebrick1}$t_1$} y
        {\color{Firebrick4}$t_2$}, con
        tasas
        {\color{Firebrick1}$\lambda_1$} y
        {\color{Firebrick4}$\lambda_2$}; la v.a.
        $t=\min\{{\color{Firebrick1}t_1},{\color{Firebrick4}t_2}\}$
        se distribuye como una v.a. exponencial
        de tasa
        {$\lambda=\color{Firebrick1}\lambda_1+ \color{Firebrick4}\lambda_2$}.
    \end{lema}

    \vfill

    \textit{Demostración}:
    \begin{multline*}
        \mathbb{P}(t>\tau) =
        \mathbb{P}({\color{Firebrick1}t_1}>\tau)\mathbb{P}({\color{Firebrick4}t_2}>\tau)
        = \left(\int_\tau^\infty {\color{Firebrick1}\lambda_1}e^{-{\color{Firebrick1}\lambda_1} t}\ dt \right)
        \left(\int_\tau^\infty {\color{Firebrick4}\lambda_2}e^{-{\color{Firebrick4}\lambda_2} t}\ dt \right)\\
        = e^{-{\color{Firebrick1}\lambda_1} \tau} e^{-{\color{Firebrick4}\lambda_2} \tau} = e^{-({\color{Firebrick1}\lambda_1}+{\color{Firebrick4}\lambda_2})\tau}
        = e^{-\lambda \tau}
    \end{multline*}
\end{frame}




\begin{frame}{\secname: \subsecname}
    \textit{Ejemplo}: los compactos llegan
    a gasolinera con tasa
    {\color{Firebrick1}$\lambda_1=\tfrac{1}{4}$
    [coches/min]}, y los todoterreno con tasa
    {\color{Firebrick4}$\lambda_2=\tfrac{1}{8}$ [coches/min]}.

    \vfill

    ¿Con qué probabilidad llega
    un coche cualquiera en 3 [min]?

    \begin{align*}
        1-\mathbb{P}(t>3)&=1-
        ({\color{Firebrick1}\lambda_1}+
        {\color{Firebrick4}\lambda_2})
        e^{-({\color{Firebrick1}\lambda_1}+
        {\color{Firebrick4}\lambda_2})\cdot 3}\\
        & =
        1 - \left({\color{Firebrick1}\frac{1}{4}}+
        {\color{Firebrick4}\frac{1}{8}}\right)
        e^{-({\color{Firebrick1}\frac{1}{4}}+
        {\color{Firebrick4}\frac{1}{8}})\cdot 3}
        = 0.12
    \end{align*}
\end{frame}







\subsection{Comparación de exponenciales}


\begin{frame}{\secname: \subsecname}
    \textit{Ejemplo}: los compactos llegan
    a gasolinera con tasa
    {\color{Firebrick1}$\lambda_1=\tfrac{1}{4}$
    [coches/min]}, y los todoterreno con tasa
    {\color{Firebrick4}$\lambda_2=\tfrac{1}{8}$ [coches/min]}.

    \vfill

    ¿Cuál es la probabilidad de que llegue
    antes un compacto, es decir,
    (${\color{Firebrick1}t_1}<
    {\color{Firebrick4}t_2}$)?

    \begin{figure}
        \begin{tikzpicture}
    % Time axis
    \draw[|->] (0,0) -- (5,0) node[anchor=west] {time [min]};

    % ticks
    \draw (1,.05) -- (1,-.05);
    \draw (2,.05) -- (2,-.05);
    \draw (3,.05) -- (3,-.05);

    
    % arrivals
    \draw[->,color=Firebrick1,thick] (1.2,1) node[anchor=west,rotate=30] {compacto} --(1.2,0);
    \draw[->,color=Firebrick4,thick] (2,1) node[anchor=west,rotate=30] {todoterreno} --(2,0);

    % times
    \draw[<->,color=Firebrick1,thick] (0,0.35) -- node[above] {$t_1$} (1.2,0.35);
    \draw[<->,color=Firebrick4,thick] (0,-0.35) -- node[below] {$t_1$} (2,-0.35);

\end{tikzpicture}

    \end{figure}

\end{frame}



\begin{frame}{\secname: \subsecname}
    \begin{lema}[Comparación de v.a. exponenciales]
        Sean las v.a. exponenciales
        {\color{Firebrick1}$t_1$} y
        {\color{Firebrick4}$t_2$}, con
        tasas
        {\color{Firebrick1}$\lambda_1$} y
        {\color{Firebrick4}$\lambda_2$}; se
        tiene que:
        \begin{equation}
            \mathbb{P}({\color{Firebrick1}t_1}<{\color{Firebrick4}t_2}) = \frac{{\color{Firebrick1}\lambda_1}}{{\color{Firebrick1}\lambda_1}+{\color{Firebrick4}\lambda_2}}
        \end{equation}
    \end{lema}

    \vfill

    \textit{Demostración}:
    \begin{equation*}
        \mathbb{P}({\color{Firebrick1}t_1}<{\color{Firebrick4}t_2})=
        \int_0^\infty \mathbb{P}({\color{Firebrick1}t_1}=t)\mathbb{P}({\color{Firebrick4}t_2}>t)\ dt = \int_0^\infty {\color{Firebrick1}\lambda_1} e^{-{\color{Firebrick1}\lambda_1} t} e^{-{\color{Firebrick4}\lambda_2} t}\ dt = \frac{{\color{Firebrick1}\lambda_1}}{{\color{Firebrick1}\lambda_1}+{\color{Firebrick4}\lambda_2}}
    \end{equation*}
\end{frame}





\begin{frame}{\secname: \subsecname}
    \textit{Ejemplo}: los compactos llegan
    a gasolinera con tasa
    {\color{Firebrick1}$\lambda_1=\tfrac{1}{4}$
    [coches/min]}, y los todoterreno con tasa
    {\color{Firebrick4}$\lambda_2=\tfrac{1}{8}$ [coches/min]}.

    \vfill

    ¿Cuál es la probabilidad de que llegue
    antes un compacto, es decir,
    (${\color{Firebrick1}t_1}<
    {\color{Firebrick4}t_2}$)?

    \begin{figure}
        \begin{tikzpicture}
    % Time axis
    \draw[|->] (0,0) -- (5,0) node[anchor=west] {time [min]};

    % ticks
    \draw (1,.05) -- (1,-.05);
    \draw (2,.05) -- (2,-.05);
    \draw (3,.05) -- (3,-.05);

    
    % arrivals
    \draw[->,color=Firebrick1,thick] (1.2,1) node[anchor=west,rotate=30] {compacto} --(1.2,0);
    \draw[->,color=Firebrick4,thick] (2,1) node[anchor=west,rotate=30] {todoterreno} --(2,0);

    % times
    \draw[<->,color=Firebrick1,thick] (0,0.35) -- node[above] {$t_1$} (1.2,0.35);
    \draw[<->,color=Firebrick4,thick] (0,-0.35) -- node[below] {$t_1$} (2,-0.35);

\end{tikzpicture}

    \end{figure}

    \begin{equation}
        \mathbb{P}({\color{Firebrick1}t_1}<{\color{Firebrick4}t_2}) = \frac{{\color{Firebrick1}\lambda_1}}{{\color{Firebrick1}\lambda_1}+{\color{Firebrick4}\lambda_2}} = 
        \frac{{\color{Firebrick1}\frac{1}{4}}}{{\color{Firebrick1}\frac{1}{4}}+{\color{Firebrick4}\frac{1}{8}}} = 0.67
    \end{equation}

\end{frame}







\section{Procesos de llegada de Poisson}


\subsection{Tiempos entre llegadas}

\begin{frame}{\secname: \subsecname}
    Buscamos una distribución que diga
    cómo de probable es que lleguen
    2 coches en 3 segundos:

    \vfill

    \begin{figure}
        \begin{tikzpicture}
    % Time axis
    \draw[|->] (0,0) -- (5,0) node[anchor=west] {time [sec]};

    % ticks
    \draw (1,.05) -- (1,-.05);
    \draw (2,.05) -- (2,-.05);
    \draw (3,.05) -- (3,-.05);

    
    % arrivals
    \draw[->,color=red,thick] (0.2,1) node[anchor=west,rotate=70] {usuario 1} --(0.2,0);
    \draw[->,color=red,thick] (0.8,1) node[anchor=west,rotate=70] {usuario 2} --(0.8,0);
    \draw[->,color=red,thick] (1.5,1)node[anchor=west,rotate=70] {usuario 3} --(1.5,0);

    \draw[<->,color=blue,thick] (0,-.35) -- node[anchor=north,pos=0.5]{$t=2$ [sec]} (2,-.35);
\end{tikzpicture}

    \end{figure}
\end{frame}



\begin{frame}{\secname: \subsecname}
    Si el tiempo entre llegadas es exponencial,
    sabemos la probabilidad de que lleguen
    ${\color{HotPink4}k=0}$
    coches en \red{$t=0.8$ [min]}.

    \vfill


    \begin{figure}
        \begin{tikzpicture}
    % Time axis
    \draw[|->] (0,0) -- (5,0) node[anchor=west] {time [min]};

    % ticks
    \draw (1,.05) -- (1,-.05);
    \draw (2,.05) -- (2,-.05);
    \draw (3,.05) -- (3,-.05);

    
    % arrivals
    % \draw[->,color=red,thick] (0.2,1) node[anchor=west,rotate=70] {coche 1} --(0.2,0);
    \draw[->,color=red,thick] (0.8,1) node[anchor=west,rotate=70] {coche 1} --(0.8,0) node[below ] {\tiny 0.8};
    \draw[->,color=red,thick] (2.5,1)node[anchor=west,rotate=70] {coche 2} --(2.5,0);

    \draw[|-|,color=HotPink4,thick] (0,-.55) -- node[anchor=north,pos=0.5]{$k=0$ coches} (0.8,-.55);

    \draw[<->,color=red,thick] (0,.45) -- node[anchor=south,pos=0.5]{$t$} (0.8,.45);
\end{tikzpicture}

    \end{figure}

    \vfill

    \begin{equation*}
        \mathbb{P}({\color{HotPink4}0\ \text{coches}}\ \text{en}\ \red{0.8 \text{ min}})=1-\mathbb{P}(\red{t>0.8})=1 - \lambda \red{0.8} e^{-\lambda \red{0.8}}
    \end{equation*}

\end{frame}



\subsection{Conteo}


\begin{frame}{\secname: \subsecname}
    Pero lo que queremos es contar
    el número de coches
    ${\color{HotPink4}N}(\blue{t})={\color{HotPink4}2}$
    que llegan en \blue{$t=3$ [min]},
    y saber qué probabilidad tiene
    $\mathbb{P}({\color{HotPink4}N}(\blue{t})={\color{HotPink4}2})$
    \begin{figure}
        \begin{tikzpicture}
    % Time axis
    \draw[|->] (0,0) -- (5,0) node[anchor=west] {time [min]};

    % ticks
    \draw (1,.05) -- (1,-.05);
    \draw (2,.05) -- (2,-.05);
    \draw (3,.05) -- (3,-.05);

    
    % arrivals
    % \draw[->,color=red,thick] (0.2,1) node[anchor=west,rotate=70] {coche 1} --(0.2,0);
    \draw[->,color=red,thick] (0.8,1) node[anchor=west,rotate=70] {coche 1} --(0.8,0) node[below ] {\tiny 0.8};
    \draw[->,color=red,thick] (2.5,1)node[anchor=west,rotate=70] {coche 2} --(2.5,0);

    \draw[<->,color=blue,thick] (0,-.45) -- node[anchor=north,pos=0.5]{$t=3$ [min]} (3,-.45);
    \draw[|-|,color=HotPink4,thick] (0,-1.2) -- node[anchor=north,pos=0.5]{$N(t)=2$ [coches]} (3,-1.2);
\end{tikzpicture}

    \end{figure}

\end{frame}






\begin{frame}{\secname: \subsecname}
    \begin{definicion}[Distribución de Poisson]
        Un proceso de llegadas
        ${\color{HotPink4}N}(\blue{t})$ con
        tasa \red{$\lambda$} es
        de Poisson si el tiempo entre
        llegadas se distribuye como una
        v.a. exponencial de media
        $\tfrac{1}{\red{\lambda}}$; y
        su función de densidad es

        \begin{equation}
            \mathbb{P}({\color{HotPink4}N}(\blue{t})={\color{HotPink4}k}) = \frac{(\red{\lambda}\blue{t})^{\color{HotPink4}k}\ e^{-\red{\lambda}\blue{t}}}{{\color{HotPink4}k}!}
        \end{equation}
    \end{definicion}

    \vfill



    \begin{figure}
        \centering
        \resizebox{.4\textwidth}{!}{%
            \begin{tikzpicture}
    % Time axis
    \draw[|->] (0,0) -- (5,0) node[anchor=west] {time [min]};

    % ticks
    \draw (1,.05) -- (1,-.05);
    \draw (2,.05) -- (2,-.05);
    \draw (3,.05) -- (3,-.05);

    
    % arrivals
    % \draw[->,color=red,thick] (0.2,1) node[anchor=west,rotate=70] {coche 1} --(0.2,0);
    \draw[->,color=red,thick] (0.8,1) node[anchor=west,rotate=70] {coche 1} --(0.8,0) node[below ] {\tiny 0.8};
    \draw[->,color=red,thick] (2.5,1)node[anchor=west,rotate=70] {coche 2} --(2.5,0);

    \draw[<->,color=blue,thick] (0,-.45) -- node[anchor=north,pos=0.5]{$t=3$ [min]} (3,-.45);
    \draw[|-|,color=HotPink4,thick] (0,-1.2) -- node[anchor=north,pos=0.5]{$N(t)=2$ [coches]} (3,-1.2);
\end{tikzpicture}
%
        }
    \end{figure}

    \textit{Ejemplo}:
    $\mathbb{P}({\color{HotPink4}N}(\blue{3})={\color{HotPink4}2}) = \frac{(\red{\lambda}\blue{3})^{\color{HotPink4}2}\ e^{-\red{\lambda}\blue{3}}}{{\color{HotPink4}2}!}\underbrace{=}_{\red{\lambda=2/3}}0.27$

\end{frame}






%% \begin{frame}{\secname}
%%     Un proceso de llegadas de Poisson nos
%%     dice la probabilidad de que lleguen
%%     $k$ usuarios en $t$ segundos:
%%     \begin{equation}
%%         \mathbb{P}(\red{k}\ \text{usuarios en}\ \blue{t})=\frac{(\lambda \blue{t})^\red{k} e^{-\lambda \blue{t} }}{\red{k}!}
%%         \label{eq:poisson}
%%     \end{equation}
%%     donde $\lambda$ es la \textbf{tasa}
%%     de llegadas [usuarios/sec].
%% 
%%     \vfill
%% 
%%     \textit{Ejemplo}: si la tasa
%%     de llegada es $\lambda=5$ [usuarios/sec],
%%     la probabilidad de
%%     que lleguen $\red{k=3}$ usuarios en $\blue{t=2}$~sec es
%%     $\tfrac{(5\cdot2)^3 e^{-5\cdot2}}{3!}=0.0075$.
%% \end{frame}




\begin{frame}{\secname: \subsecname}
    Propiedades de la distribución de Poisson:
    \begin{itemize}
        \item \textbf{media}: $\mathbb{E}[N(t)]=\lambda t$ usuarios
        \item \textbf{varianza}: $\Var[N(t)]=\lambda t$ usuarios\textsuperscript{2}
    \end{itemize}

    \begin{figure}
        \begin{tikzpicture}

\begin{axis}[every axis plot post/.append style={
  mark=none,domain=0:20,samples=20},
  axis x line*=bottom,
  axis y line*=left,
  xlabel={$k$ [usuarios]},
  ylabel={$\mathbb{P}(k\ \text{usuarios en 1 sec})$},
  enlargelimits=upper,
  width=4in,
  height=2in]
  \addplot[ultra thick,smooth,color=HotPink2] {poiss(1)} node [pos=0.03,anchor=south west] {$\lambda=1$};
  \addplot[ultra thick,smooth,color=HotPink3] {poiss(5)} node [pos=0.24,anchor=south west] {$\lambda=5$};
  \addplot[ultra thick,smooth,color=HotPink4] {poiss(10)} node [pos=0.65,anchor=south west] {$\lambda=10$};
\end{axis}


\end{tikzpicture}


    \end{figure}
\end{frame}




\subsection{Agregado}
\begin{frame}{\secname: \subsecname}
    ¿Cómo se distribuyen las llegadas
    de \textbf{\color{Firebrick1}A} y
    \textbf{\color{Firebrick4}B} juntos?

    \begin{figure}
        \begin{tikzpicture}
    % Time axis
    \draw[|->] (0,0) -- (5,0) node[anchor=west] {time [sec]};

    % ticks
    \draw (1,.05) -- (1,-.05) node[pos=1,anchor=north] {1};
    \draw (2,.05) -- (2,-.05) node[pos=1,anchor=north] {2};
    \draw (3,.05) -- (3,-.05) node[pos=1,anchor=north] {3};
    \draw (4,.05) -- (4,-.05) node[pos=1,anchor=north] {4};

    
    % arrivals A
    \draw[->,color=Firebrick1,thick] (0.2,1) node[anchor=west,rotate=70] {usuario A.1} --(0.2,0);
    \draw[->,color=Firebrick4,thick] (0.8,1) node[anchor=west,rotate=70] {usuario B.1} --(0.8,0);
    \draw[->,color=red,thick] (1.5,1)node[anchor=west,rotate=70] {usuario A.2} --(1.5,0);
    \draw[->,color=red,thick] (3,1)node[anchor=west,rotate=70] {usuario A.3} --(3,0);
    \draw[->,color=Firebrick4,thick] (3.8,1) node[anchor=west,rotate=70] {usuario B.2} --(3.8,0);

\end{tikzpicture}

    \end{figure}

    Vemos que:
    \begin{itemize}
        \item {\color{Firebrick1}$\lambda_A=\tfrac{3}{4}$} [coches/min], ya que {\color{HotPink1}$N_A(\blue{t=4\text{min}})=3$} [coches]
        \item {\color{Firebrick4}$\lambda_B=\tfrac{2}{4}$} [coches/min], ya que {\color{HotPink4}$N_B(\blue{t=4\text{min}})=2$} [coches]
    \end{itemize}

\end{frame}









\begin{frame}{\secname: \subsecname}

    \begin{lema}[Agregado procesos de Poisson]
        Sean
        \textbf{\color{Firebrick1}A} y
        \textbf{\color{Firebrick4}B} dos
        procesos de Poisson independientes
        con tasas 
        $\color{Firebrick1}\lambda_1$ y
        $\color{Firebrick4}\lambda_2$;
        el agregado es un proceso de Poisson
        de tasa
        ${\color{Firebrick3}\lambda}={\color{Firebrick1}\lambda_1}+{\color{Firebrick1}\lambda_2}$.
    \end{lema}


    \begin{figure}
        \resizebox{.3\textwidth}{!}{%
            \begin{tikzpicture}
    % Time axis
    \draw[|->] (0,0) -- (5,0) node[anchor=west] {time [sec]};

    % ticks
    \draw (1,.05) -- (1,-.05) node[pos=1,anchor=north] {1};
    \draw (2,.05) -- (2,-.05) node[pos=1,anchor=north] {2};
    \draw (3,.05) -- (3,-.05) node[pos=1,anchor=north] {3};
    \draw (4,.05) -- (4,-.05) node[pos=1,anchor=north] {4};

    
    % arrivals A
    \draw[->,color=Firebrick1,thick] (0.2,1) node[anchor=west,rotate=70] {usuario A.1} --(0.2,0);
    \draw[->,color=Firebrick4,thick] (0.8,1) node[anchor=west,rotate=70] {usuario B.1} --(0.8,0);
    \draw[->,color=red,thick] (1.5,1)node[anchor=west,rotate=70] {usuario A.2} --(1.5,0);
    \draw[->,color=red,thick] (3,1)node[anchor=west,rotate=70] {usuario A.3} --(3,0);
    \draw[->,color=Firebrick4,thick] (3.8,1) node[anchor=west,rotate=70] {usuario B.2} --(3.8,0);

\end{tikzpicture}
%
        }
    \end{figure}

    \textit{Demostración} (caso general $n$ procesos):
    \begin{equation}
        \mathbb{P}({\color{HotPink4}N}(\blue{t})=0)=\prod_i^n \mathbb{P}({\color{HotPink2}N_i}(\blue{t})=0)=\prod_i^n e^{-{\color{Firebrick1}\lambda_i} \blue{t}} = e^{-\sum_i^n {\color{Firebrick1}\lambda_i} \blue{t}} = e^{-{\color{Firebrick3}\lambda} \blue{t}}
    \end{equation}

\end{frame}







\begin{frame}{\secname: \subsecname}
    \begin{thm}[Palm-Khintchine \cite{amable}]
        Sea $\{{\color{HotPink2}N_i}(\blue{t})\}_i^n$ un conjunto de $n$
        procesos de llegada independientes
        con sendas tasas
        {\color{Firebrick1}$\lambda_i$}.
        La superposición de procesos
        \begin{equation}
            {\color{HotPink4}N}(\blue{t})=\sum_i^n {\color{HotPink2}N_i}(\blue{t}), \blue{t}\geq0
        \end{equation}
        tiende a un \textbf{proceso de Poisson}
        de tasa ${\color{Firebrick3}\lambda}=\sum_i {\color{Firebrick1}\lambda_i}$
        cuando $n\to\infty$, siempre y cuando
        se cumpla:
        \begin{enumerate}
            \item carga finita ${\color{Firebrick3}\lambda}<\infty$; y
            \item ningún proceso domine al
                agregado ${\color{Firebrick1}\lambda_i}<<{\color{Firebrick3}\lambda}$
        \end{enumerate}

        \label{th:palm}
    \end{thm}
    
\end{frame}





\begin{frame}{\secname: \subsecname}
    \textit{Ejemplo (tma. Palm-Khintchine)}:
    los tiempos de llegada de coches dependen
    del color, y son independientes del de
    otros colores. Además:
    \begin{itemize}
        \item tiempo entre coches rojos
            ${\color{Firebrick1}\sim U(0,10\text{ [min]})}$
        \item tiempo entre coches granates
            ${\color{Firebrick2}\sim N(\mu=20\text{ [min]}, \sigma=1 \text{ [min]})}$
        \item $\ldots$
        \item tiempo entre coches fucsia ${\color{Firebrick4}\sim Geo(p=0.2)}$
    \end{itemize}

    \vfill


    \begin{figure}
        \resizebox{.6\textwidth}{!}{%
          \begin{tikzpicture}
    % Time axis
    \draw[|->] (0,0) -- (10,0) node[anchor=west] {time [min]};

    % ticks
    % \draw (1,.05) -- (1,-.05) node[pos=1,anchor=north] {1};
    % \draw (2,.05) -- (2,-.05) node[pos=1,anchor=north] {2};
    % \draw (3,.05) -- (3,-.05) node[pos=1,anchor=north] {3};
    % \draw (4,.05) -- (4,-.05) node[pos=1,anchor=north] {4};

    
    % arrivals Uniform
    \draw[->,color=Firebrick1,thick] (5,1) node[anchor=south] {U1} --(5,0) node[anchor=north] {\tiny 5};
    \draw[->,color=Firebrick1,thick] (7,1) node[anchor=south] {U2} --(7,0)node[anchor=north] {\tiny 7};


    % arrivals Norm
    \draw[->,color=Firebrick2,thick] (9,1) node[anchor=south] {N1} --(9,0)node[anchor=north] {\tiny 9};

    % arrivals Geom
    \draw[->,color=Firebrick4,thick] (3,1) node[anchor=south] {G1} --(3,0) node[anchor=north] {\tiny 3};
    \draw[->,color=Firebrick4,thick] (8,1) node[anchor=south] {G2} --(8,0) node[anchor=north] {\tiny 8};

\end{tikzpicture}
%
        }
    \end{figure}

    \vfill


    El agregado será un proceso Poisson de
    tasa ${\color{Firebrick3}\lambda}=
    \sum_i {\color{Firebrick1}\lambda_i}=
    {\color{Firebrick1}\frac{1}{5}} + {\color{Firebrick2}\frac{1}{20}} + \ldots + p$
\end{frame}







\subsection{PASTA}
\begin{frame}{\secname: \subsecname}
    ``Poisson Arrivals See Time Averages'' (PASTA)\footnote{Ejemplo de \cite[Figura 3.17]{amable}}

    \vfill

    \begin{figure}
        \begin{tikzpicture}
    \begin{axis}[
  mark=none,
  axis x line*=bottom,
  axis y line*=left,
  enlargelimits=upper,
        width=7cm,height=4cm,ylabel=$X(t)$,xlabel=$t$,ymin=0,ymax=4,xmin=0,xmax=6]
        \addlegendimage{ultra thick,blue}
        \addplot[ultra thick,blue,samples at={0,2}] {2};
        \addplot[ultra thick,blue,samples at={2,4}] {1};
        \addplot[ultra thick,blue,samples at={4,6}] {3};

        \addplot +[dashed,gray,mark=none] coordinates {(2, 0) (2, 2)};
        \addplot +[dashed,gray,mark=none] coordinates {(4, 0) (4, 3)};
        \addplot +[dashed,gray,mark=none] coordinates {(6, 0) (6, 3)};

        % Arrival 1
        \addplot +[<-,Firebrick1,thick,solid,mark=none] coordinates {(1, 0) (1, .75)};
        \addplot[Firebrick1, mark=*, only marks] coordinates {(1,2)};

        % Arrival 2
        \addplot +[<-,Firebrick1,thick,solid,mark=none] coordinates {(1.6, 0) (1.6, .75)};
        \addplot[Firebrick1, mark=*, only marks] coordinates {(1.6,2)};

        % Arrival 2
        \addplot +[<-,Firebrick1,thick,solid,mark=none] coordinates {(3.7, 0) (3.7, .75)};
        \addplot[Firebrick1, mark=*, only marks] coordinates {(3.7,1)};

        % Arrival 3
        \addplot +[<-,Firebrick1,thick,solid,mark=none] coordinates {(4.4, 0) (4.4, .75)};
        \addplot[Firebrick1, mark=*, only marks] coordinates {(4.4, 3)};

    \end{axis}
\end{tikzpicture}

    \end{figure}

    En media, los \blue{valores} vistos por
    {\color{Firebrick1}llegadas} de Poisson
    es la media temporal $\overline{X}(t)$.

\end{frame}



\begin{frame}{\secname: \subsecname}
    \begin{lema}[PASTA]
        Sea \blue{$X(t)$} un proceso aleatorio,
        y \red{$Y$} la v.a. definida como el
        valor que toman las llegadas de Poisson
        al muestrear \blue{X(t)}, se tiene que:
        \begin{equation}
            \blue{\overline{X}(t)} = \red{\overline{Y}}
        \end{equation}
    \end{lema}
\end{frame}




\begin{frame}{\secname: \subsecname - ejemplo}

    Media temporal: 
    \begin{equation*}
        \blue{\overline{X}(t)}=\frac{1}{6}\int_0^6 X(t)\ dt =\frac{1}{6} (2\cdot2 + 1\cdot2 + 3\cdot2) = \blue{\frac{8}{6}}
    \end{equation*}

    Llegadas de Poisson:
    \begin{equation*}
        \red{\overline{Y}} = 2\cdot\mathbb{P}([0,2]) + 1 \cdot\mathbb{P}([2,4]) + 3\cdot \mathbb{P}([4,6])=2\cdot\frac{2}{6} + 1 \cdot\frac{2}{6}+ 3\cdot\frac{2}{6}=\red{\frac{8}{6}}
    \end{equation*}

    \vfill

    \begin{figure}
        \resizebox{.4\textwidth}{!}{%
            \begin{tikzpicture}
    \begin{axis}[
  mark=none,
  axis x line*=bottom,
  axis y line*=left,
  enlargelimits=upper,
        width=7cm,height=4cm,ylabel=$X(t)$,xlabel=$t$,ymin=0,ymax=4,xmin=0,xmax=6]
        \addlegendimage{ultra thick,blue}
        \addplot[ultra thick,blue,samples at={0,2}] {2};
        \addplot[ultra thick,blue,samples at={2,4}] {1};
        \addplot[ultra thick,blue,samples at={4,6}] {3};

        \addplot +[dashed,gray,mark=none] coordinates {(2, 0) (2, 2)};
        \addplot +[dashed,gray,mark=none] coordinates {(4, 0) (4, 3)};
        \addplot +[dashed,gray,mark=none] coordinates {(6, 0) (6, 3)};

        % Arrival 1
        \addplot +[<-,Firebrick1,thick,solid,mark=none] coordinates {(1, 0) (1, .75)};
        \addplot[Firebrick1, mark=*, only marks] coordinates {(1,2)};

        % Arrival 2
        \addplot +[<-,Firebrick1,thick,solid,mark=none] coordinates {(1.6, 0) (1.6, .75)};
        \addplot[Firebrick1, mark=*, only marks] coordinates {(1.6,2)};

        % Arrival 2
        \addplot +[<-,Firebrick1,thick,solid,mark=none] coordinates {(3.7, 0) (3.7, .75)};
        \addplot[Firebrick1, mark=*, only marks] coordinates {(3.7,1)};

        % Arrival 3
        \addplot +[<-,Firebrick1,thick,solid,mark=none] coordinates {(4.4, 0) (4.4, .75)};
        \addplot[Firebrick1, mark=*, only marks] coordinates {(4.4, 3)};

    \end{axis}
\end{tikzpicture}
%
        }
    \end{figure}

    \vfill

    con
    \begin{equation*}%
        \scriptstyle
        \mathbb{P}([0,2])=\frac{\mathbb{P}(N(0,2)=1)\mathbb{P}(N(2,4)=0)\mathbb{P}(N(4,6)=0)}{\mathbb{P}(N([0,6])=1)}
                          =\frac{\frac{(2\lambda)^1e^{-2\lambda}}{1!} \frac{(2\lambda)^0e^{-2\lambda}}{0!} \frac{(2\lambda)^0e^{-2\lambda}}{0!} }{\frac{(6\lambda)^1e^{-6\lambda}}{1!}}=\frac{2}{6}%
    \end{equation*}%
\end{frame}



\section{Teoría de Colas}



\begin{frame}{\secname}
    Estudia las prestaciones de sistemas
    de colas como la ilustrada abajo

    \begin{figure}
        
\begin{tikzpicture}[start chain=going right,>=latex,node distance=0pt]
% the rectangular shape with vertical lines
\node[rectangle split, rectangle split parts=6,
draw, rectangle split horizontal,text height=1cm,text depth=0.5cm,on chain,inner ysep=0pt] (wa) {};
\fill[white] ([xshift=-\pgflinewidth,yshift=-\pgflinewidth]wa.north west) rectangle ([xshift=-15pt,yshift=\pgflinewidth]wa.south);

% the circle
\node[draw,circle,on chain,minimum size=1.5cm] (se) {$\mu$};

% the arrows and labels
\draw[->] (se.east) -- +(20pt,0);
\draw[<-] (wa.west) -- +(-20pt,0) node[left] (user) {$\lambda$};

\node[anchor=north] at (se.south) {surtidor};
\node[anchor=north] at (wa.south) {cola};
\node[anchor=north] at (user.south) {coche};


\end{tikzpicture}

    \end{figure}

    \begin{itemize}
        \item $\lambda$: tasa de llegadas
            [vehículos/min]
        \item $\mu$: tasa de servicio
            [vehículos/min]
    \end{itemize}
\end{frame}






\begin{frame}{\secname}
    Si llegan muchos usuarios (coches)
    a la cola, hay que dimensionar:


    \begin{figure}
     \centering
     \begin{subfigure}[b]{0.3\textwidth}
         \centering
         \resizebox{\textwidth}{!}{%
             
\begin{tikzpicture}[start chain=going right,>=latex,node distance=0pt]
% the rectangular shape with vertical lines
\node[rectangle split, rectangle split parts=6,
draw, rectangle split horizontal,text height=1cm,text depth=0.5cm,on chain,inner ysep=0pt] (wa) {};
\fill[white] ([xshift=-\pgflinewidth,yshift=-\pgflinewidth]wa.north west) rectangle ([xshift=-15pt,yshift=\pgflinewidth]wa.south);

% the circle
\node[draw,circle,on chain,minimum size=3cm] (se) {$\mu$};

% the arrows and labels
\draw[->] (se.east) -- +(20pt,0);
\draw[<-] (wa.west) -- +(-20pt,0) node[left] (user) {$\lambda$};

\node[anchor=north] at (se.south) {surtidor};
\node[anchor=north] at (wa.south) {cola};
\node[anchor=north] at (user.south) {coche};


\end{tikzpicture}
%
         }
         \caption{Surtidor potente}
     \end{subfigure}
     \hfill
     \begin{subfigure}[b]{0.3\textwidth}
         \centering
         \resizebox{\textwidth}{!}{%
             \usetikzlibrary{scopes}
\begin{tikzpicture}[start chain=going right,>=latex,node distance=0pt]
% the rectangular shape with vertical lines
\node[rectangle split, rectangle split parts=6,
draw, rectangle split horizontal,text height=1cm,text depth=0.5cm,on chain,inner ysep=0pt] (wa) {};
\fill[white] ([xshift=-\pgflinewidth,yshift=-\pgflinewidth]wa.north west) rectangle ([xshift=-15pt,yshift=\pgflinewidth]wa.south);


% Phantom nodes
\node [on chain,opacity=0] {A};
\node [on chain,opacity=0] {A};
% Phantom node for server alignment
\node [on chain,opacity=0] {A};
{ [start branch=numbers going above]
  % the circle above
    \node[draw,circle,on chain,minimum size=1.5cm] (seu) {$\mu$};
}
{ [start branch=numbers going below]
  % the circle below
  \node[draw,circle,on chain,minimum size=1.5cm] (sed) {$\mu$};
}


%% % the circle up
%% \node[draw,circle,on chain=going above,minimum size=1.5cm] (se) {$\mu$};

% the arrows and labels
\draw[<-] (wa.west) -- +(-20pt,0) node[left] (user) {$\lambda$};
\node[anchor=north] at (sed.south) {surtidor};
\node[anchor=north] at (wa.south) {cola};
\node[anchor=north] at (user.south) {coche};
\draw[->] (wa.east) -- (seu.south west);
\draw[->] (wa.east) -- (sed.north west);
\draw[->] (seu.east) -- +(20pt,0);
\draw[->] (sed.east) -- +(20pt,0);

\end{tikzpicture}
%
         }
         \caption{Dos surtidores}
     \end{subfigure}
     \hfill
     \begin{subfigure}[b]{0.3\textwidth}
         \centering
         \resizebox{\textwidth}{!}{%
             \begin{tikzpicture}[start chain=going right,>=latex,node distance=0pt]



% Phantom nodes
\node [on chain,opacity=0] (begin) {A};
\node [on chain,opacity=0] {A};
% Phantom node for server alignment
\node [on chain,opacity=0] {A};
{ [start branch=numbers going above]
  % the queue above
  %%%
  % the rectangular shape with vertical lines
  \node[rectangle split, rectangle split parts=6,
  draw, rectangle split horizontal,text height=1cm,text depth=0.5cm,on chain,inner ysep=0pt] (wau) {};
  \fill[white] ([xshift=-\pgflinewidth,yshift=-\pgflinewidth]wau.north west) rectangle ([xshift=-15pt,yshift=\pgflinewidth]wau.south);
}
{ [start branch=numbers going below]
  % the queue below 
  %%%
  % the rectangular shape with vertical lines
  \node[rectangle split, rectangle split parts=6,
  draw, rectangle split horizontal,text height=1cm,text depth=0.5cm,on chain,inner ysep=0pt] (wad) {};
  \fill[white] ([xshift=-\pgflinewidth,yshift=-\pgflinewidth]wad.north west) rectangle ([xshift=-15pt,yshift=\pgflinewidth]wad.south);
}

% Server up
\node[draw,circle,minimum size=1.5cm,anchor=west] at (wau.east) (seu) {$\mu$};
% Server down
\node[draw,circle,minimum size=1.5cm,anchor=west] at (wad.east) (sed) {$\mu$};



% the arrows and labels
\node (user) at (-2,0) {$\lambda$};
\draw[->] (user.east) -- (wau.west);
\draw[->] (user.east) -- (wad.west);



\node[anchor=north] at (sed.south) {surtidor};
\node[anchor=north] at (wad.south) {cola};
\node[anchor=north] at (user.south) {coche};
\draw[->] (seu.east) -- +(20pt,0);
\draw[->] (sed.east) -- +(20pt,0);

\end{tikzpicture}
%
         }
         \caption{Dos colas+surtidor}
     \end{subfigure}
    \end{figure}

\end{frame}



\begin{frame}{\secname}
    Hay que dimensionar para \textbf{hora pico}
    $\lambda^{\max}$.

    \vfill

    \begin{figure}
        \begin{tikzpicture}

\begin{axis}[every axis plot post/.append style={
  mark=none,domain=0:22.5,samples=20},
  axis x line*=bottom,
  axis y line*=left,
  xlabel={$t$ [hora]},
  ylabel={$\lambda(t)$ [vehículos/min]},
  enlargelimits=upper,
  width=4in,
  height=2in]
  \addplot[ultra thick,smooth,color=DodgerBlue1] {10*sin(deg(x*pi/24))} node [pos=0.3,anchor=south west] {};

  \node[anchor=north,circle,inner sep=1pt,mark=*] at (axis cs:12,10) {$\lambda^{\max}$};

  \addplot [DodgerBlue4, mark=*, only marks]
coordinates {
(12,10)
};
\end{axis}


\end{tikzpicture}


    \end{figure}
\end{frame}



\section{Sistema M/M/1}

\begin{frame}{\secname}
    La notación de Kendall define cómo es 
    una cola
    \begin{center}
        A/S/c/K/N/D 
    \end{center}
    donde:
    \begin{itemize}
        \item \textbf{A} es la v.a. del tiempo
            entre llegadas;
        \item \textbf{S} es la v.a. del tiempo
            de servicio;
        \item \textbf{c} es el número de
            servidores que atienden la cola;
        \item \textbf{K} es en tamaño de la cola;
        \item \textbf{N} es la cantidad de
            llegadas; y
        \item \textbf{D} es la política de
            encolado.
    \end{itemize}
\end{frame}




\begin{frame}{\secname}
    Un M/M/1 es una cola\footnote{en notación
    Kendall M/M/1/$\infty$/$\infty$/FIFO}
    como la de la figura.

    \begin{figure}
        
\begin{tikzpicture}[start chain=going right,>=latex,node distance=0pt]
% the rectangular shape with vertical lines
\node[rectangle split, rectangle split parts=6,
draw, rectangle split horizontal,text height=1cm,text depth=0.5cm,on chain,inner ysep=0pt] (wa) {};
\fill[white] ([xshift=-\pgflinewidth,yshift=-\pgflinewidth]wa.north west) rectangle ([xshift=-15pt,yshift=\pgflinewidth]wa.south);

% the circle
\node[draw,circle,on chain,minimum size=1.5cm] (se) {$\mu$};

% the arrows and labels
\draw[->] (se.east) -- +(20pt,0);
\draw[<-] (wa.west) -- +(-20pt,0) node[left] {$\lambda$};
\node[align=center,below] at (wa.south) {Waiting \\ Area};
\node[align=center,below] at (se.south) {Service \\ Node};
\end{tikzpicture}

    \end{figure}

    donde:
    \begin{itemize}
        \item el tiempo entre llegadas es
            exponencial (M);
        \item el tiempo de servicio es
            exponencial (M);
        \item hay un servidor atendiento (1); y
        \item la cola es infinita ($\infty$).
    \end{itemize}

\end{frame}




\begin{frame}{\secname}
    Queremos responder a preguntas cómo:
    \begin{itemize}
        \item ¿cuál es la probabilidad
            de esperar en cola?
        \item ¿cuánto esperaré en la cola? o
        \item ¿cuánto tardaré en ser servido?
    \end{itemize}

    \vfill

    Para ello modelamos la cola con una
    cadena de Markov.
\end{frame}



\subsection{Cadena de Markov en tiempo continuo}

\begin{frame}{\secname: \subsecname}
    \begin{figure}
        %%% \tikzset{node distance=1cm, % Minimum distance between two nodes. Change if necessary.
%%%          every state/.style={ % Sets the properties for each state
%%%            semithick,draw=HotPink3!50,
%%%            fill=HotPink3!20},
%%%          initial text={},     % No label on start arrow
%%%          double distance=4pt, % Adjust appearance of accept states
%%%          every edge/.style={  % Sets the properties for each transition
%%%          draw, ->,>=stealth',     % Makes edges directed with bold arrowheads
%%%            auto, thick},
%%% }
\begin{tikzpicture}
   
% Estados
\node[state] (1) {$0$};
\node[state] (2) [right=of 1] {$1$};
\node[state] (3) [right=of 2] {$2$};
\node[state,fill=HotPink3!20] (4) [right=of 3] {$3$};
\node[] (5) [right=of 4] {$\cdots$};


% Transiciones
\draw [->,thick] (1.north east) to [bend left=55]  node[above] {$\lambda$}  (2.north west);
\draw [->,thick] (2.south west) to [bend left=55]  node[below] {$\mu$}  (1.south east);

\draw [->,thick] (2.north east) to [bend left=55]  node[above] {$\lambda$}  (3.north west);
\draw [->,thick] (3.south west) to [bend left=55]  node[below] {$\mu$}  (2.south east);

\draw [->,thick] (3.north east) to [bend left=55]  node[above] {$\lambda$}  (4.north west);
\draw [->,thick] (4.south west) to [bend left=55]  node[below] {$\mu$}  (3.south east);

\draw [->,thick] (4.north east) to [bend left=55]  node[above] {$\lambda$}  (5.north west);
\draw [->,thick] (5.south west) to [bend left=55]  node[below] {$\mu$}  (4.south east);



    
\end{tikzpicture}

    \end{figure}

    \begin{figure}
        \tikzset{
queue/.style={
  rectangle split,
  minimum height=1cm,
  rectangle split horizontal,
  rectangle split parts=8, 
  draw, 
  anchor=center,
  on chain,
  text height=1cm,
  text depth=0.5cm,
  inner ysep=0pt
  },
}


\begin{tikzpicture}[start chain=going right,>=latex,node distance=0pt]
% the rectangular shape with vertical lines

%\node[rectangle split, rectangle split parts=6,
%draw, rectangle split horizontal,text height=1cm,text depth=0.5cm,on chain,inner ysep=0pt] (wa) {};

\node[
  queue,
  rectangle split part fill={white,white,white,white,white,white,HotPink3!20}
  ]  (wa) {};

\fill[white] ([xshift=-\pgflinewidth,yshift=-\pgflinewidth]wa.north west) rectangle ([xshift=-15pt,yshift=\pgflinewidth]wa.south);

% the circle
\node[draw,fill=HotPink3!20,circle,on chain,minimum size=1.5cm] (se) {$\mu$};

% the arrows and labels
\draw[->] (se.east) -- +(20pt,0);
\draw[<-] (wa.west) -- +(-20pt,0) node[left] {$\lambda$};
\node[align=center,below] at (wa.south) {Waiting \\ Area};
\node[align=center,below] at (se.south) {Service \\ Node};
\end{tikzpicture}

    \end{figure}


    \vfill

    \begin{itemize}
        \item estados = \#usuarios en
            cola y servidor
        \item transiciones =
            tasa de llegada $\lambda$ /
            tasa de servicio $\mu$
    \end{itemize}
\end{frame}



%%%%%%%%%%%%%%%%%%%%%%%
%% Primera definición
%%%%%%%%%%%%%%%%%%%%%%%
%% \begin{frame}{\secname: \subsecname}
%%     \begin{definicion}[Cadena de Markov en tiempo continuo\cite{amable}]
%%         Un proceso estocástico $\{N(t),\ t>0\}$
%%         es una cadena de Markov en tiempo
%%         contínuo si se cumple la propiedad
%%         de Markov:
%%         \begin{align}
%%             & \mathbb{P}\left(N(t)=i_t|\ N(t-1)=i_{t-1}, N(t-2)=i_{t-2}, \ldots, N(0)=i_0\right)\nonumber\\
%%             & = \mathbb{P}\left(N(t)=i_t|\ N(t-1)=i_{t-1}\right)
%%         \end{align}
%%         con $t>t-1>\ldots>0$ y
%%         $\{i_t,i_{t-1},\ldots,i_0\}$ una secuencia
%%         de estados cualquiera.
%%     \end{definicion}
%% \end{frame}





% Proceso sin memoria ilustración
%% \begin{frame}{\secname: \subsecname}
%%     \begin{figure}
%%         \begin{tikzpicture}

    \draw (0,0) -- (5.5, 0) node[anchor=west] {$\ldots$};

    \draw[->] (6.5, 0) -- (9,0) node[pos=1,anchor=west] {time};
    \draw[->] (0,0) -- (0,4) node[pos=.5,above,rotate=90,] {$N(t)$ [usuarios]};

    \draw[HotPink3,ultra thick] (0.5,2) -- (1.5, 2)
        -- (1.5,1) -- (2.5, 1)
        -- (2.5, 2) -- (3.5, 2)
        -- (3.5, 4) -- (4.5, 4)
        -- (4.5, 3) -- (5.5, 3)
        -- (5.5, 1) node[anchor=west] {$\ldots$};

    \draw[HotPink3,ultra thick] (6.5, 1)
        -- (6.5, 2) -- (7.5, 2)
        -- (7.5, 3) -- (8.5, 3);


    \draw (1, .1) -- (1,-.1) node[below] {$0$};
    \node[circle,fill=HotPink4,inner sep=2pt,label=above:{$i_0=2$}] at (1, 2) {};

    \draw (2, .1) -- (2,-.1) node[below] {$1$};
    \node[circle,fill=HotPink4,inner sep=2pt,label=below:{$i_1=1$}] at (2, 1) {};

    \draw (3, .1) -- (3,-.1) node[below] {$2$};
    \node[circle,fill=HotPink4,inner sep=2pt] at (3, 2) (t2) {};
    \node[anchor=north west] at (t2) {$i_2=2$};

    \draw (4, .1) -- (4,-.1) node[below] {$3$};
    \node[circle,fill=HotPink4,inner sep=2pt,label=above:{$i_3=4$}] at (4, 4) {};

    \draw (5, .1) -- (5,-.1) node[below] {$4$};
    \node[circle,fill=HotPink4,inner sep=2pt] (t4) at (5, 3) {};
    \node[anchor=south west] at (t4) {$i_4=3$};


    % Prior-to-last state
    \draw (7, .1) -- (7,-.1) node[below] {$t-1$};
    \only<1>{\node[circle,fill=HotPink4,inner sep=2pt,label=below:{$\qquad i_{t-1}=2$}] at (7, 2) {}};
    \only<2->{\node[circle,fill=HotPink4,inner sep=4pt,label=below:{$\qquad i_{t-1}=2$}] (prior) at (7, 2) {}};

    % last state
    \draw (8, .1) -- (8,-.1) node[below] {$t$};
    \only<1>{\node[circle,fill=HotPink4,inner sep=2pt,label=above:{$i_t=3$}] at (8, 3) {}};
    \only<2->{\node[circle,fill=HotPink4,inner sep=4pt,label=above:{$i_t=3$}] (last) at (8, 3) {}};

    % Transition arrow
    \only<2->{\draw[->,very thick,dashed] (prior) to [out=0,in=270] (last);}


    \only<3>{%
        \filldraw [fill=blue, fill opacity=0.2, draw=none] (6.25,4.65) rectangle (0.3,0.3);%
        \node[fill=white] at (3,3) {Sin memoria};}


\end{tikzpicture}


%%     \end{figure}
%% \end{frame}


\begin{frame}{\secname: \subsecname}
    \begin{definicion}[Cadena de Markov en tiempo continuo\cite{amable}]
        Un proceso estocástico
        $\{N(t),\ t>0\}$ es una cadena de Markov
        si cumple:
        \begin{enumerate}
            \item el tiempo de estancia
                en el estado $i$ sigue
                una v.a. exponencial
                independiente de tasa $\nu_i$; y
            \item el proceso pasa del estado
                $i$ al $j$ con una
                probabilidad $\pi_{ij}$
                que cumple
                $\sum_j \pi_{ij} = 1,\ \forall i$
        \end{enumerate}
    \end{definicion}
\end{frame}




\begin{frame}{\secname: \subsecname}
    ¿Cumple la cadena de un M/M/1 la condición
    de Markov? \pause \textbf{Sí}.

    \vfill

    \begin{enumerate}
        \item el tiempo de estancia $T$ en un
            estado se distribuye como una
            exponencial de tasa
            $\nu=\lambda_l+\lambda_s$:
            \begin{equation*}
                \mathbb{P}(T>\tau)=\mathbb{P}(\min\{t_l,t_s\}>\tau)=e^{-(\lambda_l+\lambda_s)\tau}
            \end{equation*}
            con $t_l,t_s$ las v.a.
            exponenciales del tiempo de
            llegada y servicio.

        \item la suma de probabilidades
            de transición es uno
            \begin{multline*}
                \mathbb{P}(N(t+1)=k+1|\ N(t)=k)
                + \mathbb{P}(N(t+1)=k-1|\ N(t)=k)\\
                = \mathbb{P}(t_l<t_s)
                + \mathbb{P}(t_s<t_l)
                = \frac{\lambda_l}{\lambda_l+\lambda_s}
                + \frac{\lambda_s}{\lambda_l+\lambda_s} = 1
            \end{multline*}
    \end{enumerate}

\end{frame}






\begin{frame}{\secname: \subsecname}
    ¿Es realista asumir tiempos exponenciales?
    \begin{itemize}
        \item \textbf{tiempos de llegada}: sí por
            el teorema de
            Palm-Khintchine~\ref{th:palm}.\pause
        \item \textbf{tiempos de servicio}:
            no, pero da expresiones cerradas
            y es una cota pesimista para altas
            fiabilidades
    \end{itemize}

    \vfill

    
    \begin{figure}
        \begin{tikzpicture}
\pgfplotsset{compat=1.10}

\begin{axis}[every axis plot post/.append style={
  mark=none,samples=20},
  axis x line*=bottom,
  axis y line*=left,
  xlabel={$\tau$ [sec]},
  ylabel={$\mathbb{P}(t_s\leq \tau)$},
  enlargelimits=upper,
  width=4in,
  height=2in]

  \addplot+[domain=0:4,color=DodgerBlue4,ultra thick] {x/4} node[pos=0.5,anchor=north west] {$U(0,4)$};

  \addplot[ultra thick,smooth,color=DodgerBlue1,domain=0:6] {exponentialcdf(0.5)} node [pos=0.4,anchor=south east] {$Exp(\lambda=\tfrac{1}{2})$};

  % Phantom to highlight area
  \addplot[name path=E,ultra thick,smooth,color=DodgerBlue1,domain=3.1:6] {exponentialcdf(0.5)} ;

  \addplot+[name path=U,domain=4:6,color=DodgerBlue4,ultra thick] {1};

  \addplot[Firebrick3!20,domain=4:6] fill between[of=U and E];
  
  \draw[->,Firebrick3,thick] (axis cs:4.5,0.7) -- (axis cs:4.5,0.95) node[pos=0,anchor=north] {\small $\mathbb{P}_E(t_s\leq\tau)<\mathbb{P}_U(t_s\leq\tau)$} ;

\end{axis}



\end{tikzpicture}


    \end{figure}
\end{frame}








\subsection{Probabilidades de estado}




\begin{frame}{\secname: \subsecname}
    La cadena de Markov va pasando
    por estados.
    %\footnote{\url{https://setosa.io/blog/2014/07/26/markov-chains/index.html}}.

    En el instante $t$ la
    probabilidad de estar en cada estado es,
    e.g.:
    \begin{equation*}
        \pmb{\pi}(t) = (\pi_0(t), \pi_1(t), \pi_2(t), \pi_3(t), \ldots) = (0.02, 0.12, 0.3, 0.07, \ldots)
    \end{equation*}

    \begin{figure}
        %%% \tikzset{node distance=1cm, % Minimum distance between two nodes. Change if necessary.
%%%          every state/.style={ % Sets the properties for each state
%%%            semithick,draw=HotPink3!50,
%%%            fill=HotPink3!20},
%%%          initial text={},     % No label on start arrow
%%%          double distance=4pt, % Adjust appearance of accept states
%%%          every edge/.style={  % Sets the properties for each transition
%%%          draw, ->,>=stealth',     % Makes edges directed with bold arrowheads
%%%            auto, thick},
%%% }
\begin{tikzpicture}
   
% Estados
\node[state] (1) {$0$};
\node[state] (2) [right=of 1] {$1$};
\node[state] (3) [right=of 2] {$2$};
\node[state,fill=HotPink3!20] (4) [right=of 3] {$3$};
\node[] (5) [right=of 4] {$\cdots$};

% Probabilidades
\node[anchor=north] at (1.south) {$\pi_0(t)$};
\node[anchor=north] at (2.south) {$\pi_1(t)$};
\node[anchor=north] at (3.south) {$\pi_2(t)$};
\node[anchor=north,HotPink4] at (4.south) {\textbf{$\pi_3(t)$}};


% Transiciones
\draw [->,thick] (1.north east) to [bend left=55]  node[above] {$\lambda$}  (2.north west);
\draw [->,thick] (2.south west) to [bend left=55]  node[below] {$\mu$}  (1.south east);

\draw [->,thick] (2.north east) to [bend left=55]  node[above] {$\lambda$}  (3.north west);
\draw [->,thick] (3.south west) to [bend left=55]  node[below] {$\mu$}  (2.south east);

\draw [->,thick] (3.north east) to [bend left=55]  node[above] {$\lambda$}  (4.north west);
\draw [->,thick] (4.south west) to [bend left=55]  node[below] {$\mu$}  (3.south east);

\draw [->,thick] (4.north east) to [bend left=55]  node[above] {$\lambda$}  (5.north west);
\draw [->,thick] (5.south west) to [bend left=55]  node[below] {$\mu$}  (4.south east);



    
\end{tikzpicture}

    \end{figure}


\end{frame}





\begin{frame}{\secname: \subsecname}

    ¿Cómo varía la probabilidad de estar
    en el estado $i$ tras $\varepsilon$ [sec]?

    \begin{multline}
        \frac{d}{dt} {\color{HotPink4}\pi_i(t)}=-{\color{HotPink4}\pi_i(t)}{\color{Firebrick3}\nu_i}
        + \sum_{j\neq i} \pi_j(t) \cdot \nu_j\pi_{j i}\\
        =-{\color{HotPink4}\pi_i(t)}{\color{Firebrick3}(\lambda+\mu)} 
        + {\color{DodgerBlue1}\pi_{i-1}(t) \cdot (\lambda+\mu)
    \frac{\lambda}{\lambda+\mu}}
    + {\color{DodgerBlue4}\pi_{i+1}(t) \cdot \overbrace{(\lambda+\mu)}^{\nu_{i+1}}
        \overbrace{\frac{\mu}{\lambda+\mu}}^{\pi_{i,i+1}(t)} }\\
        =-{\color{HotPink4}\pi_i(t)} {\color{Firebrick3}(\lambda+\mu)} 
        + {\color{DodgerBlue1}\pi_{i-1}(t) \lambda }
        + {\color{DodgerBlue4}\pi_{i+1}(t) \mu}
        \label{eq:derivative}
    \end{multline}


    \begin{figure}
        %%% \tikzset{node distance=1cm, % Minimum distance between two nodes. Change if necessary.
%%%          every state/.style={ % Sets the properties for each state
%%%            semithick,draw=HotPink3!50,
%%%            fill=HotPink3!20},
%%%          initial text={},     % No label on start arrow
%%%          double distance=4pt, % Adjust appearance of accept states
%%%          every edge/.style={  % Sets the properties for each transition
%%%          draw, ->,>=stealth',     % Makes edges directed with bold arrowheads
%%%            auto, thick},
%%% }
\begin{tikzpicture}
   
% Estados
\node[state,fill=DodgerBlue1!20] (2) [right=of 1] {$i-1$};
\node[state,fill=HotPink3!20] (3) [right=of 2] {$i$};
\node[state,fill=DodgerBlue4!20] (4) [right=of 3] {$i+1$};

% Probabilidades
\node[anchor=north,DodgerBlue1] at (2.south) {$\pi_{i-1}(t)$};
\node[anchor=north,HotPink4] at (3.south) {$\pi_i(t)$};
\node[anchor=north,DodgerBlue4] at (4.south) {\textbf{$\pi_{i+1}(t)$}};


% Transiciones
\draw [->,thick,DodgerBlue1] (2.north east) to [bend left=55]  node[above] {$\lambda$}  (3.north west);
\draw [->,thick,Firebrick3] (3.south west) to [bend left=55]  node[below] {$\mu$}  (2.south east);

\draw [->,thick,Firebrick3] (3.north east) to [bend left=55]  node[above] {$\lambda$}  (4.north west);
\draw [->,thick,DodgerBlue4] (4.south west) to [bend left=55]  node[below] {$\mu$}  (3.south east);




    
\end{tikzpicture}

    \end{figure}

\end{frame}





\begin{frame}{\secname: \subsecname}
    Si $t\to\infty$, la cadena alcanza
    \cite{amable}
    una distribución estacionaria donde las
    probabilidades no varían:
    
    \begin{equation}
        \lim_{t\to\infty} \frac{d}{dt}\pi_i(t) = 0,\quad \forall i
        \label{eq:null-derivative}
    \end{equation}


    \vfill

    Nos referimos a la \textbf{distribución
    estacionaria} como
    $\pmb{\pi}=\lim_{t\to\infty} \pmb{\pi}(t)$.

    \textit{Ejemplo}:
    \begin{equation*}
        \pmb{\pi} = (\pi_0, \pi_1, \pi_2, {\color{HotPink4}\pi_3}, \ldots) = (0.12, 0.04, 0.17, {\color{HotPink4}0.06}, \ldots)
    \end{equation*}

\end{frame}




\subsection{Ecuaciones de equilibrio}

\begin{frame}{\secname: \subsecname}
    Usando \eqref{eq:derivative}
    y \eqref{eq:null-derivative} podemos
    definir la \textbf{ecuación de equilibrio}
    para encontrar la distribución estacionaria
    $\pmb{\pi}$:

    \begin{equation}
        0 = \pmb{\pi}Q = 
        (\pi_0, \pi_1, \ldots)
        \begin{pmatrix}
            -\lambda & \lambda & 0 & 0 & \ldots \\
            \mu & -(\lambda+\mu) & \lambda & 0 & \ldots \\
            0  & \mu & -(\lambda+\mu) & \lambda &  \ldots \\
            \vdots  & \vdots & \vdots & \vdots &  \ddots \\
        \end{pmatrix}
        \label{eq:equilibrio}
    \end{equation}
    con $q_{ij}=\nu_i \pi_{i j}$ es la entrada
    $(i,j)$ de la matriz de transición $Q$.

\end{frame}






\begin{frame}{\secname: \subsecname}
    De \eqref{eq:equilibrio} sacamos
    el sistema de ecuaciones de equilibrio:
    \begin{equation}
        \pi_i = \pi_{i-1} \frac{\lambda}{\mu},
        \quad \forall i>0
        \label{eq:equilibrio}
    \end{equation}
    que equivale a:
    \begin{equation}
        \pi_i = \pi_0 \rho^i, \quad \forall i>0
        \label{eq:equilibrio-potencia}
    \end{equation}
    donde $\rho=\tfrac{\lambda}{\mu}$
    es la \textbf{carga del sistema}.
\end{frame}


\begin{frame}{\secname: \subsecname}
    De~\eqref{eq:equilibrio-potencia}
    sacamos la
    probabilidad de que el sistema
    M/M/1 esté vacío:
    \begin{align}
        \pi_0 &= 1 - \sum_{i=1}^\infty \pi_i\label{eq:prob-cero}\\
              &= 1 - \sum_{i=1}^\infty \pi_0\rho^i\nonumber\\
              &= 1 - \pi_0\frac{1-\rho}{1-\rho}
        \left(\rho+\rho^2+\rho^3+\ldots  \right)\nonumber\\
        &= 1 - \pi_0\frac{1}{1-\rho}
        \lim_{\iota\to\infty}\left( \rho - \rho^2 + \rho^2 - \rho^3 + \rho^3 - \ldots - \rho^\iota \right)\nonumber\\
        &= 1 - \pi_0 \frac{\rho}{1-\rho}\nonumber
    \end{align}
    siempre y cuando $\rho<1$.
\end{frame}





\begin{frame}{\secname: \subsecname}
    Despejando en \eqref{eq:prob-cero}
    y \eqref{eq:equilibrio-potencia}
    obtenemos que

    \begin{lema}[Probabilidades estado M/M/1]
        En un sistema M/M/1 la probabilidad
        de estar en el estado $i$ es:
        \begin{equation}
            \pi_i =
            \begin{cases}
                1-\rho, & i=0\\
                (1-\rho)\rho^i, & i>0
            \end{cases}
        \end{equation}
        donde $\rho=\tfrac{\lambda}{\mu}$
        es la carga del sistema.
    \end{lema}
\end{frame}




\begin{frame}{\secname: \subsecname}
    \textit{Ejemplo}: en una gasolinera llegan
    $\lambda$ [coches/min] a un surtidor
    que sirve a tasa $\mu=1$ [coches/min].

    ¿Cómo varia la probabilidad
    de tener $i=3$ coches en función de
    $\lambda$?

    \vfill

    \begin{figure}
        \begin{tikzpicture}

\begin{axis}[every axis plot post/.append style={
  mark=none,domain=0:1,samples=20},
  axis x line*=bottom,
  axis y line*=left,
  xlabel={$\lambda$ [coches/min]},
  ylabel={$\pi_3$},
  ylabel style={at={(axis description cs:-.075,0.5)},anchor=south},
  enlargelimits=upper,
  width=4in,
  height=2in]
  \addplot[ultra thick,smooth,color=HotPink3] {(1-x)*x^3} node [pos=0.4,anchor=south east] {$(1-\rho)\rho^3$};
\end{axis}


\end{tikzpicture}


    \end{figure}
\end{frame}



\begin{frame}{\secname: \subsecname}
    \textit{Ejemplo (cont.)}: ¿cuántos coches
    aguanta el surtidor para que el 90\%
    de las veces tenga menos de 5 coches?

    \begin{equation*}
        \lambda: \sum_{i=0}^4 \pi_i = \sum_{i=0}^4 (1-\rho)\rho^i = 1-\rho^5  \geq 0.9
        \implies\lambda\leq \sqrt[5]{0.1} \text{ [coches/min]}
    \end{equation*}

    \begin{figure}
        \begin{tikzpicture}

\begin{axis}[every axis plot post/.append style={
  mark=none,samples=20},
  axis x line*=bottom,
  axis y line*=left,
  xlabel={$\lambda$ [coches/min]},
  ylabel={$\sum_{i=0}^4\pi_i$},
  enlargelimits=upper,
  width=4in,
  height=2in]

  \addplot+[smooth,color=HotPink3,domain=0:0.1^(1/5),name path=up] {1-x^5};
  \addplot[domain=0:1,ultra thick,smooth,color=HotPink3] {1-x^5} node [pos=0.6,anchor=south west] {$1-\rho^5$};


  \addplot[domain=0:0.1^(1/5),dashed, thick, black,name path=down] {0.9} node[pos=0.2,anchor=north] {0.9};


  \addplot +[mark=none] coordinates {(0.1^(1/5), 0) (0.1^(1/5), 1)} node[pos=0.5,rotate=90,anchor=south] {$\lambda=\sqrt[5]{0.1}$};



  \addplot[domain=0:1,Firebrick3!20] fill between[of=down and up];

\end{axis}


\end{tikzpicture}


    \end{figure}
\end{frame}




\subsection{Métricas famosas}

\begin{frame}{\secname: \subsecname}
    \begin{lemma}[Número medio de usuarios
        en un M/M/1]
        El número medio de usuarios en
        un sistema M/M/1 es
        \begin{equation}
            \mathbb{E}[N(t)]=\frac{\rho}{1-\rho}
        \end{equation}
    \end{lemma}

    \vfill

    \textit{Demostración}:
    \begin{equation*}
        \mathbb{E}[N(t)]=
        \sum_{i=0}^\infty i \pi_i
        =(1-\rho)\rho \sum_{i=1}^\infty i\rho^{i-1}
        = (1-\rho)\rho \frac{d}{dt}\sum_{i=1}^\infty \rho^i = \ldots = \frac{\rho}{1-\rho}
    \end{equation*}
\end{frame}



\begin{frame}{\secname: \subsecname}
    \begin{lemma}[Número medio de usuarios
        encolados en un M/M/1]
        El número medio de usuarios encolados
        en un sistema M/M/1 es
        \begin{equation}
            \mathbb{E}[Q(t)]=\frac{\rho^2}{1-\rho}
        \end{equation}

    \end{lemma}

    \vfill
    \textit{Demostración}:
    \begin{multline*}
        \mathbb{E}[Q(t)]=
        \sum_{i=1}^\infty (i-1) \pi_i
        = \sum_{i=1}^\infty i\pi_i -
        \sum_{i=1}^\infty \pi_i\\
        = \mathbb{E}[N(t)] - (1-\pi_0)
        = \frac{\rho}{1-\rho} - \rho
        = \frac{\rho^2}{1-\rho}
    \end{multline*}
\end{frame}




\begin{frame}{\secname: \subsecname}
    \textit{Ejemplo (cont.)}:
    si llegan $\lambda=1$ [coches/min],
    ¿cómo de rápido debe ser el surtidor
    para que, en media,
    haya menos de 3 coches esperando?

    \begin{equation}
        \mu: \mathbb{E}[Q(t)] = \frac{1}{\mu^2-\mu}\leq 2 \implies \mu\geq \frac{1+\sqrt{3}}{2}\text{ [coches/min]}
    \end{equation}


    \begin{figure}
        \begin{tikzpicture}

\begin{axis}[every axis plot post/.append style={
  mark=none,samples=100},
  axis x line*=bottom,
  axis y line*=left,
  xlabel={$\mu$ [coches/min]},
  ylabel={$\mathbb{E}[Q(t)]$},
  ymin=0,
  ymax=4,
  xmin=0.5,
  enlargelimits=upper,
  width=4in,
  height=2in]

  \addplot +[smooth,color=HotPink3,domain=(1+sqrt(3))/2:3, name path=up] {1/(x^2-x)};
  \addplot +[color=black,domain=(1+sqrt(3))/2:3, name path=down] {0};

  \addplot[ultra thick,smooth,color=HotPink3,domain=1.1:3] {1/(x^2-x)} node [pos=0.6,anchor=south west] {$\frac{1}{\mu^2-\mu}$};

  \addplot +[dotted,black,mark=none] coordinates {(1,0) (1,10)};

  \addplot +[black,mark=none] coordinates
      {((1+sqrt(3))/2,0) ((1+sqrt(3))/2,2.5)}
      node[pos=.4,rotate=90,anchor=south] {\tiny $\mu=\tfrac{1+\sqrt{3}}{2}$};


  \addplot[domain=1.1:3,dashed, thick, black] {2} node[pos=0.9,anchor=south] {2};


  \addplot[domain=0:1,Firebrick3!20] fill between[of=down and up];

\end{axis}


\end{tikzpicture}


    \end{figure}
\end{frame}






\begin{frame}{\secname: \subsecname}
    ¿Y si queremos sacar el tiempo 
    de espera en cola, o en ser servido?
    \pause

    \begin{thm}[Teorema de Little]
        En un sistema de colas,
        la relación entre tiempo medio
        de servicio $\mathbb{E}[T(t)]$
        y número medio de usuarios es
        \begin{equation}
            \mathbb{E}[N(t)] =
            \mathbb{E}[T(t)]\cdot \lambda
        \end{equation}
        Del mismo modo, la relación
        entre tiempo medio de espera
        en cola $\mathbb{E}[W(t)]$ y
        número medio de usuario en colas es
        \begin{equation}
            \mathbb{E}[Q(t)] =
            \mathbb{E}[W(t)]\cdot \lambda
        \end{equation}
    \end{thm}
\end{frame}



\begin{frame}{\secname: \subsecname}
    Si $\rho\to1$, el tiempo medio de servicio
    $\mathbb{E}[T(t)]\to\infty$.

    \vfill

    \begin{figure}
        \begin{tikzpicture}

\begin{axis}[every axis plot post/.append style={
  mark=none,samples=30},
  axis x line*=bottom,
  axis y line*=left,
  xlabel={$\rho$},
  ylabel={$\mathbb{E}[T(t)]$ [min]},
  ymin=0,
  enlargelimits=upper,
  width=4in,
  height=2in]

  \addplot[domain=0:.95,ultra thick,smooth,color=HotPink3] {x/(1-x)} node [pos=0.1,anchor=south east] {$\frac{\rho}{1-\rho}$};


  \addplot+[thick,dashed,black,mark=none] coordinates {(1, 0) (1, 20)};



\end{axis}


\end{tikzpicture}


    \end{figure}

\end{frame}




\begin{frame}{\secname: \subsecname}
    \textit{Ejemplo}: ¿cuál es la
    cantidad máxima de coches
    que agunta la gasolinera
    para que, en media, un coche tarde
    menos de 5 [min] en repostar?

    \begin{equation}
        \lambda: \mathbb{E}[T(t)] =
        \frac{1}{\mu-\lambda} \leq 5
        \implies \lambda \leq \frac{4}{5}
        \text{ [coches/min]}
    \end{equation}

    \begin{figure}
        \begin{tikzpicture}

\begin{axis}[every axis plot post/.append style={
  mark=none,samples=20},
  axis x line*=bottom,
  axis y line*=left,
  xlabel={$\lambda$ [coches/min]},
  ylabel={$\mathbb{E}[T(t)]$ [min]},
  ymin=0,
  enlargelimits=upper,
  width=4in,
  height=2in]

  \addplot +[domain=0:4/5,color=HotPink3, name path=up] {1/(1-x)};
  \addplot[domain=0:.9,ultra thick,smooth,color=HotPink3] {1/(1-x)} node [pos=0.1,anchor=south east] {$\frac{1}{\mu-\lambda}$};


  \addplot +[black,domain=0:4/5,name path=down] {0};


  \addplot +[mark=none] coordinates {(4/5, 0) (4/5, 10)} node[pos=0.8,rotate=90,anchor=south] {$\lambda=\tfrac{4}{5}$};


  \addplot[domain=0:.9,dashed, black] {5}
      node[pos=0.95,anchor=north] {5};

  \addplot[domain=0:4/5,Firebrick3!20] fill between[of=down and up];



\end{axis}


\end{tikzpicture}


    \end{figure}

\end{frame}



\begin{frame}[allowframebreaks]
        \frametitle{Referencias}
        \bibliographystyle{amsalpha}
        \bibliography{refs.bib}
\end{frame}


\end{document}
