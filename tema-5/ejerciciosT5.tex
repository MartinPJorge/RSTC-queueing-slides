\documentclass[xcolor={x11names}]{beamer}
\usetheme{Madrid}

\usepackage{amssymb}
\usepackage{ulem}
\usepackage[utf8]{inputenc}
\usepackage{mathtools}
\usepackage{multicol}
%\usepackage[x11names]{xcolor}
\usefonttheme{professionalfonts}
\usepackage{amsthm}


% Subfigures
\usepackage{caption}
\usepackage{subcaption}


% License
\usepackage[
    type={CC},
    modifier={by-nc-sa},
    version={3.0},
    imagewidth=4pt
]{doclicense}




% Change base colour beamer@blendedblue (originally RGB: 0.2,0.2,0.7)
\colorlet{beamer@blendedblue}{DarkSeaGreen4}







%% MATH commands
\DeclareMathOperator{\Var}{Var}


%% THEOREMS
%\newtheorem{theorem}{Theorem}
\newtheorem{thm}{Teorema}[section] % the main one
% Definición
%\theoremstyle{definition}
\newtheorem{definicion}{Definición}[section]
\newtheorem{lema}{Lema}[section]
\newtheorem{solucion}{Solución}


%% PFGplots %%
\usepackage{pgfplots}

%% Exponential distribution
\pgfmathdeclarefunction{exponential}{1}{%
  \pgfmathparse{(#1)*exp(-#1*x)}%
}
\pgfmathdeclarefunction{exponentialcdf}{1}{%
  \pgfmathparse{1-exp(-#1*x)}%
}

%% Poisson distribution
\pgfmathdeclarefunction{poiss}{1}{%
  \pgfmathparse{(#1^x)*exp(-#1)/(x!)}%
}

%% Normal distribution (#1=mu, #2=sigma)
% John D. Cook approx. https://tex.stackexchange.com/a/124629
\pgfmathdeclarefunction{normalcdf}{2}{%
  \pgfmathparse{1/(1 + exp(-0.07056*((x-#1)/#2)^3 - 1.5976*(x-#1)/#2))}%
}




\newcommand{\red}[1]{{\color{red}#1}}
\newcommand{\blue}[1]{{\color{blue}#1}}

%%%%%%%%%%
%% TIKZ %%
%%%%%%%%%%
\usepackage{tikz}
\usepackage{animate}
\usetikzlibrary{positioning}
\usetikzlibrary{shapes,arrows, positioning, calc}
\usetikzlibrary{overlay-beamer-styles}
\usetikzlibrary{chains,shapes.multipart}
\usetikzlibrary{scopes}
\usetikzlibrary{automata}
\usetikzlibrary{positioning}  %                 ...positioning nodes
\usetikzlibrary{arrows}       %                 ...customizing arrows
\usetikzlibrary{intersections}


%%%%%%%%%
%% PGF %%
%%%%%%%%%
\usepgfplotslibrary{fillbetween}


%%% Insert section name before the section %%%
\AtBeginSection[]{
  \begin{frame}
  \vfill
  \centering
  \begin{beamercolorbox}[sep=8pt,center,shadow=true,rounded=true]{title}
    \usebeamerfont{title}\insertsectionhead\par%
  \end{beamercolorbox}
  \vfill
  \end{frame}
}



\title[Ejercicios Tema 5]{Ejercicios Tema 5: Introducción al Teletráfico y Teoría de Colas}
%% \subtitle{Redes y Servicios de Telecomunicaciones (RSTC)\\
%% Grado en Ingeniería de Tecnologías y Servicios de Telecomunicación}
%\author{M. Saiful Bari\inst{1} \and Mr X\inst{2}}

\titlegraphic{%
\includegraphics[height=2.5cm]{figs/RSTC-grande.png}\\%
\doclicenseIcon {\tiny \hspace{1em}\doclicenseText}}


\author{\textcolor{white}{RSTC curso 2022-2023}}
%\author{Jorge Martín Pérez\inst{1}}
%\institute{
%    \inst{1}
%    Departamento de Ingeniería Telemática, Universidad Politécnica de Madrid
%}

\date{\today}







%%%%%%%%%%%%%%%%%%%%
%%% SLIDES START %%%
%%%%%%%%%%%%%%%%%%%%
\begin{document}


%%% TITLE %%%
\frame{\titlepage }



\begin{frame}{Ejercicios}
    \tableofcontents
\end{frame}



\section{Ejercicios distribución Exponencial}
\begin{frame}{Ejercicio 1}
    El retardo de una red de datos en un centro hospitalario se distribuye según una variable aleatoria exponencial de media 12 ms, esto es, $ R \sim exp[E[r]=12] ms $.

    En este contexto, los retardos superiores a 50ms son críticos para la monitorización de ciertos pacientes. Con esta información se pide:
    \begin{itemize}
        \item[A] Determinar si es posible garantizar un 1\% de probabilidad de obtener retardos superiores a 50ms.
        \item[B] ¿Es posible garantizar el caso anterior con un retardo medio E[r]=10ms?
    \end{itemize}
\end{frame}

\begin{frame}{Ejercicio 1}
    \begin{solucion}[1.A]
        Del enunciado deducimos que $\red{\lambda = 1/12}$ y por tanto, $ \mathbb{P}(\blue{r>50})= e^{\red{-\frac{1}{12}}\cdot \blue{50}}=0.015 $ concluyendo que no es posible aplicar dicha red al contexto crítico del enunciado.
    \end{solucion}
    \begin{solucion}[1.B]
        Del enunciado deducimos que $\red{\lambda = 1/10}$ y por tanto, $ \mathbb{P}(\blue{r>50})= e^{\red{-\frac{1}{10}}\cdot \blue{50}}=0.00673 $ concluyendo que si es posible aplicar dicha red al contexto crítico del enunciado.
    \end{solucion}
\end{frame}

\end{document}
