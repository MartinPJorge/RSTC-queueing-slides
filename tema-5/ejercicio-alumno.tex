\documentclass{upmassignment}
\usepackage[spanish]{babel}
\usepackage{ifthen}
\usepackage{amsmath}
\usepackage{amsfonts}



% Para mostrar/ocultar soluciones
\newboolean{show}
%\setboolean{show}{true}
\setboolean{show}{false}
\usepackage{environ}
\NewEnviron{solucion}{
  \ifshow
      \begin{answer}\BODY\end{answer}
  \fi}






\coursetitle{Creating assignments}
\courselabel{RSTC}
\exercisesheet{Ejercicio de Alumno}{Tema 5}
\student{\ }%
\semester{Segundo Semestre}
\date{\today}
\university{Universidad Politécnica de Madrid}
\school{Departamento de Ingeniería de Sistemas Telemáticos}
%\usepackage[pdftex]{graphicx}
%\usepackage{subfigure}


\setlength{\textwidth}{5.0in}
\linespread{1.3}
\renewcommand{\PB}{{\bfseries Problema}}















\begin{document}

Se pretende comprar un router de cola infinita
para conmutar el tráfico generado por los servicios telemáticos desplegados en un hospital de grandes dimensiones.
El router tarda un tiempo exponencial de tasa
$\mu$ en conmutar paquetes.

El hospital donde opera el router está dividido en 100 áreas, cada una de ellas con unas características determinadas desde el punto de vista del tráfico generado.

En primer lugar, tenemos la sala UCI, donde los sistemas de monitorización de constantes vitales generan 7 [paquetes/ms]. La información generada desde la zona de oncología se puede modelar según una variable aleatoria uniformemente distribuida entre 0 y 4000 [paquetes/sec]. Por su parte, tanto la zona de hematología como pediatría oncológica generan paquetes de información según una variable aleatoria de distribución normal de media 5 [paquetes/ms] y desviación típica~1~[paquete/ms]. El tráfico desde las salas de Urgencias y de espera puede modelarse mediante distribuciones de poisson de parámetro $\lambda=4$ [paquetes/ms]. El resto de las áreas siguen una distribución uniforme de media 2 [paquetes/ms]. 

El router que se va a comprar tendría una
carga $\rho=\tfrac{43}{60}$ si recibe todo
el tráfico del hospital. Se pregunta:

\begin{problemlist}
    \pbitem ¿Cuál es la tasa de servicio $\mu$
    del router que se va a comprar?

    \begin{solucion}
        \input{solucion-problema-1.tex}
    \end{solucion}

    \pbitem Calcule el tiempo que tarda
    un paquete en atravesar el router comprado.
    Calcule también tiempo medio de espera
    en la cola del router
    y probabilidad de que el router pierda
    tráfico.


    \begin{solucion}
        \input{solucion-problema-2.tex}
    \end{solucion}

    \pbitem Suponiendo que los retardos asociados a la cola de procesamiento/espera del servidor cuando alberga más de 100 peticiones suponen un colapso del sistema de información del hospital, ¿qué probabilidad de colapso ofrece el sistema implementado?

    \begin{solucion}
        \input{solucion-problema-3.tex}
    \end{solucion}


    \pbitem La salida de paquetes en una cola
    M/M/1 sigue un proceso de Poisson de tasa
    $\mu-\lambda$. Por tanto, el tiempo que
    pasa un paquete en el router es una v.a. exponencial de tasa $\mu-\lambda$. Con esta información, ¿qué capacidad de tasa de conmutación
    $\mu$ debe tener el router para que haya una probabilidad del 99.999\% de que un paquete se sirva en menos de 1 [ms]?

    \begin{solucion}
        \input{solucion-problema-4.tex}
    \end{solucion}
\end{problemlist}

\end{document}


