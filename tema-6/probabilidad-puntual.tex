\documentclass{upmassignment}
\usepackage[spanish]{babel}
\usepackage{ifthen}
\usepackage{amsmath}
\usepackage{amsfonts}



% Para mostrar/ocultar soluciones
\newboolean{show}
\setboolean{show}{true}
%\setboolean{show}{false}
\usepackage{environ}
\NewEnviron{solucion}{
  \ifshow
      \begin{answer}\BODY\end{answer}
  \fi}






\coursetitle{Creating assignments}
\courselabel{RSTC}
\exercisesheet{Probabilidad puntual}{Repaso probabilidad}
\student{\ }%
\semester{Segundo Semestre}
\date{\today}
\university{Universidad Politécnica de Madrid}
\school{Departamento de Ingeniería de Sistemas Telemáticos}
%\usepackage[pdftex]{graphicx}
%\usepackage{subfigure}


\setlength{\textwidth}{5.0in}
\linespread{1.3}
\renewcommand{\PB}{{\bfseries Problema}}















\begin{document}


\begin{problemlist}
    \pbitem 
        Sea $t$ una v.a. que sigue una
        distribución $Exp(\lambda)$, vamos
        a estudiar la probabilidad de que
        la v.a. tome el valor $\tau$,
        es decir, $\mathbb{P}(t=\tau)$:

    \begin{solucion}
        Sabemos que
        \begin{align*}
            \mathbb{P}(t=\tau)=&
            \lim_{\varepsilon\to0}
            \mathbb{P}(t\in[\tau,
            \tau+\varepsilon])\\
            =& \lim_{\varepsilon\to0}
            \int_{\tau}^{\tau+\varepsilon}
            dF_t(\theta)\\
            \underbrace{=}_{F_t(\theta)=
            1-e^{-\lambda\theta}}&
            \lim_{\varepsilon\to0}
            \int_{\tau}^{\tau+\varepsilon}
            \lambda e^{-\lambda\theta}
            \ d\theta\\
            =& \lim_{\varepsilon\to0}
            -e^{-\lambda(\tau+\varepsilon)}
            +e^{-\lambda\tau}=0
        \end{align*}

        De esta formulación debemos sacar
        la conclusión de que la probabilidad
        puntual de una v.a. contínua es 0.
    \end{solucion}

\end{problemlist}


\end{document}


