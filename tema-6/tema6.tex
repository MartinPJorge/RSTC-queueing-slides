\documentclass[xcolor={x11names}]{beamer}
\usetheme{Madrid}

\usepackage{amssymb}
\usepackage{ulem}
\usepackage[utf8]{inputenc}
\usepackage{mathtools}
\usepackage{multicol}
%\usepackage[x11names]{xcolor}
\usefonttheme{professionalfonts}


% Subfigures
\usepackage{caption}
\usepackage{subcaption}




% Change base colour beamer@blendedblue (originally RGB: 0.2,0.2,0.7)
\colorlet{beamer@blendedblue}{DarkSeaGreen4}







%% MATH commands
\DeclareMathOperator{\Var}{Var}


%% THEOREMS
%\newtheorem{theorem}{Theorem}
\newtheorem{thm}{Teorema}[section] % the main one
% Definición
%\theoremstyle{definition}
\newtheorem{definicion}{Definición}[section]
\newtheorem{lema}{Lema}[section]



%% PFGplots %%
\usepackage{pgfplots}

%% Exponential distribution
\pgfmathdeclarefunction{exponential}{1}{%
  \pgfmathparse{(#1)*exp(-#1*x)}%
}
\pgfmathdeclarefunction{exponentialcdf}{1}{%
  \pgfmathparse{1-exp(-#1*x)}%
}

%% Poisson distribution
\pgfmathdeclarefunction{poiss}{1}{%
  \pgfmathparse{(#1^x)*exp(-#1)/(x!)}%
}

%% Normal distribution (#1=mu, #2=sigma)
% John D. Cook approx. https://tex.stackexchange.com/a/124629
\pgfmathdeclarefunction{normalcdf}{2}{%
  \pgfmathparse{1/(1 + exp(-0.07056*((x-#1)/#2)^3 - 1.5976*(x-#1)/#2))}%
}




\newcommand{\red}[1]{{\color{red}#1}}
\newcommand{\blue}[1]{{\color{blue}#1}}

%%%%%%%%%%
%% TIKZ %%
%%%%%%%%%%
\usepackage{tikz}
\usepackage{animate}
\usetikzlibrary{positioning}
\usetikzlibrary{shapes,arrows, positioning, calc}
\usetikzlibrary{overlay-beamer-styles}
\usetikzlibrary{chains,shapes.multipart}
\usetikzlibrary{scopes}
\usetikzlibrary{automata}
\usetikzlibrary{positioning}  %                 ...positioning nodes
\usetikzlibrary{arrows}       %                 ...customizing arrows
\usetikzlibrary{intersections}


%%%%%%%%%
%% PGF %%
%%%%%%%%%
\usepgfplotslibrary{fillbetween}


%%% Insert section name before the section %%%
\AtBeginSection[]{
  \begin{frame}
  \vfill
  \centering
  \begin{beamercolorbox}[sep=8pt,center,shadow=true,rounded=true]{title}
    \usebeamerfont{title}\insertsectionhead\par%
  \end{beamercolorbox}
  \vfill
  \end{frame}
}



\title[Tema 6]{Tema 6: Teletráfico en redes de datos}
%\author{M. Saiful Bari\inst{1} \and Mr X\inst{2}}

\author{Jorge Martín Pérez\inst{1}}
\institute{
    \inst{1}
    Departamento de Ingeniería Telemática, Universidad Politécnica de Madrid
}

\date{\today}







%%%%%%%%%%%%%%%%%%%%
%%% SLIDES START %%%
%%%%%%%%%%%%%%%%%%%%
\begin{document}


%%% TITLE %%%
\frame{\titlepage}


\begin{frame}{Contenido}
    \tableofcontents
\end{frame}




\section{Introducción}
\begin{frame}{\secname}
    Hemos visto colas M/M/1
    \begin{figure}
        
\begin{tikzpicture}[start chain=going right,>=latex,node distance=0pt]
% the rectangular shape with vertical lines
\node[rectangle split, rectangle split parts=6,
draw, rectangle split horizontal,text height=1cm,text depth=0.5cm,on chain,inner ysep=0pt] (wa) {};
\fill[white] ([xshift=-\pgflinewidth,yshift=-\pgflinewidth]wa.north west) rectangle ([xshift=-15pt,yshift=\pgflinewidth]wa.south);

% the circle
\node[draw,circle,on chain,minimum size=1.5cm] (se) {$\mu$};

% the arrows and labels
\draw[->] (se.east) -- +(20pt,0);
\draw[<-] (wa.west) -- +(-20pt,0) node[left] {$\lambda$};
\node[align=center,below] at (wa.south) {Waiting \\ Area};
\node[align=center,below] at (se.south) {Service \\ Node};
\end{tikzpicture}

    \end{figure}
    con tiempos:
    \begin{itemize}
        \item de llegada exponenciales
            $t_l\sim Exp(\lambda)$
        \item de servicio exponenciales
            $t_s\sim Exp(\mu)$
    \end{itemize}

    \pause
    \vfill

    {\color{blue}
    Pero, ¿y si el tiempo de servicio $t_s$
    sigue otra distribución?
    }
     
    \pause
    \begin{itemize}
        {\color{red}
        \item sistema M/G/1
        }
    \end{itemize}
\end{frame}



\begin{frame}{\secname}
    Hemos estudiado una sola cola
    \begin{figure}
        
\begin{tikzpicture}[start chain=going right,>=latex,node distance=0pt]
% the rectangular shape with vertical lines
\node[rectangle split, rectangle split parts=6,
draw, rectangle split horizontal,text height=1cm,text depth=0.5cm,on chain,inner ysep=0pt] (wa) {};
\fill[white] ([xshift=-\pgflinewidth,yshift=-\pgflinewidth]wa.north west) rectangle ([xshift=-15pt,yshift=\pgflinewidth]wa.south);

% the circle
\node[draw,circle,on chain,minimum size=1.5cm] (se) {$\mu$};

% the arrows and labels
\draw[->] (se.east) -- +(20pt,0);
\draw[<-] (wa.west) -- +(-20pt,0) node[left] {$\lambda$};
\node[align=center,below] at (wa.south) {Waiting \\ Area};
\node[align=center,below] at (se.south) {Service \\ Node};
\end{tikzpicture}

    \end{figure}

    \vfill
    {\color{blue}
    Pero, ¿y si hay más colas?
    }
    \pause
    \begin{itemize}
        {\color{red}
        \item redes de Jackson
        }
    \end{itemize}
    \begin{figure}
        \begin{tikzpicture}[color=Firebrick2,thick]


% the rectangular shape with vertical lines
\node[rectangle split, rectangle split parts=6,
draw, rectangle split horizontal,text height=1cm,text depth=0.5cm,inner ysep=0pt] (wa) {};
\fill[white] ([xshift=-\pgflinewidth,yshift=-\pgflinewidth]wa.north west) rectangle ([xshift=-15pt,yshift=\pgflinewidth]wa.south);

% the circle
\node[draw,circle,minimum size=1.5cm,anchor=west] (se) at (wa.east) {$\mu$};

% the arrows and labels
\draw[->] (se.east) -- +(20pt,0) node (seEnd) {};
\draw[<-] (wa.west) -- +(-20pt,0) node[left] {$\lambda$};




%%%%%%%%%%%%%%%%%%%
%%% Second queue
%%%%%%%%%%%%%%%%%%%

% the rectangular shape with vertical lines
\node[rectangle split, rectangle split parts=6,
    draw, rectangle split horizontal,text height=1cm,text depth=0.5cm,inner ysep=0pt,anchor=west] (wa2) at (seEnd.east) {};
\fill[white] ([xshift=-\pgflinewidth,yshift=-\pgflinewidth]wa2.north west) rectangle ([xshift=-15pt,yshift=\pgflinewidth]wa2.south);

% the circle
\node[draw,circle,minimum size=1.5cm,anchor=west] (se2) at (wa2.east) {$\mu$};

% the arrows and labels
\draw[->] (se2.east) -- +(20pt,0) node (seEnd2) {};


\end{tikzpicture}

    \end{figure}
\end{frame}


\begin{frame}{Contenido}
    \tableofcontents
\end{frame}


\section{Sistema M/G/1}
\subsection{No Markoviano}
\begin{frame}{\secname: \subsecname}
    Tiempo de servicio sigue una distribución
    general\footnote{Por ejemplo,
    $G(\mu)=U(\tfrac{1}{2\mu}, \tfrac{2}{3\mu})$}
    $t_s\sim G(\mu)$.

    \begin{figure}
        
\begin{tikzpicture}[start chain=going right,>=latex,node distance=0pt]
% the rectangular shape with vertical lines
\node[rectangle split, rectangle split parts=6,
draw, rectangle split horizontal,text height=1cm,text depth=0.5cm,on chain,inner ysep=0pt] (wa) {};
\fill[white] ([xshift=-\pgflinewidth,yshift=-\pgflinewidth]wa.north west) rectangle ([xshift=-15pt,yshift=\pgflinewidth]wa.south);

% the circle
\node[draw,circle,on chain,minimum size=1.5cm] (se) {$\mu$};

% the arrows and labels
\draw[->] (se.east) -- +(20pt,0);
\draw[<-] (wa.west) -- +(-20pt,0) node[left] {$\lambda$};
\node[align=center,below] at (wa.south) {Waiting \\ Area};
\node[align=center,below] at (se.south) {Service \\ Node};
\end{tikzpicture}

    \end{figure}

    \vfill
    Para modelar como cadena de Markov
    es necesario que
    \begin{itemize}
        \item tiempo estancia en estado
            $t_i\sim Exp(\nu_i)$.
    \end{itemize}
\end{frame}



\begin{frame}{\secname: \subsecname}
    \begin{figure}
        %%% \tikzset{node distance=1cm, % Minimum distance between two nodes. Change if necessary.
%%%          every state/.style={ % Sets the properties for each state
%%%            semithick,draw=HotPink3!50,
%%%            fill=HotPink3!20},
%%%          initial text={},     % No label on start arrow
%%%          double distance=4pt, % Adjust appearance of accept states
%%%          every edge/.style={  % Sets the properties for each transition
%%%          draw, ->,>=stealth',     % Makes edges directed with bold arrowheads
%%%            auto, thick},
%%% }
\begin{tikzpicture}
   

\draw[|->] (0,0) -- (5,0) node[anchor=west] {$t$};

% Estados
\node[state,fill=HotPink3!20] (current)
    at (0,1.3) {$i=3$};
\node[dashed,state,fill=DodgerBlue4!20] (arrive)
    at (2,2) {$i=4$};
\node[dashed,state,fill=DodgerBlue1!20] (exit)
    at (4,2) {$i=2$};

% Posibles eventos
\draw[thick,dashed,<-] (2,0) to (arrive.south);
\draw[thick,dashed,->] (4,0) to (exit.south);

% Tiempos hasta eventos
\draw[|-|,thick,gray,DodgerBlue4] (0,0.2) -- node[pos=.5,fill=white,inner sep=0] {$t_l$} (1.95,0.2);
\draw[|-|,thick,gray,DodgerBlue1] (0,0.5) -- node[pos=.8,fill=white,inner sep=0] {$t_s$} (3.95,0.5);

\node (legend) at (3,3) 
    {\tiny posibles transiciones};
\draw[->] (legend.east)
    to[out=0,in=90] (exit.north);
\draw[->] (legend.west) 
    to[out=180,in=90] (arrive.north);
    
\end{tikzpicture}

    \end{figure}

    \vfill
    Veamos si se cumple que $t_i\sim Exp(\nu_i)$:
    \begin{multline}
        \mathbb{P}(t_i>\tau) =
        \mathbb{P}(\min\{t_l,t_s\}>\tau)=
        \mathbb{P}(t_l>\tau)
        \mathbb{P}(t_s>\tau)\\
        =
        \left(1 - \frac{\tau}{\mu} \right)
        e^{-\mu\tau} \neq e^{-\nu_i \tau}
    \end{multline}
    con $t_s\sim G(\mu)=U\left(\tfrac{1}{2\mu},
    \tfrac{2}{3\mu}\right),\
    \tau\in\left[\tfrac{1}{2\mu},
    \tfrac{2}{3\mu}\right]$.
\end{frame}





\section{Redes de Jackson}





\end{document}
