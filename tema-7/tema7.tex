\documentclass[xcolor={x11names}]{beamer}
\usetheme{Madrid}

\usepackage{amssymb}
\usepackage{ulem}
\usepackage[utf8]{inputenc}
\usepackage{mathtools}
\usepackage{multicol}
%\usepackage[x11names]{xcolor}
\usefonttheme{professionalfonts}



% Subfigures
\usepackage{caption}
\usepackage{subcaption}

% Check/cross mark
\usepackage{pifont}% http://ctan.org/pkg/pifont
\newcommand{\cmark}{\ding{51}}%
\newcommand{\xmark}{\ding{55}}%



% License
\usepackage[
    type={CC},
    modifier={by-nc-sa},
    version={4.0},
    imagewidth=4pt
]{doclicense}




% Change base colour beamer@blendedblue (originally RGB: 0.2,0.2,0.7)
\colorlet{beamer@blendedblue}{DarkSeaGreen4}







%% MATH commands
\DeclareMathOperator{\Var}{Var}


%% THEOREMS
%\newtheorem{theorem}{Theorem}
\newtheorem{thm}{Teorema}[section] % the main one
% Definición
%\theoremstyle{definition}
\newtheorem{definicion}{Definición}[section]
\newtheorem{lema}{Lema}[section]



%% PFGplots %%
\usepackage{pgfplots}

%% Exponential distribution
\pgfmathdeclarefunction{exponential}{1}{%
  \pgfmathparse{(#1)*exp(-#1*x)}%
}
\pgfmathdeclarefunction{exponentialcdf}{1}{%
  \pgfmathparse{1-exp(-#1*x)}%
}

%% Poisson distribution
\pgfmathdeclarefunction{poiss}{1}{%
  \pgfmathparse{(#1^x)*exp(-#1)/(x!)}%
}

%% Normal distribution (#1=mu, #2=sigma)
% John D. Cook approx. https://tex.stackexchange.com/a/124629
\pgfmathdeclarefunction{normalcdf}{2}{%
  \pgfmathparse{1/(1 + exp(-0.07056*((x-#1)/#2)^3 - 1.5976*(x-#1)/#2))}%
}






\newcommand{\red}[1]{{\color{red}#1}}
\newcommand{\blue}[1]{{\color{blue}#1}}

%%%%%%%%%%
%% TIKZ %%
%%%%%%%%%%
\usepackage{tikz}
\usepackage{animate}
\usetikzlibrary{positioning}
\usetikzlibrary{shapes,arrows, positioning, calc}
\usetikzlibrary{overlay-beamer-styles}
\usetikzlibrary{chains,shapes.multipart}
\usetikzlibrary{scopes}
\usetikzlibrary{automata}
\usetikzlibrary{positioning}  %                 ...positioning nodes
\usetikzlibrary{arrows}       %                 ...customizing arrows
\usetikzlibrary{intersections}
% math
\usetikzlibrary{math}

%%%%%%%%%
%% PGF %%
%%%%%%%%%
\usepgfplotslibrary{fillbetween}
\usepackage{pgffor} % loops


%%% Insert section name before the section %%%
\AtBeginSection[]{
  \begin{frame}
  \vfill
  \centering
  \begin{beamercolorbox}[sep=8pt,center,shadow=true,rounded=true]{title}
    \usebeamerfont{title}\insertsectionhead\par%
  \end{beamercolorbox}
  \vfill
  \end{frame}
}



\title[Tema 7]{Tema 7: Teletráfico en redes de telecomunicaciones}
%% \subtitle{Redes y Servicios de Telecomunicaciones (RSTC)\\
%% Grado en Ingeniería de Tecnologías y Servicios de Telecomunicación}
%\author{M. Saiful Bari\inst{1} \and Mr X\inst{2}}

\titlegraphic{%
\includegraphics[height=2.5cm]{../tema-5/figs/RSTC-grande.png}\\%
\doclicenseIcon {\tiny \hspace{1em}\doclicenseText}\\%
\href{https://github.com/MartinPJorge/RSTC-queueing-slides}{\includegraphics[height=1cm]{../tema-5/figs/github-logo.png}}%
}


\author{\textcolor{white}{RSTC curso 2022-2023}}
%\author{Jorge Martín Pérez\inst{1}}
%\institute{
%    \inst{1}
%    Departamento de Ingeniería Telemática, Universidad Politécnica de Madrid
%}

\date{\today}







%%%%%%%%%%%%%%%%%%%%
%%% SLIDES START %%%
%%%%%%%%%%%%%%%%%%%%
\begin{document}


%%% TITLE %%%
\frame{\titlepage}


\begin{frame}[allowframebreaks]{Contenido}
    \tableofcontents
\end{frame}


\section{Sistema M/M/N}
\begin{frame}{\secname}
    Un sistema de cola única
    con $t_l\sim Exp(\lambda)$
    y N servidores en paralelo
    con $t_s\sim Exp(\mu)$.
    \begin{figure}
        \input{figs/mmn.tex}
    \end{figure}
\end{frame}


\subsection{Cadena de Markov}
\begin{frame}{\secname: \subsecname}
    Un sistema M/M/N es un proceso
    estocástico Markoviano\footnote{
    El tiempo de estancia es
    exponencial no homogéneo.}.
    \begin{figure}
        \resizebox{!}{.2\textwidth}{%
            \begin{tikzpicture}
   
% Estados
\node[state,minimum size=4em] (1) {$0$};
\node[state,minimum size=4em] (2) [right=of 1] {$1$};
\node[state,minimum size=4em,fill=HotPink3!20] (3) [right=of 2] {$2$};
\node[state,minimum size=4em] (4) [right=of 3] {$3$};
\node[minimum size=3em] (5) [right=of 4] {$\cdots$};
\node[state,minimum size=4em] (N) [right=of 5] {$N$};
\node[state,minimum size=4em] (N1) [right=of N] {$N+1$};
\node[minimum size=3em] (N2) [right=of N1] {$\cdots$};


% Transiciones
\draw [->,thick] (1.north east) to [bend left=55]  node[above] {$\lambda$}  (2.north west);
\draw [->,thick] (2.south west) to [bend left=55]  node[below] {$\mu$}  (1.south east);

\draw [->,thick] (2.north east) to [bend left=55]  node[above] {$\lambda$}  (3.north west);
\draw [->,thick] (3.south west) to [bend left=55]  node[below] {$2\mu$}  (2.south east);

\draw [->,thick] (3.north east) to [bend left=55]  node[above] {$\lambda$}  (4.north west);
\draw [->,thick] (4.south west) to [bend left=55]  node[below] {$3\mu$}  (3.south east);

\draw [->,thick] (4.north east) to [bend left=55]  node[above] {$\lambda$}  (5.north west);
\draw [->,thick] (5.south west) to [bend left=55]  node[below] {$4\mu$}  (4.south east);


\draw [->,thick] (5.north east) to [bend left=55]  node[above] {$\lambda$}  (N.north west);
\draw [->,thick] (N.south west) to [bend left=55]  node[below] {$N\mu$}  (5.south east);


\draw [->,thick] (N.north east) to [bend left=55]  node[above] {$\lambda$}  (N1.north west);
\draw [->,thick] (N1.south west) to [bend left=55]  node[below] {$N\mu$}  (N.south east);


\draw [->,thick] (N1.north east) to [bend left=55]  node[above] {$\lambda$}  (N2.north west);
\draw [->,thick] (N2.south west) to [bend left=55]  node[below] {$N\mu$}  (N1.south east);

\draw[dashed] (N.north) -- ($(0,1.5)+(N.north)$);
\draw[dashed] (N.south) -- ($(0,-1.5)+(N.south)$);
    
\end{tikzpicture}

        }
    \end{figure}
    Sus tasas de transición no son
    homogéneas $q_{i,j}=i\mu,\ i\leq N$.
    \begin{figure}
        \resizebox{!}{.2\textwidth}{%
            \input{figs/mmn-highlight.tex}
        }
    \end{figure}
\end{frame}



\subsection{Ecuaciones de equilibrio}
\begin{frame}{\secname: \subsecname}
    \begin{figure}
        \resizebox{!}{.2\textwidth}{%
            \begin{tikzpicture}
   
% Estados
\node[state,minimum size=4em] (1) {$0$};
\node[state,minimum size=4em] (2) [right=of 1] {$1$};
\node[state,minimum size=4em,fill=HotPink3!20] (3) [right=of 2] {$2$};
\node[state,minimum size=4em] (4) [right=of 3] {$3$};
\node[minimum size=3em] (5) [right=of 4] {$\cdots$};
\node[state,minimum size=4em] (N) [right=of 5] {$N$};
\node[state,minimum size=4em] (N1) [right=of N] {$N+1$};
\node[minimum size=3em] (N2) [right=of N1] {$\cdots$};


% Transiciones
\draw [->,thick] (1.north east) to [bend left=55]  node[above] {$\lambda$}  (2.north west);
\draw [->,thick] (2.south west) to [bend left=55]  node[below] {$\mu$}  (1.south east);

\draw [->,thick] (2.north east) to [bend left=55]  node[above] {$\lambda$}  (3.north west);
\draw [->,thick] (3.south west) to [bend left=55]  node[below] {$2\mu$}  (2.south east);

\draw [->,thick] (3.north east) to [bend left=55]  node[above] {$\lambda$}  (4.north west);
\draw [->,thick] (4.south west) to [bend left=55]  node[below] {$3\mu$}  (3.south east);

\draw [->,thick] (4.north east) to [bend left=55]  node[above] {$\lambda$}  (5.north west);
\draw [->,thick] (5.south west) to [bend left=55]  node[below] {$4\mu$}  (4.south east);


\draw [->,thick] (5.north east) to [bend left=55]  node[above] {$\lambda$}  (N.north west);
\draw [->,thick] (N.south west) to [bend left=55]  node[below] {$N\mu$}  (5.south east);


\draw [->,thick] (N.north east) to [bend left=55]  node[above] {$\lambda$}  (N1.north west);
\draw [->,thick] (N1.south west) to [bend left=55]  node[below] {$N\mu$}  (N.south east);


\draw [->,thick] (N1.north east) to [bend left=55]  node[above] {$\lambda$}  (N2.north west);
\draw [->,thick] (N2.south west) to [bend left=55]  node[below] {$N\mu$}  (N1.south east);

\draw[dashed] (N.north) -- ($(0,1.5)+(N.north)$);
\draw[dashed] (N.south) -- ($(0,-1.5)+(N.south)$);
    
\end{tikzpicture}

        }
    \end{figure}
    Con $i<N$ tenemos
    \begin{equation*}
        \lambda\pi_{i-1}
        +(i+1)\mu\pi_{i+1}
        = \pi_i(\lambda+i\mu)
    \end{equation*}
    pero con $i\geq N$ tenemos
    \begin{equation*}
        \lambda\pi_{i-1}
        +N\mu\pi_{i+1}
        = \pi_i(\lambda+N\mu)
    \end{equation*}
\end{frame}




\begin{frame}{\secname: \subsecname}
    \begin{lema}[Ecuaciones equilibrio M/M/N]
        En régimen estacionario de un sistema
        M/M/N, la probabilidad de estar en el
        estado $i$ es:
        \begin{equation}
            \pi_i=\begin{cases}
                \frac{A^i}{i!}\pi_0, \quad&
                i\leq N\\
                \rho^i \frac{N^N}{N!}\pi_0,
                \quad& i\geq N
            \end{cases}
        \end{equation}
        con $A=\tfrac{\lambda}{\mu}$,
        $\rho=\tfrac{\lambda}{N\mu}$; y
        \begin{equation}
            \pi_0=\left(
                \left(\sum_{i=0}^{N-1}
                \frac{A^i}{i!}
                \right)
                + \frac{A^N}{N!}\frac{1}{1-\rho}
            \right)^{-1}
        \end{equation}
    \end{lema}
\end{frame}



\begin{frame}{\secname: \subsecname}
    \textit{Demostración}:
    \begin{itemize}
        \item para $i\leq N$ se tiene:
        \begin{figure}
            \resizebox{!}{.1\textwidth}{%
                \begin{tikzpicture}
   
% Estados
\node[state,minimum size=4em] (1) {$0$};
\node[state,minimum size=4em] (2) [right=of 1] {$1$};
\node[state,minimum size=4em,fill=HotPink3!20] (3) [right=of 2] {$2$};
\node[state,minimum size=4em] (4) [right=of 3] {$3$};


% Transiciones
\draw [->,thick] (1.north east) to [bend left=55]  node[above] {$\lambda$}  (2.north west);
\draw [->,thick] (2.south west) to [bend left=55]  node[below] {$\mu$}  (1.south east);

\draw [->,thick] (2.north east) to [bend left=55]  node[above] {$\lambda$}  (3.north west);
\draw [->,thick] (3.south west) to [bend left=55]  node[below] {$2\mu$}  (2.south east);

\draw [->,thick] (3.north east) to [bend left=55]  node[above] {$\lambda$}  (4.north west);
\draw [->,thick] (4.south west) to [bend left=55]  node[below] {$3\mu$}  (3.south east);

\draw [->,thick] (4.north east) to [bend left=55]  node[above] {$\lambda$}  (5.north west);
\draw [->,thick] (5.south west) to [bend left=55]  node[below] {$4\mu$}  (4.south east);

    
\end{tikzpicture}

            }
        \end{figure}
        sabiendo que $\pi_0\lambda=\pi_1\mu$
        se tiene $\pi_1=A\pi_0$.
        Por tanto tenemos que
        \begin{align*}
            \lambda\pi_0+2\mu\pi_2=
            \pi_1(\lambda+\mu)
            \Longleftrightarrow&
            \pi_2=\frac{A^2}{2}\pi_0\\
            %
            \lambda\pi_1+3\mu\pi_3=
            \pi_2(\lambda+2\mu)
            \Longleftrightarrow&
            \pi_3=\frac{A^3}{3!}\pi_0\\
            \ldots\\
            \lambda\pi_{i-1}+(i+1)\mu\pi_{i+1}=
            \pi_i(\lambda+i\mu)
            \Longleftrightarrow&
            \pi_i=\frac{A^i}{i!}\pi_0\\
        \end{align*}
        y deducimos $\pi_i=\tfrac{A}{i}\pi_{i-1}$.
    \end{itemize}
\end{frame}




\begin{frame}{\secname: \subsecname}
    \textit{Demostración}:
    \begin{itemize}
        \item para $i\geq N$ se tiene:
        \begin{figure}
            \resizebox{!}{.2\textwidth}{%
                \begin{tikzpicture}
   
\node[state,minimum size=4em] (5) [right=of 4] {$N-2$};
\node[state,minimum size=4em] (N) [right=of 5] {$N-1$};
\node[state,minimum size=4em] (N1) [right=of N] {$N$};
\node[state,minimum size=3em] (N2) [right=of N1] {$N+1$};




\draw [->,thick] (5.north east) to [bend left=55]  node[above] {$\lambda$}  (N.north west);
\draw [->,thick] (N.south west) to [bend left=55]  node[below] {$(N-1)\mu$}  (5.south east);


\draw [->,thick] (N.north east) to [bend left=55]  node[above] {$\lambda$}  (N1.north west);
\draw [->,thick] (N1.south west) to [bend left=55]  node[below] {$N\mu$}  (N.south east);


\draw [->,thick] (N1.north east) to [bend left=55]  node[above] {$\lambda$}  (N2.north west);
\draw [->,thick] (N2.south west) to [bend left=55]  node[below] {$N\mu$}  (N1.south east);

\draw[dashed] (N1.north) -- ($(0,1.5)+(N1.north)$);
\draw[dashed] (N1.south) -- ($(0,-1.5)+(N1.south)$);
    
\end{tikzpicture}

            }
        \end{figure}
        \begin{align*}
            \pi_{N-1}(\lambda+(N-1)\mu)
            =&\pi_N N\mu + \pi_{N-2}\lambda\\
            \pi_{N}
            =& \frac{1}{N\mu}
            \left(
                \pi_{N-1}\lambda
                + \pi_{N-1}\mu(N-1)
                - \pi_{N-2}\lambda
            \right)\\
        \end{align*}
        sustituyendo $\pi_{N-1}=\tfrac{A}{N-1}
        \pi_{N-2}$, sacamos la recursión
        \begin{equation*}
            \pi_i=\frac{\lambda}{N\mu}\pi_{i-1},
            \quad i\geq N
        \end{equation*}
    \end{itemize}
\end{frame}

\begin{frame}{\secname: \subsecname}
    \textit{Demostración}:
    \begin{itemize}
        \item para $i\geq N$ se tiene:
        \begin{equation*}
            \pi_i=\frac{\lambda}{N\mu}\pi_{i-1},
            \quad i\geq N
        \end{equation*}
        y si usamos $\pi_i=\tfrac{A^i}{i!}\pi_0$
        con $i\leq N$ obtenemos:
        \begin{align*}
            \pi_{i}=&\frac{A}{N}\pi_{i-1}
            =\frac{A^2}{N^2}\pi_{i-2}=
            \cdots
            = \left(
                \frac{A}{N}
            \right)^{i-N}
            \pi_N
            = \left(
                \frac{A}{N}
            \right)^{i-N}
            \frac{A^N}{N!}\pi_0\\
            =& \rho^i \frac{N^N}{N!}\pi_0
        \end{align*}
    \end{itemize}
\end{frame}




\begin{frame}{\secname: \subsecname}
    \textit{Demostración}:
    \begin{itemize}
        \item para $i=0$ se tiene:
            \begin{align*}
                1=&\sum_{i=0}^{N-1}\pi_i
                 + \sum_{i=N}^\infty \pi_i\\
                 =& \sum_{i=0}^{N-1}
                 \frac{A^i}{i!}\pi_0
                 + \sum_{i=N}^\infty
                 \rho^i \frac{N^N}{N!}\pi_0\\
                 =& \pi_0\sum_{i=0}^{N-1}
                 \frac{A^i}{i!}
                 + \pi_0 \frac{N^N}{N!}\sum_{i=N}^\infty
                 \rho^i
            \end{align*}
            sabiendo que
            $\sum_{i=N}^\infty
            \rho^i=\tfrac{\rho^N}{1-\rho}$,
            sacamos
            \begin{equation*}
            \pi_0=\left(
                \left(\sum_{i=0}^{N-1}
                \frac{A^i}{i!}
                \right)
                + \frac{A^N}{N!}\frac{1}{1-\rho}
            \right)^{-1}
            \end{equation*}
    \end{itemize}
\end{frame}



\subsection{Segunda distribución de Earlang}
\begin{frame}{\secname: \subsecname}
    \begin{definicion}[Segunda distribución de
        Earlang]
        Popularmente, la probabilidad de que
        un M/M/N esté vacío
                \begin{equation*}
                \pi_0=\left(
                    \left(\sum_{i=0}^{N-1}
                    \frac{A^i}{i!}
                    \right)
                    + \frac{A^N}{N!}\frac{1}{1-\rho}
                \right)^{-1}
                \end{equation*}
        se conoce como la segunda
        distribución de Earlang, y a la razón
        $A$ se le llaman ``Earlangs''.
    \end{definicion}

    \vfill

    \textit{Ejemplo}: si tenemos tasas
    $\lambda=10$ y $\mu=2$; los Earlangs
    $A=\tfrac{\lambda}{\mu}=5$ me dicen que
    necesito $>5$ servidores; y con $N=7$
    tenemos
    \begin{equation*}
    \pi_0=\left(
        \left(\sum_{i=0}^{6}
        \frac{5^i}{i!}
        \right)
        + \frac{5^7}{7!}\frac{1}{1-\tfrac{5}{7}}
    \right)^{-1}
    \simeq 0.006
    \end{equation*}
\end{frame}




\subsection{Distribución de Earlang-C}
\begin{frame}{\secname: \subsecname}
    \begin{lema}[Earlang-C]
        La probabilidad de esperar en un
        sistema M/M/N viene dada por
        la distribución Earlang-C:
        \begin{equation}
            E_C(N,A)=\mathbb{P}(N(t)\geq N)
            = \frac{A^N}{N!}\frac{1}{1-\rho}\pi_0
        \end{equation}
    \end{lema}
    \begin{figure}
        \resizebox{!}{.2\textwidth}{%
            \usetikzlibrary{calc}
\begin{tikzpicture}


% the rectangular shape with vertical lines
\node[rectangle split, rectangle split parts=6,
draw, rectangle split horizontal,text height=1cm,text depth=0.5cm,inner ysep=0pt] (wa) {};
\fill[white] ([xshift=-\pgflinewidth,yshift=-\pgflinewidth]wa.north west) rectangle ([xshift=-15pt,yshift=\pgflinewidth]wa.south);

% the circle
\node[draw,circle,minimum size=1.5cm,anchor=west,fill=HotPink3!20] (se) at
    ($(wa.east)+(1,2)$) {$\mu$};
\node[draw,circle,minimum size=1.5cm,anchor=west,fill=HotPink3!20] (se2) at
    ($(wa.east)+(1,.3)$) {$\mu$};
\node (dots) at ($(wa.east)+(1.7,-1)$) {$\vdots$};
\node[draw,circle,minimum size=1.5cm,anchor=west,fill=HotPink3!20] (seN) at
    ($(wa.east)+(1,-2.3)$) {$\mu$};

% server labels
\node[anchor=west] at (se.east) {Servidor 1};
\node[anchor=west] at (se2.east) {Servidor 2};
\node[anchor=west] at (seN.east) {Servidor N};

% the arrows and labels
\draw[<-] (wa.west) -- +(-20pt,0) node[left] {$\lambda$};

\draw[->] (wa.east) -- (se.west);
\draw[->] (wa.east) -- (se2.west);
\draw[->] (wa.east) -- (seN.west);







\end{tikzpicture}

        }
    \end{figure}
    \vfill
    \textit{Demostración}:
        \begin{equation}
            \mathbb{P}(N(t)\geq N)=
            \sum_{i=N}^{\infty}\pi_i
            = \sum_{i=N}^{\infty}
            \rho^i \frac{N^N}{N!}\pi_0
            = \frac{N^N}{N!}\frac{\rho^N}{1-\rho}\pi_0
        \end{equation}
\end{frame}




\subsection{Métricas famosas}
\begin{frame}{\secname: \subsecname}
    \begin{lema}[Número medio en cola M/M/N]
        En un sistema M/M/N el número medio
        de usuarios encolados es
        \begin{equation}
            \mathbb{E}[Q(t)]=
            \frac{\rho}{1-\rho}
            E_C(N,A)
        \end{equation}
    \end{lema}
    \vfill
    \textit{Demostración}:
    \begin{multline*}
        \mathbb{E}[Q(t)]=\sum_{i=N}^\infty
        (i-N)\pi_i=
        \sum_{i=N}^\infty \rho^i \frac{N^N}{N!}
        \pi_0(i-N)=
        \frac{N^N}{N!}\pi_0
        \sum_{i=N}^\infty(i-N)\rho^i\\
        = 
        \frac{N^N}{N!}\pi_0
        \left[
            \sum_{i=N}^\infty
            i\rho^i - N\sum_{i=N}^\infty
            \rho^i
        \right]
        = 
        \frac{N^N}{N!}\pi_0
        \left[
            \frac{\rho^{N+1}}{(1-\rho)^2}
        \right]\\
        = \frac{\rho}{1-\rho}E_C(N,A)
    \end{multline*}
\end{frame}



\begin{frame}{\secname: \subsecname}
    \textit{Ejemplo}: sea un sistema con una
    carga de $A=\tfrac{\lambda}{\mu}=3$
    Earlangs, ¿cuántos servidores $N$ hay
    que poner para que, en media,
    haya menos de 10
    usuarios encolados?
    \begin{multline*}
        N:\mathbb{E}[Q(t)]=
        \frac{\rho}{1-\rho}E_C(N,A)
        = \frac{\tfrac{3}{N}}{1-\tfrac{3}{N}}
        \frac{N^N}{N!}\frac{\left(\frac{3}{N}\right)^N}{1-\tfrac{3}{N}}\pi_0\\
        =\frac{N^N}{N!}\frac{\left(\frac{3}{N}\right)^{N+1}}{\left(1-\tfrac{3}{N}\right)^2}
                \left(
                    \left(\sum_{i=0}^{N-1}
                    \frac{3^i}{i!}
                    \right)
                    + \frac{3^N}{N!}\frac{1}{1-\tfrac{3}{N}}
                \right)^{-1}<10
    \end{multline*}
\end{frame}





\begin{frame}{\secname: \subsecname}
    \textit{Ejemplo (cont.)}: vemos que con
    una intensidad de $A=3$ Earlangs necesitamos
    $N>3$ para que $\mathbb{E}[Q(t)]\leq 10$.
    \vfill
    \begin{figure}
        \begin{tikzpicture}[
declare function={free(\k,\I)=\I^\k/(\k!);}
]




%%%% DEFINE SUMATION USING
%https://tex.stackexchange.com/a/526600/62559
\newcounter{isum}
\pgfplotsset{summand/.initial=max}
\pgfmathdeclarefunction{sum}{2}{%
\begingroup%
\pgfkeys{/pgf/fpu,/pgf/fpu/output format=fixed}%
\edef\myfun{\pgfkeysvalueof{/pgfplots/summand}}%
\pgfmathsetmacro{\mysum}{0}%
\pgfmathsetmacro{\myx}{#2}%
\pgfmathtruncatemacro{\imax}{#1}%
\setcounter{isum}{0}%
\loop
\pgfmathsetmacro{\mysum}{\mysum+\myfun(\value{isum},#2)}%
\ifnum\value{isum}<\imax\relax
\stepcounter{isum}\repeat
\pgfmathparse{\mysum}%
\pgfmathsmuggle\pgfmathresult\endgroup%
}%







\begin{axis}[every axis plot post/.append style={
  mark=none,samples=100},
  axis x line*=bottom,
  axis y line*=left,
  xlabel={$A$},
  ylabel={$\mathbb{E}[Q(t)]$},
  %ymode=log,
  ymin=0,
  ymax=10,
  enlargelimits=upper,
  width=4in,
  height=2in]

  % N=3
  \pgfmathsetmacro{\N}{3}
  \addplot+[summand=free,ultra thick, smooth,
          color=HotPink1,domain=0:2.9] {
      \N^\N/\N! * (\x/\N)^(\N+1)/(1-\x/\N)^2 *%
      1/(sum(\N-1,\x) + \x^\N/\N! * 1/(1-\x/\N))}
      node [pos=0.2,anchor=south east] {$N=3$};


  % N=4
  \pgfmathsetmacro{\N}{4}
  \addplot+[summand=free,ultra thick, smooth,
          color=HotPink2,domain=0:3.9] {
      \N^\N/\N! * (\x/\N)^(\N+1)/(1-\x/\N)^2 *%
      1/(sum(\N-1,\x) + \x^\N/\N! * 1/(1-\x/\N))}
      node [pos=0.2,anchor=north west] {$N=4$};



  % N=5
  \pgfmathsetmacro{\N}{5}
  \addplot+[summand=free,ultra thick, smooth,
          color=HotPink3,domain=0:4.9] {
      \N^\N/\N! * (\x/\N)^(\N+1)/(1-\x/\N)^2 *%
      1/(sum(\N-1,\x) + \x^\N/\N! * 1/(1-\x/\N))}
      node [pos=0.1,anchor=north west] {$N=5$};

  \draw[color=black,dashed]
        (axis cs:3,0) -- (axis cs:3,10);



\end{axis}



\end{tikzpicture}


    \end{figure}
\end{frame}





\begin{frame}{\secname: \subsecname}
    \begin{lema}[Número medio usuarios en M/M/N]
        En un sistema M/M/N el tiempo medio
        de usuarios en el sistema es
        \begin{equation}
            \mathbb{E}[N(t)]=
            \mathbb{E}[Q(t)]+A
        \end{equation}
    \end{lema}
    \vfill
    \textit{Demostración}:
    con Little sabemos
    $\mathbb{E}[W(t)]=\tfrac{1}{\lambda}
    \mathbb{E}[Q(t)]$, y también sabemos que
    \begin{equation*}
        \mathbb{E}[T(t)]
        =\mathbb{E}[W(t)]+\mathbb{E}[t_s]
        = \frac{1}{\lambda}\mathbb{E}[Q(t)]
        + \frac{1}{\mu}
    \end{equation*}
    y usando Little de nuevo
    ($\mathbb{E}[N(t)]=\mathbb{E}[T(t)]\lambda$)
    llegamos a
    \begin{equation*}
        \mathbb{E}[N(t)]=
        \mathbb{E}[Q(t)]+A
    \end{equation*}
\end{frame}


\section{Sistema M/M/1/K}
\begin{frame}{\secname}
    Si tuviéramos un servidor con cola
    finita de tamaño $K-1$ hay pérdidas.
    \begin{figure}
        \usetikzlibrary{calc}
\begin{tikzpicture}

\tikzset{onslide/.code args={<#1>#2}{%
    \only<#1>{\pgfkeysalso{#2}} % \pgfkeysalso doesn't change the path
}}

% the rectangular shape with vertical lines
\node[rectangle split, rectangle split parts=6,
draw, rectangle split horizontal,
rectangle split part fill={white,white,white,HotPink3!20},
text height=1cm,text depth=0.5cm,inner ysep=0pt,
onslide=<2->{fill=HotPink3!20}] (wa) {};
\draw[<->] ($(wa.north west)+(0,.3)$)
    -- ($(wa.north east)+(0,.3)$)
    node[midway,above] {$K-1$};
%\fill[white] ([xshift=-\pgflinewidth,yshift=-\pgflinewidth]wa.north west) rectangle ([xshift=-15pt,yshift=\pgflinewidth]wa.south);

% the circle
\node[draw,circle,minimum size=1.5cm,anchor=west,
    fill=HotPink3!20] (se) at
    (wa.east) {$\mu$};

% the arrows and labels
\draw[<-] (wa.west) --
    +(-35pt,0) node (out) {} --
    +(-70pt,0) node[left] (arrive) {$\lambda$};
\draw[->] (se.east) -- ($(se.east)+(.5,0)$);


% draw the exiting
\onslide<3>{
\node[circle,draw=HotPink4,fill=HotPink4,minimum size=2,
    inner sep=0] (outcirc) at (out) {};
\draw[-|,ultra thick,color=HotPink4]
    (arrive) -- 
    (outcirc) --
    +(0,-25pt) node[below,align=center]
    {pérdidas\\(cola llena)};
};




\end{tikzpicture}

    \end{figure}
\end{frame}





\subsection{Cadena de Markov}
\begin{frame}{\secname: \subsecname}
    Un sistema M/M/1/K es un proceso
    estocástico markoviano, por tanto
    se puede modelar con una cadena
    de Markov \emph{finita}.
    \begin{figure}
        \resizebox{!}{.15\textwidth}{%
            \begin{tikzpicture}
   
% Estados
\node[state,minimum size=4em] (1) {$0$};
\node[state,minimum size=4em] (2) [right=of 1] {$1$};
\node[state,minimum size=4em] (3) [right=of 2] {$2$};
\node[state,minimum size=4em] (4) [right=of 3] {$3$};
\node[minimum size=3em] (5) [right=of 4] {$\cdots$};
\node[state,minimum size=4em] (N) [right=of 5] {$K-1$};
\node[state,minimum size=4em,fill=HotPink3!20] (N1) [right=of N] {$K$};


% Transiciones
\draw [->,thick] (1.north east) to [bend left=55]  node[above] {$\lambda$}  (2.north west);
\draw [->,thick] (2.south west) to [bend left=55]  node[below] {$\mu$}  (1.south east);

\draw [->,thick] (2.north east) to [bend left=55]  node[above] {$\lambda$}  (3.north west);
\draw [->,thick] (3.south west) to [bend left=55]  node[below] {$\mu$}  (2.south east);

\draw [->,thick] (3.north east) to [bend left=55]  node[above] {$\lambda$}  (4.north west);
\draw [->,thick] (4.south west) to [bend left=55]  node[below] {$\mu$}  (3.south east);

\draw [->,thick] (4.north east) to [bend left=55]  node[above] {$\lambda$}  (5.north west);
\draw [->,thick] (5.south west) to [bend left=55]  node[below] {$\mu$}  (4.south east);


\draw [->,thick] (5.north east) to [bend left=55]  node[above] {$\lambda$}  (N.north west);
\draw [->,thick] (N.south west) to [bend left=55]  node[below] {$\mu$}  (5.south east);


\draw [->,thick] (N.north east) to [bend left=55]  node[above] {$\lambda$}  (N1.north west);
\draw [->,thick] (N1.south west) to [bend left=55]  node[below] {$\mu$}  (N.south east);

    
\end{tikzpicture}

        }
    \end{figure}
    La cadena tiene tasas
    de transición $q_{i,j}$ homogéneas
    \begin{align*}
        q_{i+1,i}=q_{j+1,j}=\mu,\quad \forall i,j\\
        q_{i-1,i}=q_{j-1,j}=\lambda,\quad \forall i,j\\
    \end{align*}
    \begin{figure}
        \resizebox{!}{.15\textwidth}{%
            \usetikzlibrary{calc}
\begin{tikzpicture}


% the rectangular shape with vertical lines
\node[rectangle split, rectangle split parts=6,
draw, rectangle split horizontal,
text height=1cm,text depth=0.5cm,inner ysep=0pt,
fill=HotPink3!20] (wa) {};
\draw[<->] ($(wa.north west)+(0,.3)$)
    -- ($(wa.north east)+(0,.3)$)
    node[midway,above] {$K-1$};
%\fill[white] ([xshift=-\pgflinewidth,yshift=-\pgflinewidth]wa.north west) rectangle ([xshift=-15pt,yshift=\pgflinewidth]wa.south);

% the circle
\node[draw,circle,minimum size=1.5cm,anchor=west,
    fill=HotPink3!20] (se) at
    (wa.east) {$\mu$};

% the arrows and labels
\draw[<-] (wa.west) --
    +(-35pt,0) node (out) {} --
    +(-70pt,0) node[left] (arrive) {$\lambda$};
\draw[->] (se.east) -- ($(se.east)+(.5,0)$);


% draw the exiting
\node[circle,draw=HotPink4,fill=HotPink4,minimum size=2,
    inner sep=0] (outcirc) at (out) {};
\draw[-|]
    (arrive) -- 
    (outcirc) --
    +(0,-25pt) node[below,align=center]{};




\end{tikzpicture}

        }
    \end{figure}
\end{frame}




\subsection{Ecuaciones de equilibrio}
\begin{frame}{\secname: \subsecname}
    \begin{lema}{Ecuaciones de equilibrio M/M/1/K}
        En régimen estacionario, las ecuaciones de
        equilibrio de un sistema M/M/1/K
        resultan en
        \begin{equation}
            \pi_i =
            \begin{cases}
                \frac{1-\rho}{1-\rho^{K+1}}, & i=0\\
                \rho^i\pi_0, & 0<i\leq K
            \end{cases}
            \label{eq:mm1k-equilibrium}
        \end{equation}
    \end{lema}
    \vfill
    \textit{Demostración}: de la cadena se deduce
    por inducción que $\pi_i=\rho^i \pi_0$.
    Sabiendo que $\sum_i^K \pi_i=1$, tenemos
    \begin{equation*}
        1=\pi_0\sum_{i=0}^K\rho^i
        \Longleftrightarrow
        \pi_0=\frac{1}{\sum_{i=0}^K \rho^i}
        =\frac{1}{\frac{1}{1-\rho}(1-\rho^{K+1})}
    \end{equation*}
\end{frame}


\subsection{Probabilidad de bloqueo}
\begin{frame}{\secname: \subsecname}
    El sistema M/M/1/K bloquea la
    entrada de usuarios si
    la cola está llena
    \begin{figure}
        \resizebox{!}{.15\textwidth}{%
            \usetikzlibrary{calc}
\begin{tikzpicture}


% the rectangular shape with vertical lines
\node[rectangle split, rectangle split parts=6,
draw, rectangle split horizontal,
text height=1cm,text depth=0.5cm,inner ysep=0pt,
fill=HotPink3!20] (wa) {};
\draw[<->] ($(wa.north west)+(0,.3)$)
    -- ($(wa.north east)+(0,.3)$)
    node[midway,above] {$K-1$};
%\fill[white] ([xshift=-\pgflinewidth,yshift=-\pgflinewidth]wa.north west) rectangle ([xshift=-15pt,yshift=\pgflinewidth]wa.south);

% the circle
\node[draw,circle,minimum size=1.5cm,anchor=west,
    fill=HotPink3!20] (se) at
    (wa.east) {$\mu$};

% the arrows and labels
\draw[<-] (wa.west) --
    +(-35pt,0) node (out) {} --
    +(-70pt,0) node[left] (arrive) {$\lambda$};
\draw[->] (se.east) -- ($(se.east)+(.5,0)$);


% draw the exiting
\node[circle,draw=HotPink4,fill=HotPink4,minimum size=2,
    inner sep=0] (outcirc) at (out) {};
\draw[-|]
    (arrive) -- 
    (outcirc) --
    +(0,-25pt) node[below,align=center]{};




\end{tikzpicture}

        }
    \end{figure}
    lo cual sucede con probabilidad
    \begin{equation*}
        \pi_K=\rho^K\frac{1-\rho}{1-\rho^{K+1}}
    \end{equation*}
    según las ecuaciones de
    equilibrio~\eqref{eq:mm1k-equilibrium}.
\end{frame}




\subsection{Caudal cursado}
\begin{frame}{\secname: \subsecname}
    El caudal cursado será el de los
    usuarios admitidos $\overline{\lambda}$
    \begin{figure}
        \resizebox{!}{.25\textwidth}{%
            \usetikzlibrary{calc}
\begin{tikzpicture}


% the rectangular shape with vertical lines
\node[rectangle split, rectangle split parts=6,
draw, rectangle split horizontal,
text height=1cm,text depth=0.5cm,inner ysep=0pt,
rectangle split part fill={white,white,white,HotPink3!20}] (wa) {};
\draw[<->] ($(wa.north west)+(0,.3)$)
    -- ($(wa.north east)+(0,.3)$)
    node[midway,above] {$K-1$};
%\fill[white] ([xshift=-\pgflinewidth,yshift=-\pgflinewidth]wa.north west) rectangle ([xshift=-15pt,yshift=\pgflinewidth]wa.south);

% the circle
\node[draw,circle,minimum size=1.5cm,anchor=west,
    fill=HotPink3!20] (se) at
    (wa.east) {$\mu$};

% the arrows and labels
\draw[<-] (wa.west) --
    +(-49pt,0) node (out) {} --
    +(-80pt,0) node[left] (arrive) {$\lambda$};
\draw[->] (se.east) -- ($(se.east)+(.5,0)$);


% draw the exiting
\node[circle,draw=HotPink4,fill=HotPink4,minimum size=3,
    inner sep=0] (outcirc) at (out) {};
\draw[-|]
    (arrive) -- 
    (outcirc) --
    +(0,-25pt) node[midway,anchor=west,align=center]
    {$\pi_K$};
\node[anchor=south west] at (out.east) {$1-\pi_K$};

\draw[->,ultra thick,color=HotPink4] (arrive) -- (wa.west);



\end{tikzpicture}

        }
    \end{figure}
    el cual es precisamente
    $\overline{\lambda}=\lambda(1-\pi_K)$.
\end{frame}


\begin{frame}{\secname: \subsecname}
    \textit{Ejemplo}: sea un servidor con
    tasa $\mu=1$~[usuarios/sec] y
    $\lambda=0.8$~[usuarios/sec], calcule
    el tamaño de cola de modo que el
    caudal cursado $\overline{\lambda}$
    quede por encima de 0.7~[usuarios/sec].

    \vfill

    \begin{multline*}
        k: \overline{\lambda}=
        \lambda(1-\pi_K)\geq0.7
        \Longleftrightarrow
        \lambda\left(1-\rho^K\frac{1-\rho}{1-\rho^{K+1}} \right)
        \geq0.7\\
        \Longleftrightarrow
        \underbrace{\frac{1-\frac{0.7}{\lambda}}{1-\rho}}_{A}
        \geq \frac{\rho^K}{1-\rho^{K+1}}
        \Longleftrightarrow
        \rho^K\leq \frac{A}{1+A\rho}
    \end{multline*}
    sustituyendo $A=\tfrac{1-0.7/0.8}{1-0.8}=\tfrac{5}{8}$
    y $\rho=\tfrac{8}{10}$
    queda
    \begin{equation*}
        \rho^K\leq\frac{5}{12}
        \Longleftrightarrow
        K\geq\log_\rho\frac{5}{12}
        \Longleftrightarrow
        K\geq 3.92
    \end{equation*}
\end{frame}




\begin{frame}{\secname: \subsecname}
    \textit{Ejemplo}: sea un servidor con
    tasa $\mu=1$~[usuarios/sec] y
    $\lambda=0.8$~[usuarios/sec], calcule
    el tamaño de cola de modo que el
    caudal cursado $\overline{\lambda}$
    quede por encima de 0.7~[usuarios/sec].

    \vfill

    \begin{figure}
        \begin{tikzpicture}[
declare function={effective(\x,\K)=\x*(1-\x^\K * (1-\x)/(1-\x^(\K+1)));}
]







\begin{axis}[every axis plot post/.append style={
  mark=none,samples=100},
  axis x line*=bottom,
  axis y line*=left,
  xlabel={$\lambda$},
  ylabel={$\overline{\lambda}$},
  %ymode=log,
  ymin=0,
  ymax=1,
  enlargelimits=upper,
  width=4in,
  height=2in]


  % K=1
  \pgfmathsetmacro{\K}{1}
  \addplot+[ultra thick, smooth,
          color=HotPink1,domain=0:2.9] {
              effective(\x,\K) 
      }
      node [pos=0.5,anchor=north west] {$K=1$};
  \addplot [only marks,samples at={0.8},
          color=HotPink1] {
      effective(\x,\K)};

  % K=3
  \pgfmathsetmacro{\K}{3}
  \addplot+[ultra thick, smooth,
          color=HotPink3,domain=0:2.9] {
              effective(\x,\K) 
      }
      node [pos=0.5,anchor=north west] {$K=3$};
  \addplot [only marks,samples at={0.8},
          color=HotPink3] {
      effective(\x,\K)};

  % K=9
  \pgfmathsetmacro{\K}{9}
  \addplot+[ultra thick, smooth,
          color=HotPink4,domain=0:2.9] {
              effective(\x,\K) 
      }
      node [pos=0.5,anchor=south east] {$K=9$};
  \addplot [only marks,samples at={0.8},
          color=HotPink4] {
      effective(\x,\K)};




  \addplot +[mark=none,dashed,black] coordinates {(.8, 0) (.8, 1)} node[pos=0.1,anchor=west] {$\lambda=0.8$};
  \addplot +[mark=none,dashed,black] coordinates {(.0, .7) (3, .7)} node[pos=.05,anchor=south] {$0.7$};


\end{axis}



\end{tikzpicture}


    \end{figure}
\end{frame}


\subsection{métricas famosas}
\begin{frame}{\secname: \subsecname}
    \begin{lema}[Número medio usuarios M/M/1/K]
        El número medio de usuarios en un
        sistema M/M/1/K es
        \begin{equation}
            \mathbb{E}[N(t)]=
            \frac{\rho}{1-\rho}-
            \frac{(K+1)\rho^{K+1}}{1-\rho^{K+1}}
        \end{equation}
    \end{lema}

    \vfill

    \textit{Demostración}:
    \begin{multline*}
        \mathbb{E}[N(t)]=
        \sum_{i=0}^{K}i \pi_i
        =\pi_0\sum_{i=0}^K i\rho^i
        =\pi_0\rho\frac{d}{dt}
        \sum_{i=0}^K\rho^i
        = \pi_0\rho \frac{d}{dt}
        \frac{1-\rho^{K+1}}{1-\rho}\\
        =\frac{1-\rho}{1-\rho^{K+1}}\rho
        \frac{-(K+1)\rho^K(1-\rho)+(1-\rho^{K+1})}{(1-\rho)^2}=\ldots
    \end{multline*}
\end{frame}



\begin{frame}{\secname: \subsecname}
    Para aplicar Little hay que considerar
    el caudal efectivo $\overline{\lambda}$.
    \begin{figure}
        \resizebox{!}{.15\textwidth}{%
            \usetikzlibrary{calc}
\begin{tikzpicture}


% the rectangular shape with vertical lines
\node[rectangle split, rectangle split parts=6,
draw, rectangle split horizontal,
text height=1cm,text depth=0.5cm,inner ysep=0pt,
rectangle split part fill={white,white,white,HotPink3!20}] (wa) {};
\draw[<->] ($(wa.north west)+(0,.3)$)
    -- ($(wa.north east)+(0,.3)$)
    node[midway,above] {$K-1$};
%\fill[white] ([xshift=-\pgflinewidth,yshift=-\pgflinewidth]wa.north west) rectangle ([xshift=-15pt,yshift=\pgflinewidth]wa.south);

% the circle
\node[draw,circle,minimum size=1.5cm,anchor=west,
    fill=HotPink3!20] (se) at
    (wa.east) {$\mu$};

% the arrows and labels
\draw[<-] (wa.west) --
    +(-49pt,0) node (out) {} --
    +(-80pt,0) node[left] (arrive) {$\lambda$};
\draw[->] (se.east) -- ($(se.east)+(.5,0)$);


% draw the exiting
\node[circle,draw=HotPink4,fill=HotPink4,minimum size=3,
    inner sep=0] (outcirc) at (out) {};
\draw[-|]
    (arrive) -- 
    (outcirc) --
    +(0,-25pt) node[midway,anchor=west,align=center]
    {$\pi_K$};
\node[anchor=south west] at (out.east) {$1-\pi_K$};

\draw[->,ultra thick,color=HotPink4] (arrive) -- (wa.west);



\end{tikzpicture}

        }
    \end{figure}
    En concreto
    \begin{equation*}
        \mathbb{E}[T(t)]=
        \frac{\mathbb{E}[N(t)]}{\lambda(1-\pi_K)}
    \end{equation*}
    Y sabiendo que el tiempo medio de
    espera de los atendidos es
    \begin{equation*}
        \mathbb{E}[W(t)]=
        \mathbb{E}[T(t)]-\mathbb{E}[t_s]
        =\frac{\mathbb{E}[N(t)]}{\lambda(1-\pi_K)}
        - \frac{1}{\mu}
    \end{equation*}
    sacamos el número medio de usuarios
    encolados con Little
    \begin{equation*}
        \mathbb{E}[Q(t)]=
        \lambda(1-\pi_K)
        \mathbb{E}[W(t)]=
        \mathbb{E}[N(t)]-
        \frac{\overline{\lambda}}{\mu}
    \end{equation*}
\end{frame}





\begin{frame}{\secname: \subsecname}
    \textit{Ejemplo}: sea un sistema
    M/M/1/K con $K=10$,
    $\lambda=0.8$~[usuarios/sec],
    $\mu=1$~[usuarios/sec];
    ¿cuál es el tiempo medio que pasa
    un usuario atendido?
    \vfill
    
    Primero sacabos el número medio
    de usuarios
    \begin{multline*}
        \mathbb{E}[N(t)]=
        \frac{\rho}{1-\rho}-
        \frac{(K+1)\rho^{K+1}}{1-\rho^{K+1}}\\
        =\frac{0.8}{1-0.8}-
        \frac{(10+1)0.8^{10+1}}{1-0.8^{10+1}}
        \simeq2.97~\text{[usuarios]}
    \end{multline*}
    y luego usamos Little
    \begin{equation*}
        \mathbb{E}[T(t)]=
        \frac{\mathbb{E}[N(t)]}{\lambda(1-\pi_K)}
        = \frac{2.97}{0.8
        \left( 1- 0.8^{10}\frac{1-0.8}{1-0.8^{10+1}} \right)
        }
        \simeq 3.80~\text{[sec]}
    \end{equation*}
\end{frame}






\section{Sistema M/M/N/N}
\begin{frame}{\secname}
    Si tenemos N servidores sin cola
    también puede haber pérdidas.
    \begin{figure}
        \usetikzlibrary{calc}
\begin{tikzpicture}



\tikzset{onslide/.code args={<#1>#2}{%
    \only<#1>{\pgfkeysalso{#2}} % \pgfkeysalso doesn't change the path
}}



% Sistem entrance
\node[circle,inner sep=0, minimum size=4,
    draw,fill=black]
    (wa) {};


% the circle
\node[draw,circle,minimum size=1.5cm,anchor=west,onslide=<2->{fill=HotPink3!20}] (se) at
    ($(wa.east)+(1,2)$) {$\mu$};
\node[draw,circle,minimum size=1.5cm,anchor=west,onslide=<2->{fill=HotPink3!20}] (se2) at
    ($(wa.east)+(1,.3)$) {$\mu$};
\node (dots) at ($(wa.east)+(1.7,-1)$) {$\vdots$};
\node[draw,circle,minimum size=1.5cm,anchor=west,onslide=<2->{fill=HotPink3!20}] (seN) at
    ($(wa.east)+(1,-2.3)$) {$\mu$};

% server labels
\node[anchor=west] at (se.east) {Servidor 1};
\node[anchor=west] at (se2.east) {Servidor 2};
\node[anchor=west] at (seN.east) {Servidor N};

% the arrows and labels
\draw[<-] (wa.west) -- +(-70pt,0) node[left]
    (arrival) {$\lambda$};

\draw[->] (wa.east) -- (se.west);
\draw[->] (wa.east) -- (se2.west);
\draw[->] (wa.east) -- (seN.west);


\onslide<2>{
    \node[circle,draw,inner sep=0,
        minimum size=4,fill=HotPink4] (out)
        at ($(wa.west)-(35pt,0)$) {};
    \draw[ultra thick,color=HotPink4,-|]
        (arrival) -- (out)
        -- ($(out)-(0,30pt)$)
        node[below,align=center]
        {pérdidas\\(servidores ocupados)};
}




\end{tikzpicture}

    \end{figure}
\end{frame}


\subsection{Cadena de Markov}
\begin{frame}{\secname: \subsecname}
    Un sistema M/M/N/N es un proceso
    estocástico markoviano, y se puede
    modelar con una cadena de Markov
    finita
    \begin{figure}
        \resizebox{!}{.15\textwidth}{%
            \input{figs/cadena-mmnn.tex}
        }
    \end{figure}
    La cadena tiene tasas de transición
    no homogéneas
    \begin{equation*}
        q_{i+1,i}=(i+1)\mu,\ \forall i\leq N
    \end{equation*}
    
    \begin{figure}
        \resizebox{!}{.2\textwidth}{%
            \input{figs/mmnn-full.tex}
        }
    \end{figure}
\end{frame}


\subsection{Ecuaciones de equilibrio}
\begin{frame}{\secname: \subsecname}
    \begin{lema}[Ecuaciones equilibrio
        sistema M/M/N/N]
        En un sistema M/M/N/N en régimen
        estacionario, sus ecuaciones de
        equilibrio resultan en
        \begin{equation}
            \pi_i=\begin{cases}
                \left(
                    \sum_{i=0}^N \frac{A^i}{i!}
                \right)^{-1}, & i=0\\
                \frac{A^i}{i!}\pi_0, &
                \quad 0<i\leq N
            \end{cases}
        \end{equation}
    \end{lema}

    \vfill
    \textit{Demostración}: de la
    cadena deducimos
    $\pi_i=\tfrac{A^i}{i!}\pi_0$ por
    inducción, y usando la ley de probabilidad
    total tenemos
    \begin{equation*}
        1 = \sum_{i=0}^N \pi_i
        \Longleftrightarrow
        1 = \pi_0\sum_{i=0}^N \frac{A^i}{i!}
        \Longleftrightarrow
        \pi_0=\left(
            \sum_{i=0}^N \frac{A^i}{i!}
        \right)^{-1}
    \end{equation*}
\end{frame}





\subsection{Probabilidad de bloqueo}
\begin{frame}{\secname: \subsecname}
    \begin{lema}[Probabilidad de bloqueo]
        En un sistema M/M/N/N se bloquean
        llegadas con probabilidad
        \begin{equation}
            E_B(N,A)=\pi_N=
            \frac{\frac{A^N}{N!}}{\sum_{i=0}^N\frac{A^i}{i!}}
        \end{equation}
        y se conoce $E_B(N,A)$
        como la 1ª distribución
        de Earlang, o Earlang-B.
    \end{lema}

    \vfill

    \textit{Ejemplo}: ¿cuál es la probabilidad
    de bloqueo en un M/M/2/2 con una
    intensidad de $A=1.2$ Earlangs?
    \begin{figure}
        \resizebox{!}{.1\textwidth}{%
            \input{figs/mm22-full.tex}
        }
    \end{figure}
    \begin{equation*}
        E_B(N,A)=\pi_N=
        \frac{\frac{A^N}{N!}}{\sum_{i=0}^N\frac{A^i}{i!}}
        =\frac{1.2^2}{2!}\left(
            \sum_{i=0}^2 \frac{1.2^i}{i!}
        \right)^{-1}
        = 0.25
    \end{equation*}
\end{frame}




\begin{frame}{\secname: \subsecname}
    \textit{Ejemplo}: sea un sistema
    M/M/N/N con carga $A=2.6$ Earlangs,
    ¿cuál es el mínimo
    número de servidores para que
    la probabilidad de bloqueo sea
    menor al 20\%?

    \begin{equation*}
        N: E_B(N,A)=\pi_N
        =\frac{\frac{A^N}{N!}}{\sum_{i=0}^N\frac{A^i}{i!}}
        =\frac{\frac{2.6^N}{N!}}{\sum_{i=0}^N\frac{2.6^i}{i!}}\leq 0.2
    \end{equation*}

    \begin{figure}
        \begin{tikzpicture}[
declare function={free(\k,\I)=\I^\k/(\k!);}
]



%%%% DEFINE SUMATION USING
%https://tex.stackexchange.com/a/526600/62559
%\newcounter{isum}
\pgfplotsset{summand/.initial=max}
\pgfmathdeclarefunction{sum}{2}{%
\begingroup%
\pgfkeys{/pgf/fpu,/pgf/fpu/output format=fixed}%
\edef\myfun{\pgfkeysvalueof{/pgfplots/summand}}%
\pgfmathsetmacro{\mysum}{0}%
\pgfmathsetmacro{\myx}{#2}%
\pgfmathtruncatemacro{\imax}{#1}%
\setcounter{isum}{0}%
\loop
\pgfmathsetmacro{\mysum}{\mysum+\myfun(\value{isum},#2)}%
\ifnum\value{isum}<\imax\relax
\stepcounter{isum}\repeat
\pgfmathparse{\mysum}%
\pgfmathsmuggle\pgfmathresult\endgroup%
}%







\begin{axis}[every axis plot post/.append style={
  mark=none,samples=100},
  axis x line*=bottom,
  axis y line*=left,
  xlabel={$A$},
  ylabel={$E_B(N,A)$},
  %ymode=log,
  ymin=0,
  ymax=1,
  enlargelimits=upper,
  width=4in,
  height=2in]

  % N=3
  \pgfmathsetmacro{\N}{3}
  \addplot+[summand=free,ultra thick,
          smooth, color=HotPink1,
      domain=0:15] {\x^\N / \N! / (sum(\N,\x))} node [anchor=south west] {$N=3$};%
   \addplot [only marks,samples at={2.6},
       color=HotPink1] {
   \x^\N / \N! / (sum(\N,\x))};


  % N=4
  \pgfmathsetmacro{\N}{4}
  \addplot+[summand=free,ultra thick,
          smooth, color=HotPink2,
      domain=0:15] {\x^\N / \N! / (sum(\N,\x))} node [anchor=west] {$N=4$};%
   \addplot [only marks,samples at={2.6},
       color=HotPink2] {
   \x^\N / \N! / (sum(\N,\x))};

  % N=5
  \pgfmathsetmacro{\N}{5}
  \addplot+[summand=free,ultra thick,
          smooth, color=HotPink4,
      domain=0:15] {\x^\N / \N! / (sum(\N,\x))} node [anchor=north west] {$N=5$};%
   \addplot [only marks,samples at={2.6},
       color=HotPink4] {
   \x^\N / \N! / (sum(\N,\x))};



  \addplot+[dashed,draw=black] coordinates {(0,0.2) (16,0.2)} node [pos=0,anchor=south west,color=black] {0.2};
  \addplot+[dashed,draw=black] coordinates {(2.6,0) (2.6,1)} node[pos=.9,anchor=west] {$A=2.6$};


\end{axis}



\end{tikzpicture}


    \end{figure}
\end{frame}





\subsection{Caudal cursado}
\begin{frame}{\secname: \subsecname}
    El caudal cursado $\overline{\lambda}$
    es el flujo que no se rechaza:
    \begin{figure}
        \resizebox{!}{.3\textwidth}{%
            \usetikzlibrary{calc}
\begin{tikzpicture}



\tikzset{onslide/.code args={<#1>#2}{%
    \only<#1>{\pgfkeysalso{#2}} % \pgfkeysalso doesn't change the path
}}



% Sistem entrance
\node[circle,inner sep=0, minimum size=4,
    draw,fill=black]
    (wa) {};


% the circle
\node[draw,circle,minimum size=1.5cm,anchor=west,fill=HotPink3!20] (se) at
    ($(wa.east)+(1,2)$) {$\mu$};
\node[draw,circle,minimum size=1.5cm,anchor=west,fill=HotPink3!20] (se2) at
    ($(wa.east)+(1,.3)$) {$\mu$};
\node (dots) at ($(wa.east)+(1.7,-1)$) {$\vdots$};
\node[draw,circle,minimum size=1.5cm,anchor=west,fill=HotPink3!20] (seN) at
    ($(wa.east)+(1,-2.3)$) {$\mu$};

% server labels
\node[anchor=west] at (se.east) {Servidor 1};
\node[anchor=west] at (se2.east) {Servidor 2};
\node[anchor=west] at (seN.east) {Servidor N};

% the arrows and labels
\draw[<-] (wa.west) -- +(-90pt,0) node[left]
    (arrival) {$\lambda$};

\draw[->] (wa.east) -- (se.west);
\draw[->] (wa.east) -- (se2.west);
\draw[->] (wa.east) -- (seN.west);


\node[circle,draw,inner sep=0,
    minimum size=4] (out)
    at ($(wa.west)-(45pt,0)$) {};
\draw[-|]
    (arrival) -- (out)
    -- ($(out)-(0,30pt)$)
    node[pos=.5,anchor=west] {$\pi_N$};
\node at ($(out)+(22.5pt,0.3)$) {$1-\pi_N$};

\draw[line width=1mm, color=HotPink4,->]
    (arrival) -- (wa.west);




\end{tikzpicture}

        }
    \end{figure}
    es decir
    \begin{equation*}
        \overline{\lambda}=
        (1-\pi_B)\lambda
    \end{equation*}
\end{frame}



\subsection{Métricas famosas}
\begin{frame}{\secname: \subsecname}
    \begin{lema}[Tiempo medio en sistema
        M/M/N/N]
        El tiempo medio en un sistema
        M/M/N/N es
        $\mathbb{E}[T(t)]=\frac{1}{\mu}$.
    \end{lema}
    
    \begin{figure}
        \resizebox{!}{.2\textwidth}{%
            \usetikzlibrary{calc}
\begin{tikzpicture}



\tikzset{onslide/.code args={<#1>#2}{%
    \only<#1>{\pgfkeysalso{#2}} % \pgfkeysalso doesn't change the path
}}



% Sistem entrance
\node[circle,inner sep=0, minimum size=4,
    draw,fill=black]
    (wa) {};


% the circle
\node[draw,circle,minimum size=1.5cm,anchor=west,fill=HotPink3!20] (se) at
    ($(wa.east)+(1,2)$) {$\mu$};
\node[draw,circle,minimum size=1.5cm,anchor=west,fill=HotPink3!20] (se2) at
    ($(wa.east)+(1,.3)$) {$\mu$};
\node (dots) at ($(wa.east)+(1.7,-1)$) {$\vdots$};
\node[draw,circle,minimum size=1.5cm,anchor=west,fill=HotPink3!20] (seN) at
    ($(wa.east)+(1,-2.3)$) {$\mu$};

% server labels
\node[anchor=west] at (se.east) {Servidor 1};
\node[anchor=west] at (se2.east) {Servidor 2};
\node[anchor=west] at (seN.east) {Servidor N};

% the arrows and labels
\draw[<-] (wa.west) -- +(-90pt,0) node[left]
    (arrival) {$\lambda$};

\draw[->] (wa.east) -- (se.west);
\draw[->] (wa.east) -- (se2.west);
\draw[->] (wa.east) -- (seN.west);


\node[circle,draw,inner sep=0,
    minimum size=4] (out)
    at ($(wa.west)-(45pt,0)$) {};
\draw[-|]
    (arrival) -- (out)
    -- ($(out)-(0,30pt)$)
    node[pos=.5,anchor=west] {$\pi_N$};
\node at ($(out)+(22.5pt,0.3)$) {$1-\pi_N$};

\draw[line width=1mm, color=HotPink4,->]
    (arrival) -- (wa.west);




\end{tikzpicture}

        }
    \end{figure}

    \textit{Demostración}: como no
    hay espera en cola $W(t)=0,\ \forall t$;
    se tiene que
    \begin{equation*}
        \mathbb{E}[T(t)]=\mathbb{E}[t_s]
        =\frac{1}{\mu}
    \end{equation*}
    y por Little sabemos
    que $\mathbb{E}[N(t)]=\lambda(1-\pi_N)\mathbb{E}[T(t)]=A(1-\pi_N)$.
\end{frame}



\begin{frame}[allowframebreaks]
        \nocite{amable}
        \frametitle{Referencias}
        \bibliographystyle{amsalpha}
        \bibliography{refs.bib}
\end{frame}





\end{document}
