\documentclass[xcolor={x11names}]{beamer}
\usetheme{Madrid}

\usepackage{amssymb}
\usepackage{ulem}
\usepackage[utf8]{inputenc}
\usepackage{mathtools}
\usepackage{multicol}
%\usepackage[x11names]{xcolor}
\usefonttheme{professionalfonts}


% Subfigures
\usepackage{caption}
\usepackage{subcaption}

% Check/cross mark
\usepackage{pifont}% http://ctan.org/pkg/pifont
\newcommand{\cmark}{\ding{51}}%
\newcommand{\xmark}{\ding{55}}%


% License
\usepackage[
    type={CC},
    modifier={by-nc-sa},
    version={4.0},
    imagewidth=4pt
]{doclicense}




% Change base colour beamer@blendedblue (originally RGB: 0.2,0.2,0.7)
\colorlet{beamer@blendedblue}{DarkSeaGreen4}







%% MATH commands
\DeclareMathOperator{\Var}{Var}


%% THEOREMS
%\newtheorem{theorem}{Theorem}
\newtheorem{thm}{Teorema}[section] % the main one
% Definición
%\theoremstyle{definition}
\newtheorem{definicion}{Definición}[section]
\newtheorem{lema}{Lema}[section]



%% PFGplots %%
\usepackage{pgfplots}

%% Exponential distribution
\pgfmathdeclarefunction{exponential}{1}{%
  \pgfmathparse{(#1)*exp(-#1*x)}%
}
\pgfmathdeclarefunction{exponentialcdf}{1}{%
  \pgfmathparse{1-exp(-#1*x)}%
}

%% Poisson distribution
\pgfmathdeclarefunction{poiss}{1}{%
  \pgfmathparse{(#1^x)*exp(-#1)/(x!)}%
}

%% Normal distribution (#1=mu, #2=sigma)
% John D. Cook approx. https://tex.stackexchange.com/a/124629
\pgfmathdeclarefunction{normalcdf}{2}{%
  \pgfmathparse{1/(1 + exp(-0.07056*((x-#1)/#2)^3 - 1.5976*(x-#1)/#2))}%
}






\newcommand{\red}[1]{{\color{red}#1}}
\newcommand{\blue}[1]{{\color{blue}#1}}

%%%%%%%%%%
%% TIKZ %%
%%%%%%%%%%
\usepackage{tikz}
\usepackage{animate}
\usetikzlibrary{positioning}
\usetikzlibrary{shapes,arrows, positioning, calc}
\usetikzlibrary{overlay-beamer-styles}
\usetikzlibrary{chains,shapes.multipart}
\usetikzlibrary{scopes}
\usetikzlibrary{automata}
\usetikzlibrary{positioning}  %                 ...positioning nodes
\usetikzlibrary{arrows}       %                 ...customizing arrows
\usetikzlibrary{intersections}


%%%%%%%%%
%% PGF %%
%%%%%%%%%
\usepgfplotslibrary{fillbetween}


%%% Insert section name before the section %%%
\AtBeginSection[]{
  \begin{frame}
  \vfill
  \centering
  \begin{beamercolorbox}[sep=8pt,center,shadow=true,rounded=true]{title}
    \usebeamerfont{title}\insertsectionhead\par%
  \end{beamercolorbox}
  \vfill
  \end{frame}
}



\title[Tema 7]{Tema 7: Teletráfico en redes de telecomunicaciones}
%% \subtitle{Redes y Servicios de Telecomunicaciones (RSTC)\\
%% Grado en Ingeniería de Tecnologías y Servicios de Telecomunicación}
%\author{M. Saiful Bari\inst{1} \and Mr X\inst{2}}

\titlegraphic{%
\includegraphics[height=2.5cm]{../tema-5/figs/RSTC-grande.png}\\%
\doclicenseIcon {\tiny \hspace{1em}\doclicenseText}\\%
\href{https://github.com/MartinPJorge/RSTC-queueing-slides}{\includegraphics[height=1cm]{../tema-5/figs/github-logo.png}}%
}


\author{\textcolor{white}{RSTC curso 2022-2023}}
%\author{Jorge Martín Pérez\inst{1}}
%\institute{
%    \inst{1}
%    Departamento de Ingeniería Telemática, Universidad Politécnica de Madrid
%}

\date{\today}







%%%%%%%%%%%%%%%%%%%%
%%% SLIDES START %%%
%%%%%%%%%%%%%%%%%%%%
\begin{document}


%%% TITLE %%%
\frame{\titlepage}


\begin{frame}{Contenido}
    \tableofcontents
\end{frame}


\section{Sistema M/M/N}
\begin{frame}{\secname}
    Un sistema de cola única
    con $t_l\sim Exp(\lambda)$
    y N servidores en paralelo
    con $t_s\sim Exp(\mu)$.
    \begin{figure}
        \input{figs/mmn.tex}
    \end{figure}
\end{frame}


\subsection{Cadena de Markov}
\begin{frame}{\secname: \subsecname}
    Un sistema M/M/N es un proceso
    estocástico Markoviano\footnote{
    El tiempo de estancia es
    exponencial no homogéneo.}.
    \begin{figure}
        \resizebox{!}{.2\textwidth}{%
            \begin{tikzpicture}
   
% Estados
\node[state,minimum size=4em] (1) {$0$};
\node[state,minimum size=4em] (2) [right=of 1] {$1$};
\node[state,minimum size=4em,fill=HotPink3!20] (3) [right=of 2] {$2$};
\node[state,minimum size=4em] (4) [right=of 3] {$3$};
\node[minimum size=3em] (5) [right=of 4] {$\cdots$};
\node[state,minimum size=4em] (N) [right=of 5] {$N$};
\node[state,minimum size=4em] (N1) [right=of N] {$N+1$};
\node[minimum size=3em] (N2) [right=of N1] {$\cdots$};


% Transiciones
\draw [->,thick] (1.north east) to [bend left=55]  node[above] {$\lambda$}  (2.north west);
\draw [->,thick] (2.south west) to [bend left=55]  node[below] {$\mu$}  (1.south east);

\draw [->,thick] (2.north east) to [bend left=55]  node[above] {$\lambda$}  (3.north west);
\draw [->,thick] (3.south west) to [bend left=55]  node[below] {$2\mu$}  (2.south east);

\draw [->,thick] (3.north east) to [bend left=55]  node[above] {$\lambda$}  (4.north west);
\draw [->,thick] (4.south west) to [bend left=55]  node[below] {$3\mu$}  (3.south east);

\draw [->,thick] (4.north east) to [bend left=55]  node[above] {$\lambda$}  (5.north west);
\draw [->,thick] (5.south west) to [bend left=55]  node[below] {$4\mu$}  (4.south east);


\draw [->,thick] (5.north east) to [bend left=55]  node[above] {$\lambda$}  (N.north west);
\draw [->,thick] (N.south west) to [bend left=55]  node[below] {$N\mu$}  (5.south east);


\draw [->,thick] (N.north east) to [bend left=55]  node[above] {$\lambda$}  (N1.north west);
\draw [->,thick] (N1.south west) to [bend left=55]  node[below] {$N\mu$}  (N.south east);


\draw [->,thick] (N1.north east) to [bend left=55]  node[above] {$\lambda$}  (N2.north west);
\draw [->,thick] (N2.south west) to [bend left=55]  node[below] {$N\mu$}  (N1.south east);

\draw[dashed] (N.north) -- ($(0,1.5)+(N.north)$);
\draw[dashed] (N.south) -- ($(0,-1.5)+(N.south)$);
    
\end{tikzpicture}

        }
    \end{figure}
    Sus tasas de transición no son
    homogéneas $q_{i,j}=i\mu,\ i\leq N$.
    \begin{figure}
        \resizebox{!}{.2\textwidth}{%
            \input{figs/mmn-highlight.tex}
        }
    \end{figure}
\end{frame}



\subsection{Ecuaciones de equilibrio}
\begin{frame}{\secname: \subsecname}
    \begin{figure}
        \resizebox{!}{.2\textwidth}{%
            \begin{tikzpicture}
   
% Estados
\node[state,minimum size=4em] (1) {$0$};
\node[state,minimum size=4em] (2) [right=of 1] {$1$};
\node[state,minimum size=4em,fill=HotPink3!20] (3) [right=of 2] {$2$};
\node[state,minimum size=4em] (4) [right=of 3] {$3$};
\node[minimum size=3em] (5) [right=of 4] {$\cdots$};
\node[state,minimum size=4em] (N) [right=of 5] {$N$};
\node[state,minimum size=4em] (N1) [right=of N] {$N+1$};
\node[minimum size=3em] (N2) [right=of N1] {$\cdots$};


% Transiciones
\draw [->,thick] (1.north east) to [bend left=55]  node[above] {$\lambda$}  (2.north west);
\draw [->,thick] (2.south west) to [bend left=55]  node[below] {$\mu$}  (1.south east);

\draw [->,thick] (2.north east) to [bend left=55]  node[above] {$\lambda$}  (3.north west);
\draw [->,thick] (3.south west) to [bend left=55]  node[below] {$2\mu$}  (2.south east);

\draw [->,thick] (3.north east) to [bend left=55]  node[above] {$\lambda$}  (4.north west);
\draw [->,thick] (4.south west) to [bend left=55]  node[below] {$3\mu$}  (3.south east);

\draw [->,thick] (4.north east) to [bend left=55]  node[above] {$\lambda$}  (5.north west);
\draw [->,thick] (5.south west) to [bend left=55]  node[below] {$4\mu$}  (4.south east);


\draw [->,thick] (5.north east) to [bend left=55]  node[above] {$\lambda$}  (N.north west);
\draw [->,thick] (N.south west) to [bend left=55]  node[below] {$N\mu$}  (5.south east);


\draw [->,thick] (N.north east) to [bend left=55]  node[above] {$\lambda$}  (N1.north west);
\draw [->,thick] (N1.south west) to [bend left=55]  node[below] {$N\mu$}  (N.south east);


\draw [->,thick] (N1.north east) to [bend left=55]  node[above] {$\lambda$}  (N2.north west);
\draw [->,thick] (N2.south west) to [bend left=55]  node[below] {$N\mu$}  (N1.south east);

\draw[dashed] (N.north) -- ($(0,1.5)+(N.north)$);
\draw[dashed] (N.south) -- ($(0,-1.5)+(N.south)$);
    
\end{tikzpicture}

        }
    \end{figure}
    Con $i<N$ tenemos
    \begin{equation*}
        \lambda\pi_{i-1}
        +(i+1)\mu\pi_{i+1}
        = \pi_i(\lambda+i\mu)
    \end{equation*}
    pero con $i\geq N$ tenemos
    \begin{equation*}
        \lambda\pi_{i-1}
        +N\mu\pi_{i+1}
        = \pi_i(\lambda+N\mu)
    \end{equation*}
\end{frame}




\begin{frame}{\secname: \subsecname}
    \begin{lema}[Ecuaciones equilibrio M/M/N]
        En régimen estacionario de un sistema
        M/M/N, la probabilidad de estar en el
        estado $i$ es:
        \begin{equation}
            \pi_i=\begin{cases}
                \frac{A^i}{i!}\pi_0, \quad&
                i\leq N\\
                \rho^i \frac{N^N}{N!}\pi_0,
                \quad& i\geq N
            \end{cases}
        \end{equation}
        con $A=\tfrac{\lambda}{\mu}$,
        $\rho=\tfrac{\lambda}{N\mu}$; y
        \begin{equation}
            \pi_0=\left(
                \left(\sum_{i=0}^{N-1}
                \frac{A^i}{i!}
                \right)
                + \frac{A^N}{N!}\frac{1}{1-\rho}
            \right)^{-1}
        \end{equation}
    \end{lema}
\end{frame}



\begin{frame}{\secname: \subsecname}
    \textit{Demostración}:
    \begin{itemize}
        \item para $i\leq N$ se tiene:
        \begin{figure}
            \resizebox{!}{.1\textwidth}{%
                \begin{tikzpicture}
   
% Estados
\node[state,minimum size=4em] (1) {$0$};
\node[state,minimum size=4em] (2) [right=of 1] {$1$};
\node[state,minimum size=4em,fill=HotPink3!20] (3) [right=of 2] {$2$};
\node[state,minimum size=4em] (4) [right=of 3] {$3$};


% Transiciones
\draw [->,thick] (1.north east) to [bend left=55]  node[above] {$\lambda$}  (2.north west);
\draw [->,thick] (2.south west) to [bend left=55]  node[below] {$\mu$}  (1.south east);

\draw [->,thick] (2.north east) to [bend left=55]  node[above] {$\lambda$}  (3.north west);
\draw [->,thick] (3.south west) to [bend left=55]  node[below] {$2\mu$}  (2.south east);

\draw [->,thick] (3.north east) to [bend left=55]  node[above] {$\lambda$}  (4.north west);
\draw [->,thick] (4.south west) to [bend left=55]  node[below] {$3\mu$}  (3.south east);

\draw [->,thick] (4.north east) to [bend left=55]  node[above] {$\lambda$}  (5.north west);
\draw [->,thick] (5.south west) to [bend left=55]  node[below] {$4\mu$}  (4.south east);

    
\end{tikzpicture}

            }
        \end{figure}
        sabiendo que $\pi_0\lambda=\pi_1\mu$
        se tiene $\pi_1=A\pi_0$.
        Por tanto tenemos que
        \begin{align*}
            \lambda\pi_0+2\mu\pi_2=
            \pi_1(\lambda+\mu)
            \Longleftrightarrow&
            \pi_2=\frac{A^2}{2}\pi_0\\
            %
            \lambda\pi_1+3\mu\pi_3=
            \pi_2(\lambda+2\mu)
            \Longleftrightarrow&
            \pi_3=\frac{A^3}{3!}\pi_0\\
            \ldots\\
            \lambda\pi_{i-1}+(i+1)\mu\pi_{i+1}=
            \pi_i(\lambda+i\mu)
            \Longleftrightarrow&
            \pi_i=\frac{A^i}{i!}\pi_0\\
        \end{align*}
        y deducimos $\pi_i=\tfrac{A}{i}\pi_{i-1}$.
    \end{itemize}
\end{frame}




\begin{frame}{\secname: \subsecname}
    \textit{Demostración}:
    \begin{itemize}
        \item para $i\geq N$ se tiene:
        \begin{figure}
            \resizebox{!}{.2\textwidth}{%
                \begin{tikzpicture}
   
\node[state,minimum size=4em] (5) [right=of 4] {$N-2$};
\node[state,minimum size=4em] (N) [right=of 5] {$N-1$};
\node[state,minimum size=4em] (N1) [right=of N] {$N$};
\node[state,minimum size=3em] (N2) [right=of N1] {$N+1$};




\draw [->,thick] (5.north east) to [bend left=55]  node[above] {$\lambda$}  (N.north west);
\draw [->,thick] (N.south west) to [bend left=55]  node[below] {$(N-1)\mu$}  (5.south east);


\draw [->,thick] (N.north east) to [bend left=55]  node[above] {$\lambda$}  (N1.north west);
\draw [->,thick] (N1.south west) to [bend left=55]  node[below] {$N\mu$}  (N.south east);


\draw [->,thick] (N1.north east) to [bend left=55]  node[above] {$\lambda$}  (N2.north west);
\draw [->,thick] (N2.south west) to [bend left=55]  node[below] {$N\mu$}  (N1.south east);

\draw[dashed] (N1.north) -- ($(0,1.5)+(N1.north)$);
\draw[dashed] (N1.south) -- ($(0,-1.5)+(N1.south)$);
    
\end{tikzpicture}

            }
        \end{figure}
        \begin{align*}
            \pi_{N-1}(\lambda+(N-1)\mu)
            =&\pi_N N\mu + \pi_{N-2}\lambda\\
            \pi_{N}
            =& \frac{1}{N\mu}
            \left(
                \pi_{N-1}\lambda
                + \pi_{N-1}\mu(N-1)
                - \pi_{N-2}\lambda
            \right)\\
        \end{align*}
        sustituyendo $\pi_{N-1}=\tfrac{A}{N-1}
        \pi_{N-2}$, sacamos la recursión
        \begin{equation*}
            \pi_i=\frac{\lambda}{N\mu}\pi_{i-1},
            \quad i\geq N
        \end{equation*}
    \end{itemize}
\end{frame}

\begin{frame}{\secname: \subsecname}
    \textit{Demostración}:
    \begin{itemize}
        \item para $i\geq N$ se tiene:
        \begin{equation*}
            \pi_i=\frac{\lambda}{N\mu}\pi_{i-1},
            \quad i\geq N
        \end{equation*}
        y si usamos $\pi_i=\tfrac{A^i}{i!}\pi_0$
        con $i\leq N$ obtenemos:
        \begin{align*}
            \pi_{i}=&\frac{A}{N}\pi_{i-1}
            =\frac{A^2}{N^2}\pi_{i-2}=
            \cdots
            = \left(
                \frac{A}{N}
            \right)^{i-N}
            \pi_N
            = \left(
                \frac{A}{N}
            \right)^{i-N}
            \frac{A^N}{N!}\pi_0\\
            =& \rho^i \frac{N^N}{N!}\pi_0
        \end{align*}
    \end{itemize}
\end{frame}




\begin{frame}{\secname: \subsecname}
    \textit{Demostración}:
    \begin{itemize}
        \item para $i=0$ se tiene:
            \begin{align*}
                1=&\sum_{i=0}^{N-1}\pi_i
                 + \sum_{i=N}^\infty \pi_i\\
                 =& \sum_{i=0}^{N-1}
                 \frac{A^i}{i!}\pi_0
                 + \sum_{i=N}^\infty
                 \rho^i \frac{N^N}{N!}\pi_0\\
                 =& \pi_0\sum_{i=0}^{N-1}
                 \frac{A^i}{i!}
                 + \pi_0 \frac{N^N}{N!}\sum_{i=N}^\infty
                 \rho^i
            \end{align*}
            sabiendo que
            $\sum_{i=N}^\infty
            \rho^i=\tfrac{\rho^N}{1-\rho}$,
            sacamos
            \begin{equation*}
            \pi_0=\left(
                \left(\sum_{i=0}^{N-1}
                \frac{A^i}{i!}
                \right)
                + \frac{A^N}{N!}\frac{1}{1-\rho}
            \right)^{-1}
            \end{equation*}
    \end{itemize}
\end{frame}



\subsection{Segunda distribución de Earlang}
\begin{frame}{\secname: \subsecname}
    \begin{definicion}[Segunda distribución de
        Earlang]
        Popularmente, la probabilidad de que
        un M/M/N esté vacío
                \begin{equation*}
                \pi_0=\left(
                    \left(\sum_{i=0}^{N-1}
                    \frac{A^i}{i!}
                    \right)
                    + \frac{A^N}{N!}\frac{1}{1-\rho}
                \right)^{-1}
                \end{equation*}
        se conoce como la segunda
        distribución de Earlang, y a la razón
        $A$ se le llaman ``Earlangs''.
    \end{definicion}

    \vfill

    \textit{Ejemplo}: si tenemos tasas
    $\lambda=10$ y $\mu=2$; los Earlangs
    $A=\tfrac{\lambda}{\mu}=5$ me dicen que
    necesito $>5$ servidores; y con $N=7$
    tenemos
    \begin{equation*}
    \pi_0=\left(
        \left(\sum_{i=0}^{6}
        \frac{5^i}{i!}
        \right)
        + \frac{5^7}{7!}\frac{1}{1-\tfrac{5}{7}}
    \right)^{-1}
    \simeq 0.006
    \end{equation*}
\end{frame}




\subsection{Distribución de Earlang-C}
\begin{frame}{\secname: \subsecname}
    \begin{lema}[Earlang-C]
        La probabilidad de esperar en un
        sistema M/M/N viene dada por
        la distribución Earlang-C:
        \begin{equation}
            E_C(N,A)=\mathbb{P}(N(t)\geq N)
            = \frac{A^N}{N!}\frac{1}{1-\rho}\pi_0
        \end{equation}
    \end{lema}
    \begin{figure}
        \resizebox{!}{.2\textwidth}{%
            \usetikzlibrary{calc}
\begin{tikzpicture}


% the rectangular shape with vertical lines
\node[rectangle split, rectangle split parts=6,
draw, rectangle split horizontal,text height=1cm,text depth=0.5cm,inner ysep=0pt] (wa) {};
\fill[white] ([xshift=-\pgflinewidth,yshift=-\pgflinewidth]wa.north west) rectangle ([xshift=-15pt,yshift=\pgflinewidth]wa.south);

% the circle
\node[draw,circle,minimum size=1.5cm,anchor=west,fill=HotPink3!20] (se) at
    ($(wa.east)+(1,2)$) {$\mu$};
\node[draw,circle,minimum size=1.5cm,anchor=west,fill=HotPink3!20] (se2) at
    ($(wa.east)+(1,.3)$) {$\mu$};
\node (dots) at ($(wa.east)+(1.7,-1)$) {$\vdots$};
\node[draw,circle,minimum size=1.5cm,anchor=west,fill=HotPink3!20] (seN) at
    ($(wa.east)+(1,-2.3)$) {$\mu$};

% server labels
\node[anchor=west] at (se.east) {Servidor 1};
\node[anchor=west] at (se2.east) {Servidor 2};
\node[anchor=west] at (seN.east) {Servidor N};

% the arrows and labels
\draw[<-] (wa.west) -- +(-20pt,0) node[left] {$\lambda$};

\draw[->] (wa.east) -- (se.west);
\draw[->] (wa.east) -- (se2.west);
\draw[->] (wa.east) -- (seN.west);







\end{tikzpicture}

        }
    \end{figure}
    \vfill
    \textit{Demostración}:
        \begin{equation}
            \mathbb{P}(N(t)\geq N)=
            \sum_{i=N}^{\infty}\pi_i
            = \sum_{i=N}^{\infty}
            \rho^i \frac{N^N}{N!}\pi_0
            = \frac{N^N}{N!}\frac{\rho^N}{1-\rho}\pi_0
        \end{equation}
\end{frame}




\subsection{Métricas famosas}
\begin{frame}{\secname: \subsecname}
    \begin{lema}[Número medio en cola M/M/N]
        En un sistema M/M/N el número medio
        de usuarios encolados es
        \begin{equation}
            \mathbb{E}[Q(t)]=
            \frac{\rho}{1-\rho}
            E_C(N,A)
        \end{equation}
    \end{lema}
    \vfill
    \textit{Demostración}:
    \begin{multline*}
        \mathbb{E}[Q(t)]=\sum_{i=N}^\infty
        (i-N)\pi_N=
        \sum_{i=N}^\infty \rho^i \frac{N^N}{N!}
        \pi_0(i-N)=
        \frac{N^N}{N!}\pi_0
        \sum_{i=N}^\infty(i-N)\rho^i\\
        = 
        \frac{N^N}{N!}\pi_0
        \left[
            \sum_{i=N}^\infty
            i\rho^i - N\sum_{i=N}^\infty
            \rho^i
        \right]
        = 
        \frac{N^N}{N!}\pi_0
        \left[
            \frac{\rho^{N+1}}{(1-\rho)^2}
        \right]\\
        = \frac{\rho}{1-\rho}E_C(N,A)
    \end{multline*}
\end{frame}



\begin{frame}{\secname: \subsecname}
    \textit{Ejemplo}: sea un sistema con una
    carga de $A=\tfrac{\lambda}{\mu}=3$
    Earlangs, ¿cuántos servidores $N$ hay
    que poner para que, en media,
    haya menos de 10
    usuarios encolados?
    \begin{multline*}
        N:\mathbb{E}[Q(t)]=
        \frac{\rho}{1-\rho}E_C(N,A)
        = \frac{\tfrac{3}{N}}{1-\tfrac{3}{N}}
        \frac{N^N}{N!}\frac{\left(\frac{3}{N}\right)^N}{1-\tfrac{3}{N}}\pi_0\\
        =\frac{N^N}{N!}\frac{\left(\frac{3}{N}\right)^{N+1}}{\left(1-\tfrac{3}{N}\right)^2}
                \left(
                    \left(\sum_{i=0}^{N-1}
                    \frac{3^i}{i!}
                    \right)
                    + \frac{3^N}{N!}\frac{1}{1-\tfrac{3}{N}}
                \right)^{-1}<10
    \end{multline*}
\end{frame}





\begin{frame}{\secname: \subsecname}
    \textit{Ejemplo (cont.)}: vemos que con
    una intensidad de $A=3$ Earlangs necesitamos
    $N>3$ para que $\mathbb{E}[Q(t)]\leq 10$.
    \vfill
    \begin{figure}
        \begin{tikzpicture}[
declare function={free(\k,\I)=\I^\k/(\k!);}
]




%%%% DEFINE SUMATION USING
%https://tex.stackexchange.com/a/526600/62559
\newcounter{isum}
\pgfplotsset{summand/.initial=max}
\pgfmathdeclarefunction{sum}{2}{%
\begingroup%
\pgfkeys{/pgf/fpu,/pgf/fpu/output format=fixed}%
\edef\myfun{\pgfkeysvalueof{/pgfplots/summand}}%
\pgfmathsetmacro{\mysum}{0}%
\pgfmathsetmacro{\myx}{#2}%
\pgfmathtruncatemacro{\imax}{#1}%
\setcounter{isum}{0}%
\loop
\pgfmathsetmacro{\mysum}{\mysum+\myfun(\value{isum},#2)}%
\ifnum\value{isum}<\imax\relax
\stepcounter{isum}\repeat
\pgfmathparse{\mysum}%
\pgfmathsmuggle\pgfmathresult\endgroup%
}%







\begin{axis}[every axis plot post/.append style={
  mark=none,samples=100},
  axis x line*=bottom,
  axis y line*=left,
  xlabel={$A$},
  ylabel={$\mathbb{E}[Q(t)]$},
  %ymode=log,
  ymin=0,
  ymax=10,
  enlargelimits=upper,
  width=4in,
  height=2in]

  % N=3
  \pgfmathsetmacro{\N}{3}
  \addplot+[summand=free,ultra thick, smooth,
          color=HotPink1,domain=0:2.9] {
      \N^\N/\N! * (\x/\N)^(\N+1)/(1-\x/\N)^2 *%
      1/(sum(\N-1,\x) + \x^\N/\N! * 1/(1-\x/\N))}
      node [pos=0.2,anchor=south east] {$N=3$};


  % N=4
  \pgfmathsetmacro{\N}{4}
  \addplot+[summand=free,ultra thick, smooth,
          color=HotPink2,domain=0:3.9] {
      \N^\N/\N! * (\x/\N)^(\N+1)/(1-\x/\N)^2 *%
      1/(sum(\N-1,\x) + \x^\N/\N! * 1/(1-\x/\N))}
      node [pos=0.2,anchor=north west] {$N=4$};



  % N=5
  \pgfmathsetmacro{\N}{5}
  \addplot+[summand=free,ultra thick, smooth,
          color=HotPink3,domain=0:4.9] {
      \N^\N/\N! * (\x/\N)^(\N+1)/(1-\x/\N)^2 *%
      1/(sum(\N-1,\x) + \x^\N/\N! * 1/(1-\x/\N))}
      node [pos=0.1,anchor=north west] {$N=5$};

  \draw[color=black,dashed]
        (axis cs:3,0) -- (axis cs:3,10);



\end{axis}



\end{tikzpicture}


    \end{figure}
\end{frame}





\begin{frame}{\secname: \subsecname}
    \begin{lema}[Número medio usuarios en M/M/N]
        En un sistema M/M/N el tiempo medio
        de usuarios en el sistema es
        \begin{equation}
            \mathbb{E}[N(t)]=
            \mathbb{E}[Q(t)]+A
        \end{equation}
    \end{lema}
    \vfill
    \textit{Demostración}:
    con Little sabemos
    $\mathbb{E}[W(t)]=\tfrac{1}{\lambda}
    \mathbb{E}[Q(t)]$, y también sabemos que
    \begin{equation*}
        \mathbb{E}[T(t)]
        =\mathbb{E}[W(t)]+\mathbb{E}[t_s]
        = \frac{1}{\lambda}\mathbb{E}[Q(t)]
        + \frac{1}{\mu}
    \end{equation*}
    y usando Little de nuevo
    ($\mathbb{E}[N(t)]=\mathbb{E}[T(t)]\lambda$)
    llegamos a
    \begin{equation*}
        \mathbb{E}[N(t)]=
        \mathbb{E}[Q(t)]+A
    \end{equation*}
\end{frame}




\begin{frame}[allowframebreaks]
        \frametitle{Referencias}
        \bibliographystyle{amsalpha}
        \bibliography{refs.bib}
\end{frame}





\end{document}
