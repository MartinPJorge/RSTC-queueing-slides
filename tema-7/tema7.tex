\documentclass[xcolor={x11names}]{beamer}
\usetheme{Madrid}

\usepackage{amssymb}
\usepackage{ulem}
\usepackage[utf8]{inputenc}
\usepackage{mathtools}
\usepackage{multicol}
%\usepackage[x11names]{xcolor}
\usefonttheme{professionalfonts}


% Subfigures
\usepackage{caption}
\usepackage{subcaption}

% Check/cross mark
\usepackage{pifont}% http://ctan.org/pkg/pifont
\newcommand{\cmark}{\ding{51}}%
\newcommand{\xmark}{\ding{55}}%


% License
\usepackage[
    type={CC},
    modifier={by-nc-sa},
    version={4.0},
    imagewidth=4pt
]{doclicense}




% Change base colour beamer@blendedblue (originally RGB: 0.2,0.2,0.7)
\colorlet{beamer@blendedblue}{DarkSeaGreen4}







%% MATH commands
\DeclareMathOperator{\Var}{Var}


%% THEOREMS
%\newtheorem{theorem}{Theorem}
\newtheorem{thm}{Teorema}[section] % the main one
% Definición
%\theoremstyle{definition}
\newtheorem{definicion}{Definición}[section]
\newtheorem{lema}{Lema}[section]



%% PFGplots %%
\usepackage{pgfplots}

%% Exponential distribution
\pgfmathdeclarefunction{exponential}{1}{%
  \pgfmathparse{(#1)*exp(-#1*x)}%
}
\pgfmathdeclarefunction{exponentialcdf}{1}{%
  \pgfmathparse{1-exp(-#1*x)}%
}

%% Poisson distribution
\pgfmathdeclarefunction{poiss}{1}{%
  \pgfmathparse{(#1^x)*exp(-#1)/(x!)}%
}

%% Normal distribution (#1=mu, #2=sigma)
% John D. Cook approx. https://tex.stackexchange.com/a/124629
\pgfmathdeclarefunction{normalcdf}{2}{%
  \pgfmathparse{1/(1 + exp(-0.07056*((x-#1)/#2)^3 - 1.5976*(x-#1)/#2))}%
}




\newcommand{\red}[1]{{\color{red}#1}}
\newcommand{\blue}[1]{{\color{blue}#1}}

%%%%%%%%%%
%% TIKZ %%
%%%%%%%%%%
\usepackage{tikz}
\usepackage{animate}
\usetikzlibrary{positioning}
\usetikzlibrary{shapes,arrows, positioning, calc}
\usetikzlibrary{overlay-beamer-styles}
\usetikzlibrary{chains,shapes.multipart}
\usetikzlibrary{scopes}
\usetikzlibrary{automata}
\usetikzlibrary{positioning}  %                 ...positioning nodes
\usetikzlibrary{arrows}       %                 ...customizing arrows
\usetikzlibrary{intersections}


%%%%%%%%%
%% PGF %%
%%%%%%%%%
\usepgfplotslibrary{fillbetween}


%%% Insert section name before the section %%%
\AtBeginSection[]{
  \begin{frame}
  \vfill
  \centering
  \begin{beamercolorbox}[sep=8pt,center,shadow=true,rounded=true]{title}
    \usebeamerfont{title}\insertsectionhead\par%
  \end{beamercolorbox}
  \vfill
  \end{frame}
}



\title[Tema 6]{Tema 6: Teletráfico en redes de telecomunicaciones}
%% \subtitle{Redes y Servicios de Telecomunicaciones (RSTC)\\
%% Grado en Ingeniería de Tecnologías y Servicios de Telecomunicación}
%\author{M. Saiful Bari\inst{1} \and Mr X\inst{2}}

\titlegraphic{%
\includegraphics[height=2.5cm]{../tema-5/figs/RSTC-grande.png}\\%
\doclicenseIcon {\tiny \hspace{1em}\doclicenseText}\\%
\href{https://github.com/MartinPJorge/RSTC-queueing-slides}{\includegraphics[height=1cm]{../tema-5/figs/github-logo.png}}%
}


\author{\textcolor{white}{RSTC curso 2022-2023}}
%\author{Jorge Martín Pérez\inst{1}}
%\institute{
%    \inst{1}
%    Departamento de Ingeniería Telemática, Universidad Politécnica de Madrid
%}

\date{\today}







%%%%%%%%%%%%%%%%%%%%
%%% SLIDES START %%%
%%%%%%%%%%%%%%%%%%%%
\begin{document}


%%% TITLE %%%
\frame{\titlepage}


\begin{frame}{Contenido}
    \tableofcontents
\end{frame}


\section{Sistema M/M/N}
\begin{frame}{\secname}
    Un sistema de cola única
    con $t_l\sim Exp(\lambda)$
    y N servidores en paralelo
    con $t_s\sim Exp(\mu)$.
    \begin{figure}
        \input{figs/mmn.tex}
    \end{figure}
\end{frame}


\subsection{Cadena de Markov}
\begin{frame}{\secname: \subsecname}
    Un sistema M/M/N es un proceso
    estocástico Markoviano\footnote{
    El tiempo de estancia es
    exponencial no homogéneo.}.
    \begin{figure}
        \resizebox{!}{.2\textwidth}{%
            \begin{tikzpicture}
   
% Estados
\node[state,minimum size=4em] (1) {$0$};
\node[state,minimum size=4em] (2) [right=of 1] {$1$};
\node[state,minimum size=4em,fill=HotPink3!20] (3) [right=of 2] {$2$};
\node[state,minimum size=4em] (4) [right=of 3] {$3$};
\node[minimum size=3em] (5) [right=of 4] {$\cdots$};
\node[state,minimum size=4em] (N) [right=of 5] {$N$};
\node[state,minimum size=4em] (N1) [right=of N] {$N+1$};
\node[minimum size=3em] (N2) [right=of N1] {$\cdots$};


% Transiciones
\draw [->,thick] (1.north east) to [bend left=55]  node[above] {$\lambda$}  (2.north west);
\draw [->,thick] (2.south west) to [bend left=55]  node[below] {$\mu$}  (1.south east);

\draw [->,thick] (2.north east) to [bend left=55]  node[above] {$\lambda$}  (3.north west);
\draw [->,thick] (3.south west) to [bend left=55]  node[below] {$2\mu$}  (2.south east);

\draw [->,thick] (3.north east) to [bend left=55]  node[above] {$\lambda$}  (4.north west);
\draw [->,thick] (4.south west) to [bend left=55]  node[below] {$3\mu$}  (3.south east);

\draw [->,thick] (4.north east) to [bend left=55]  node[above] {$\lambda$}  (5.north west);
\draw [->,thick] (5.south west) to [bend left=55]  node[below] {$4\mu$}  (4.south east);


\draw [->,thick] (5.north east) to [bend left=55]  node[above] {$\lambda$}  (N.north west);
\draw [->,thick] (N.south west) to [bend left=55]  node[below] {$N\mu$}  (5.south east);


\draw [->,thick] (N.north east) to [bend left=55]  node[above] {$\lambda$}  (N1.north west);
\draw [->,thick] (N1.south west) to [bend left=55]  node[below] {$N\mu$}  (N.south east);


\draw [->,thick] (N1.north east) to [bend left=55]  node[above] {$\lambda$}  (N2.north west);
\draw [->,thick] (N2.south west) to [bend left=55]  node[below] {$N\mu$}  (N1.south east);

\draw[dashed] (N.north) -- ($(0,1.5)+(N.north)$);
\draw[dashed] (N.south) -- ($(0,-1.5)+(N.south)$);
    
\end{tikzpicture}

        }
    \end{figure}
    Sus tasas de transición no son
    homogéneas $q_{i,j}=i\mu,\ i\leq N$.
    \begin{figure}
        \resizebox{!}{.2\textwidth}{%
            \input{figs/mmn-highlight.tex}
        }
    \end{figure}
\end{frame}



\subsection{Ecuaciones de equilibrio}
\begin{frame}{\secname: \subsecname}
    \begin{figure}
        \resizebox{!}{.2\textwidth}{%
            \begin{tikzpicture}
   
% Estados
\node[state,minimum size=4em] (1) {$0$};
\node[state,minimum size=4em] (2) [right=of 1] {$1$};
\node[state,minimum size=4em,fill=HotPink3!20] (3) [right=of 2] {$2$};
\node[state,minimum size=4em] (4) [right=of 3] {$3$};
\node[minimum size=3em] (5) [right=of 4] {$\cdots$};
\node[state,minimum size=4em] (N) [right=of 5] {$N$};
\node[state,minimum size=4em] (N1) [right=of N] {$N+1$};
\node[minimum size=3em] (N2) [right=of N1] {$\cdots$};


% Transiciones
\draw [->,thick] (1.north east) to [bend left=55]  node[above] {$\lambda$}  (2.north west);
\draw [->,thick] (2.south west) to [bend left=55]  node[below] {$\mu$}  (1.south east);

\draw [->,thick] (2.north east) to [bend left=55]  node[above] {$\lambda$}  (3.north west);
\draw [->,thick] (3.south west) to [bend left=55]  node[below] {$2\mu$}  (2.south east);

\draw [->,thick] (3.north east) to [bend left=55]  node[above] {$\lambda$}  (4.north west);
\draw [->,thick] (4.south west) to [bend left=55]  node[below] {$3\mu$}  (3.south east);

\draw [->,thick] (4.north east) to [bend left=55]  node[above] {$\lambda$}  (5.north west);
\draw [->,thick] (5.south west) to [bend left=55]  node[below] {$4\mu$}  (4.south east);


\draw [->,thick] (5.north east) to [bend left=55]  node[above] {$\lambda$}  (N.north west);
\draw [->,thick] (N.south west) to [bend left=55]  node[below] {$N\mu$}  (5.south east);


\draw [->,thick] (N.north east) to [bend left=55]  node[above] {$\lambda$}  (N1.north west);
\draw [->,thick] (N1.south west) to [bend left=55]  node[below] {$N\mu$}  (N.south east);


\draw [->,thick] (N1.north east) to [bend left=55]  node[above] {$\lambda$}  (N2.north west);
\draw [->,thick] (N2.south west) to [bend left=55]  node[below] {$N\mu$}  (N1.south east);

\draw[dashed] (N.north) -- ($(0,1.5)+(N.north)$);
\draw[dashed] (N.south) -- ($(0,-1.5)+(N.south)$);
    
\end{tikzpicture}

        }
    \end{figure}
    Con $i<N$ tenemos
    \begin{equation*}
        \lambda\pi_{i-1}
        +(i+1)\mu\pi_{i+1}
        = \pi_i(\lambda+i\mu)
    \end{equation*}
    pero con $i\geq N$ tenemos
    \begin{equation*}
        \lambda\pi_{i-1}
        +N\mu\pi_{i+1}
        = \pi_i(\lambda+N\mu)
    \end{equation*}
\end{frame}



\begin{frame}[allowframebreaks]
        \frametitle{Referencias}
        \bibliographystyle{amsalpha}
        \bibliography{refs.bib}
\end{frame}





\end{document}
