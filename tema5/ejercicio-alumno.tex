\documentclass{upmassignment}
\usepackage[spanish]{babel}
\usepackage{ifthen}
\usepackage{amsmath}
\usepackage{amsfonts}



% Para mostrar/ocultar soluciones
\newboolean{show}
%\setboolean{show}{true}
\setboolean{show}{false}
\usepackage{environ}
\NewEnviron{solucion}{
  \ifshow
      \begin{answer}\BODY\end{answer}
  \fi}






\coursetitle{Creating assignments}
\courselabel{RSTC}
\exercisesheet{Ejercicio de Alumno}{Tema 5}
\student{\ }%
\semester{Segundo Semestre}
\date{\today}
\university{Universidad Politécnica de Madrid}
\school{Departamento de Ingeniería de Sistemas Telemáticos}
%\usepackage[pdftex]{graphicx}
%\usepackage{subfigure}


\setlength{\textwidth}{5.0in}
\linespread{1.3}
\renewcommand{\PB}{{\bfseries Problema}}















\begin{document}

Considere un router de cola infinito cuyo tiempo de enrutado tiene
una varianza $V$. El tiempo entre llegadas al router sigue una
distribución exponencial de tasa $\lambda$.
El tiempo de servicio del router
tiene una media de $\tfrac{1}{\mu}$\,\textrm{sec}.


\begin{problemlist}
    \pbitem ¿Qué carga aguanta el router para que
    un paquete tarde, en media, menos de $x$\,\textrm{sec} en salir? Exprese la solución en función de $V,x,\mu$.

    \begin{solucion}
        \input{../../RSTC-solutions/tema-5/solucion-problema1.tex}
    \end{solucion}

    \pbitem Suponga que el tiempo
    de servicio del router es exponencial,
    el 10\% del tráfico
    procesado sale de la red y un 90\%
    del tráfico restante pasa a otro
    router con el doble de tasa de servicio
    y tiempo de servicio exponencial.
    ¿Qué carga debe tener el primer
    router para que la probabilidad
    de tener $K=(0,0)$ paquetes en sendos
    routers sea mayor a un 10\%?


    \begin{solucion}
        \input{../../RSTC-solutions/tema-5/solucion-problema2.tex}
    \end{solucion}

    \pbitem Suponga que el segundo router
    sigue un tiempo de servicio
    determinista.
    ¿Cuál sería el tiempo medio en
    atravesar los dos routers?

    \begin{solucion}
        \input{../../RSTC-solutions/tema-5/solucion-problema3.tex}
    \end{solucion}

\end{problemlist}

\end{document}


